\documentclass[fleqn]{article}
\oddsidemargin 0.0in
\textwidth 6.0in
\thispagestyle{empty}
\usepackage{import}
\usepackage{amsmath}
\usepackage{graphicx}
\usepackage{flexisym}
\usepackage{amssymb}
\usepackage{bigints} 
\usepackage[english]{babel}
\usepackage[utf8x]{inputenc}
\usepackage{float}
\usepackage[colorinlistoftodos]{todonotes}

\definecolor{hwColor}{HTML}{AD53BA}

\begin{document}

  \begin{titlepage}

    \newcommand{\HRule}{\rule{\linewidth}{0.5mm}}

    \center


    \textsc{\LARGE Arizona State University}\\[1.5cm]

    \textsc{\LARGE Linear Algebra }\\[1.5cm]


    \begin{figure}
      \includegraphics[width=\linewidth]{asu.png}
    \end{figure}


    \HRule \\[0.4cm]
    { \huge \bfseries Quiz Four}\\[0.4cm] 
    \HRule \\[1.5cm]

    \textbf{Behnam Amiri}

    \bigbreak

    \textbf{Prof: Sergei Suslov}

    \bigbreak


    \textbf{{\large \today}\\[2cm]}

    \vfill

  \end{titlepage}

  \begin{enumerate}
    \item (10 points) Find the angle between the vectors $u=(-1, 2, -3, 4)$ and $v=(1, -3, 2, -4)$ and 
    verify the the Cauchy-Schwarz inequality, $|\left\langle u, v\right\rangle| \leq ||u|| ~ ||v||$, for 
    them using the standard inner product in $\mathbb{R}^4: ~ \left\langle u,v\right\rangle =u_1 v_1+u_2 v_2+u_3 v_3+u_4 v_4$.


    \item (10 points) Verify that the set of three vectors $v_1=(0, 1, 0), v_2=(-\dfrac{5}{13}, 0, \dfrac{12}{13}),$ and \\
    $v_3=(\dfrac{12}{13}, 0, \dfrac{5}{13})$ forms an orthonormal basis for $\mathbb{R}^3$ with the standard Euclidean inner
    product. Express the vector $u=(1, 2, 3)$ ) as a linear combination of the vectors in the given set: 
    $u=c_1 v_1+c_2 v_2+c_3 v_3$ and find the coordinates $c_1, c_2$ and $c_3$.

    \item (10 points) Prove the Cauchy-Schwarz inequality, $|\left\langle u, v\right\rangle| \leq ||u|| ~ ||v||$,
    for two arbitrary vectors in an abstract inner product space $V$. Write this inequality explicitly in $\mathbb{R}^4$
    with the standard inner product: $\left\langle u,v\right\rangle =u_1 v_1+u_2 v_2+u_3 v_3+u_4 v_4$.


    \item (10 points) Convert the basis $v_1=(1, -1, 0), v_2=(0, 1, -1),$ and $v_3=(-1, 1, -1)$ for $\mathbb{R}^3$ into
    an orthonormal basis, using the Gram-Schmidt process and the standard inner product in $\mathbb{R}^3$.


    \item (Extra credit, 5 points) The linear transformation, $L \left[p(x)\right]=\dfrac{d}{dx}p(x)+p(0)$, maps 
    a polynomial $p(x)$ of degree $\leq 2$  into a polynomial of degree $\leq 1$, namely, $L: P_2 \rightarrow P_1$.
    Find the matrix representation of $L$ with respect to the ordered bases $\{ x^2, x, 1\}$ and $\{ x, 1\}$.

    \item (10 points) Show that for any two vectors $u$ and $v$ in an inner product space $V$,
    $$||u+v||^2+||u-v||=2\left(||u||^2+||v||^2\right)$$
    Give a geometric interpretation of this result for the vector space $\mathbb{R}^2$.

  \end{enumerate}

\end{document}
