\documentclass[fleqn]{article}
\oddsidemargin 0.0in
\textwidth 6.0in
\thispagestyle{empty}
\usepackage{import}
\usepackage{amsmath}
\usepackage{graphicx}
\usepackage{flexisym}
\usepackage{amssymb}
\usepackage{bigints} 
\usepackage[english]{babel}
\usepackage[utf8x]{inputenc}
\usepackage{float}
\usepackage[colorinlistoftodos]{todonotes}

\definecolor{hwColor}{HTML}{AD53BA}

\begin{document}

  \begin{titlepage}

    \newcommand{\HRule}{\rule{\linewidth}{0.5mm}}

    \center


    \textsc{\LARGE Arizona State University}\\[1.5cm]

    \textsc{\LARGE Linear Algebra }\\[1.5cm]


    \begin{figure}
      \includegraphics[width=\linewidth]{asu.png}
    \end{figure}


    \HRule \\[0.4cm]
    { \huge \bfseries Homework Four}\\[0.4cm] 
    \HRule \\[1.5cm]

    \textbf{Behnam Amiri}

    \bigbreak

    \textbf{Prof: Sergei Suslov}

    \bigbreak


    \textbf{{\large \today}\\[2cm]}

    \vfill

  \end{titlepage}

  \begin{enumerate}
    \item (3 points) Find a basis for the row space, a basis for a column space, and a basis for the null
    space of the following matrix:
    $$
      \begin{pmatrix}
        1 & 3 & -2 & 1
        \\
        2 & 1 & 3 & 2
        \\
        3 & 4 & 5 & 5
      \end{pmatrix}
    $$

      \textcolor{hwColor}{
        We start by putting the matrix in row echelon form. \\
        \\
        $
          A=\begin{pmatrix}
            1 & 3 & -2 & 1
            \\
            2 & 1 & 3 & 2
            \\
            3 & 4 & 5 & 5
          \end{pmatrix}
          \\
          \\
          \rule{15cm}{1pt}
          \\
          \\
          -2R_1+R_2 \rightarrow R_2
          \\
          \\
          =\begin{pmatrix}
            1 & 3 & -2 & 1
            \\
            0 & -5 & 7 & 0
            \\
            3 & 4 & 5 & 5
          \end{pmatrix}
          \\
          \\
          \rule{15cm}{1pt}
          \\
          \\
          -3R_1+R_3 \rightarrow R_3
          \\
          \\
          =\begin{pmatrix}
            1 & 3 & -2 & 1
            \\
            0 & -5 & 7 & 0
            \\
            0 & -5 & 11 & 2
          \end{pmatrix}
          \\
          \\
          \rule{15cm}{1pt}
          \\
          \\
          -\dfrac{1}{5}R_2 \rightarrow R_2
          \\
          \\
          =\begin{pmatrix}
            1 & 3 & -2 & 1
            \\
            0 & 1 & -\dfrac{7}{5} & 0
            \\
            0 & -5 & 11 & 2
          \end{pmatrix}
          \\
          \\
          \rule{15cm}{1pt}
          \\
          \\
          5R_2+R_3 \rightarrow R_3
          \\
          \\
          =\begin{pmatrix}
            1 & 3 & -2 & 1
            \\
            0 & 1 & -\dfrac{7}{5} & 0
            \\
            0 & 0 & 4 & 2
          \end{pmatrix}
          \\
          \\
          \rule{15cm}{1pt}
          \\
          \\
          \dfrac{1}{4}R_3 \rightarrow R_3
          \\
          \\
          =\begin{pmatrix}
            1 & 3 & -2 & 1
            \\
            0 & 1 & -\dfrac{7}{5} & 0
            \\
            0 & 0 & 1 & \dfrac{1}{2}
          \end{pmatrix}
          \\
          \\
          \rule{15cm}{1pt}
          \\
          \\
          \dfrac{7}{5}R_3+R_2 \rightarrow R_2
          \\
          \\
          =\begin{pmatrix}
            1 & 3 & -2 & 1
            \\
            0 & 1 & 0 & \dfrac{7}{10}
            \\
            0 & 0 & 1 & \dfrac{1}{2}
          \end{pmatrix}
          \\
          \\
          \rule{15cm}{1pt}
          \\
          \\
          2R_3+R_1 \rightarrow R_1
          \\
          \\
          =\begin{pmatrix}
            1 & 3 & 0 & 2
            \\
            0 & 1 & 0 & \dfrac{7}{10}
            \\
            0 & 0 & 1 & \dfrac{1}{2}
          \end{pmatrix}
          \\
          \\
          \rule{15cm}{1pt}
          \\
          \\
          -3R_2+R_1 \rightarrow R_1
          \\
          \\
          \therefore ~~~~~ A=\begin{pmatrix}
            1 & 0 & 0 & -\dfrac{1}{10}
            \\
            \\
            0 & 1 & 0 & \dfrac{7}{10}
            \\
            \\
            0 & 0 & 1 & \dfrac{1}{2}
          \end{pmatrix}
        $
        \\
        \\
        Row operations do not change the row space, so the rows of the matrix at the end have the same span
        as our given matrix. Furthermore, the nonzero rows of a matrix in row echelon form are linearly independent.
        Therefore, the row space has a basis: \\
        \\
        $\{  \left[1, 0, 0, -\dfrac{1}{10}\right], \left[0, 1, 0, \dfrac{7}{10}\right], \left[0, 0, 1, \dfrac{1}{2}\right] \}$
        \\
        \\
        From the final matrix, it is clear that the first, second, and third columns 
        of the matrix are the pivot columns. Thus, the first, second, and third columns
        of the original matrix form a basis for the column space.
        \\
        \\
        $
          \{ 
            \begin{bmatrix}
              1
              \\
              2
              \\
              3
            \end{bmatrix},
            \begin{bmatrix}
              3
              \\
              1
              \\
              4
            \end{bmatrix},
            \begin{bmatrix}
              -2
              \\
              3
              \\
              5
            \end{bmatrix} 
          \}
          \\
          \\
        $
        Finally, the row space and column space each have bases with three vectors, 
        so they have dimension three. Therefore, the rank of $A$ is $3$.
        \\
        \\
        We already found the reduced row echelon form of the of given matrix. Now, we can write the following 
        system of euqations:
        \\
        \\
        $
          \begin{cases}
            x_1-\dfrac{1}{10}x_4=0
            \\
            \\
            x_2+\dfrac{7}{10}x_4=0
            \\
            \\
            x_3+\dfrac{1}{2}x_4=0
          \end{cases}
        $
        \\
        \\
        This system has infinity solutions. One solution can be written as the following:
        \\
        \\
        $
          \begin{cases}
            x_1=\dfrac{1}{10}
            \\
            \\
            x_2=-\dfrac{7}{10}
            \\
            \\
            x_3=-\dfrac{1}{2}
            \\
            \\
            x_4=1
          \end{cases}
        $
        \\
        \\
        Therefore the null space has a basis formed by the set 
        $
          \{ 
            \begin{pmatrix}
              \dfrac{1}{10}
              \\
              \\
              -\dfrac{7}{10}
              \\
              \\
              -\dfrac{1}{2}
              \\
              \\
              1
            \end{pmatrix}
          \}
        $.
      }
   

    % \item (2 points) Determine whether the following set form a subspace in $\mathbb{R}^4$:
    % $$
    %   \{ \left(x_1, x_2, x_3, x_4\right)^T \text{such that } x_1-x_2-x_3-x_4=0, ~ x_1+x_2+x_3-x_4=0 \} 
    % $$

    \item (2 points) Are the vectors $v_1=(5, -1, 7), ~ v_2=(-3, 2, -9),$ and $v_3=(1, -2, 4)$ linearly independent? Explain.
    
      \textcolor{hwColor}{
        These vectors will be linearly independent if the deteminent of matrix form by these vectors is not zero.
        \\
        \\
        $
          A=\begin{bmatrix}
            5 & -1 & 7
            \\
            -3 & 2 & -9 
            \\
            1 & -2 & 4
          \end{bmatrix}
          \\
          \\
          \\
          det(A)=\begin{vmatrix}
            5 & -1 & 7
            \\
            -3 & 2 & -9 
            \\
            1 & -2 & 4
          \end{vmatrix}=
          5(-1)^{1+1} \begin{vmatrix}
            2 & -9
            \\
            -2 & 4
          \end{vmatrix}
          +(-1)(-1)^{1+2} \begin{vmatrix}
            -3 & -9 
            \\
            1 & 4
          \end{vmatrix}
          +7(-1)^{1+1} \begin{vmatrix}
            -3 & 2
            \\
            1 & -2
          \end{vmatrix}
          \\
          \\
          \\
          \therefore ~~~~ det(A)=-25
        $
        \\
        \\
        Hence, these vectors are linearly independent.
      }

    \item (3 points) Let $v_1, v_2, v_3$ be linearly independence vectors in $\mathbb{R}^n$ and let
    $$
      u_1=v_1+v_2, ~~~~~ u_2=v_2+v_3, ~~~~~ u_3=v_3+v_1
    $$
    Are the vectors $u_1, u_2, u_3$ linearly independent? Prove your answer.

      \textcolor{hwColor}{
        Let's start off with: \\
        \\
        $
          c_1u_1+c_2u_2+c_3u_3=0
          \\
          \\
          c_1\left(v_1+v_2\right)+c_2\left(v_2+v_3\right)+c_3\left(v_3+v_1\right)=0
          \\
          \\
          \left(c_1+c_3\right)v_1+\left(c_1+c_2\right)v_2+\left(c_2+c_3\right)v_3=0
          \\
          \\
          \\
          \therefore ~~~~ \begin{cases}
            c_1+c_3=0
            \\
            c_1+c_2=0
            \\
            c_2+c_3=0
          \end{cases} \Longrightarrow \begin{pmatrix}
            1 & 0 & 1
            \\
            1 & 1 & 0
            \\
            0 & 1 & 1
          \end{pmatrix} \begin{pmatrix}
            c_1
            \\
            c_2
            \\
            c_3
          \end{pmatrix}=\begin{pmatrix}
            0
            \\
            0
            \\
            0
          \end{pmatrix}
          \\
          \\
          \\
          \begin{vmatrix}
            1 & 0 & 1
            \\
            1 & 1 & 0
            \\
            0 & 1 & 1
          \end{vmatrix} \neq 0 ~~~~ \checkmark
        $
        \\
        \\
        So the system of equation has only zero solution when $c_1=c_2=c_3=0$. Hence, $\{u_1, u_2, u_3\}$
        is linearly independent set.
      }

  \end{enumerate}

\end{document}
