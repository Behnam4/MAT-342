\documentclass[fleqn]{article}
\oddsidemargin 0.0in
\textwidth 6.0in
\thispagestyle{empty}
\usepackage{import}
\usepackage{amsmath}
\usepackage{graphicx}
\usepackage{flexisym}
\usepackage{amssymb}
\usepackage{bigints} 
\usepackage[english]{babel}
\usepackage[utf8x]{inputenc}
\usepackage{float}
\usepackage[colorinlistoftodos]{todonotes}

\definecolor{hwColor}{HTML}{AD53BA}

\begin{document}

  \begin{titlepage}

    \newcommand{\HRule}{\rule{\linewidth}{0.5mm}}

    \center


    \textsc{\LARGE Arizona State University}\\[1.5cm]

    \textsc{\LARGE Linear Algebra }\\[1.5cm]


    \begin{figure}
      \includegraphics[width=\linewidth]{asu.png}
    \end{figure}


    \HRule \\[0.4cm]
    { \huge \bfseries Homework Five}\\[0.4cm] 
    \HRule \\[1.5cm]

    \textbf{Behnam Amiri}

    \bigbreak

    \textbf{Prof: Sergei Suslov}

    \bigbreak


    \textbf{{\large \today}\\[2cm]}

    \vfill

  \end{titlepage}

  \begin{enumerate}
    \item (15 points) Solve the system of linear equations 
    $$
      \begin{cases}
        x+y+z-w=2
        \\
        x-y+z+w=6
        \\
        x+y+3z-2w=4
        \\
        -x+y-2z+w=-1
      \end{cases}
    $$

      \textcolor{hwColor}{
        We tackle this problem by finding the augmented matrix.
        \\
        \\
        $
          A=\left(\begin{array}{cccc|c}  
            1 & 1 & 1 & -1 & 2 \\  
            1 & -1 & 1 & 1 & 6 \\
            1 & 1 & 3 & -2 & 4 \\
            -1 & 1 & -2 & 1 & -1 
          \end{array}\right)
          \\
          \\
          \rule{15cm}{1pt}
          \\
          \\
          -R_1+R_2 \rightarrow R_2 
          \\
          \\
          =\left(\begin{array}{cccc|c}  
            1 & 1 & 1 & -1 & 2 \\  
            0 & -2 & 0 & 2 & 4 \\
            1 & 1 & 3 & -2 & 4 \\
            -1 & 1 & -2 & 1 & -1 
          \end{array}\right)
          \\
          \\
          \rule{15cm}{1pt}
          \\
          \\
          -R_1+R_3 \rightarrow R_3 
          \\
          \\
          =\left(\begin{array}{cccc|c}  
            1 & 1 & 1 & -1 & 2 \\  
            0 & -2 & 0 & 2 & 4 \\
            0 & 0 & 2 & -1 & 2 \\
            -1 & 1 & -2 & 1 & -1 
          \end{array}\right)
          \\
          \\
          \rule{15cm}{1pt}
          \\
          \\
          R_1+R_4 \rightarrow R_4 
          \\
          \\
          =\left(\begin{array}{cccc|c}  
            1 & 1 & 1 & -1 & 2 \\  
            0 & -2 & 0 & 2 & 4 \\
            0 & 0 & 2 & -1 & 2 \\
            0 & 2 & -1 & 0 & 1 
          \end{array}\right)
          \\
          \\
          \rule{15cm}{1pt}
          \\
          \\
          -\dfrac{1}{2}R_2 \rightarrow R_2 
          \\
          \\
          =\left(\begin{array}{cccc|c}  
            1 & 1 & 1 & -1 & 2 \\  
            0 & 1 & 0 & -1 & -2 \\
            0 & 0 & 2 & -1 & 2 \\
            0 & 2 & -1 & 0 & 1 
          \end{array}\right)
          \\
          \\
          \rule{15cm}{1pt}
          \\
          \\
          -2R_2+R_4 \rightarrow R_4 
          \\
          \\
          =\left(\begin{array}{cccc|c}  
            1 & 1 & 1 & -1 & 2 \\  
            0 & 1 & 0 & -1 & -2 \\
            0 & 0 & 2 & -1 & 2 \\
            0 & 0 & -1 & 2 & 5 
          \end{array}\right)
          \\
          \\
          \rule{15cm}{1pt}
          \\
          \\
          \dfrac{1}{2}R_3 \rightarrow R_3 
          \\
          \\
          =\left(\begin{array}{cccc|c}  
            1 & 1 & 1 & -1 & 2 \\  
            0 & 1 & 0 & -1 & -2 \\
            0 & 0 & 1 & -\dfrac{1}{2} & 1 \\
            0 & 0 & -1 & 2 & 5 
          \end{array}\right)
          \\
          \\
          \rule{15cm}{1pt}
          \\
          \\
          R_3+R_4 \rightarrow R_4 
          \\
          \\
          =\left(\begin{array}{cccc|c}  
            1 & 1 & 1 & -1 & 2 \\  
            0 & 1 & 0 & -1 & -2 \\
            0 & 0 & 1 & -\dfrac{1}{2} & 1 \\
            0 & 0 & 0 & \dfrac{3}{2} & 6 
          \end{array}\right)
          \\
          \\
          \rule{15cm}{1pt}
          \\
          \\
          \dfrac{2}{3} R_4 \rightarrow R_4 
          \\
          \\
          =\left(\begin{array}{cccc|c}  
            1 & 1 & 1 & -1 & 2 \\  
            0 & 1 & 0 & -1 & -2 \\
            0 & 0 & 1 & -\dfrac{1}{2} & 1 \\
            0 & 0 & 0 & 1 & 4 
          \end{array}\right)
          \\
          \\
          \rule{15cm}{1pt}
          \\
          \\
          \dfrac{1}{2} R_4+R_3 \rightarrow R_3 
          \\
          \\
          =\left(\begin{array}{cccc|c}  
            1 & 1 & 1 & -1 & 2 \\  
            0 & 1 & 0 & -1 & -2 \\
            0 & 0 & 1 & 0 & 3 \\
            0 & 0 & 0 & 1 & 4 
          \end{array}\right)
          \\
          \\
          \rule{15cm}{1pt}
          \\
          \\
          R_4+R_2 \rightarrow R_2 
          \\
          \\
          =\left(\begin{array}{cccc|c}  
            1 & 1 & 1 & -1 & 2 \\  
            0 & 1 & 0 & 0 & 2 \\
            0 & 0 & 1 & 0 & 3 \\
            0 & 0 & 0 & 1 & 4 
          \end{array}\right)
          \\
          \\
          \rule{15cm}{1pt}
          \\
          \\
          R_4+R_1 \rightarrow R_1 
          \\
          \\
          =\left(\begin{array}{cccc|c}  
            1 & 1 & 1 & 0 & 6 \\  
            0 & 1 & 0 & 0 & 2 \\
            0 & 0 & 1 & 0 & 3 \\
            0 & 0 & 0 & 1 & 4 
          \end{array}\right)
          \\
          \\
          \rule{15cm}{1pt}
          \\
          \\
          -R_3+R_1 \rightarrow R_1 
          \\
          \\
          =\left(\begin{array}{cccc|c}  
            1 & 1 & 0 & 0 & 3 \\  
            0 & 1 & 0 & 0 & 2 \\
            0 & 0 & 1 & 0 & 3 \\
            0 & 0 & 0 & 1 & 4 
          \end{array}\right)
          \\
          \\
          \rule{15cm}{1pt}
          \\
          \\
          -R_2+R_1 \rightarrow R_1 
          \\
          \\
          =\left(\begin{array}{cccc|c}  
            1 & 0 & 0 & 0 & 1 \\  
            0 & 1 & 0 & 0 & 2 \\
            0 & 0 & 1 & 0 & 3 \\
            0 & 0 & 0 & 1 & 4 
          \end{array}\right)
          \\
          \\
          \\
          \therefore ~~~~~ \begin{cases}
            x=1
            \\
            y=2
            \\
            z=3
            \\
            w=4
          \end{cases} ~~~~~ \checkmark
        $
      }

    \pagebreak

    \item 
    $$
      A=\begin{pmatrix}
        1 & 0 & 0
        \\
        1 & -1 & 0
        \\
        1 & 3 & 2
      \end{pmatrix} ~~~
      and ~~~
      B=\begin{pmatrix}
        2 & -1 & 2
        \\
        2 & -2 & 1
        \\
        1 & -1 & 3
      \end{pmatrix}
    $$
    
      \begin{itemize}
        \item (10 points) Find $AB, BA$ and $AB-BA$.
        \item (15 points) Does $A^{-1}$ exist? Does $B^{-1}$ exist? If so, find them.
      \end{itemize}

    \item (10 points) Find
    $$
      det \begin{pmatrix}
        1 & -3 & 2 & 0
        \\
        -1 & 4 & 3 & 1
        \\
        1 & 3 & -3 & 1
        \\
        0 & 2 & 1 & -1
      \end{pmatrix}
    $$

    \item (10 points) Are the vectors $v_1=(2, -1, 2), ~~ v_2=(2, -2, 1),$ and $v_3=(1, -1, 3)$ linearly independent?

    \item (10 points) Use Cramer’s rule to determine the solutions, if any exist, to the following system
    of equations.
    $$
      \begin{cases}
        2x-y+z=1
        \\
        x+2y-z=2
        \\
        -x+y+z=3
      \end{cases}
    $$
    

    \item (10 points) Verify the Cauchy–Schwarz inequality $|u.v| \leq ||u||.||v||$ for the 
    vectors $u=(1, -2, -3, 4)$ and $v=(-1, -2, 2, 1)$ using the standard inner product in $R^4$.
    Find the angle between these vectors.

    \item (15 points) Convert the basis $v_1=(1, -1, 0), v_2=(0, 1, -1), v_3=(1, 0, -1)$ for $R^3$ into 
    an orthonormal basis, using the Gram–Schmidt process and the standard inner product in $R^3$.

    \item Find the eigenvalues and associated eigenvectors of a given matrix $A$.
      \begin{itemize}
        \item (10 points) $A=\begin{pmatrix}
          3 & -1
          \\
          1 & 4
        \end{pmatrix}$.

        \item (15 points) $A=\begin{pmatrix}
          1 & 2 & 1 
          \\
          0 & 3 & 1 
          \\
          0 & 5 & -1
        \end{pmatrix}$
      \end{itemize}


    \item (15 points) Find both a basis for the row space and a basis for the column space of the
    matrix
    $$
      \begin{pmatrix}
        1 & 3 & -1 & -1
        \\
        2 & 1 & -2 & 0
        \\
        3 & 4 & -3 & 2
      \end{pmatrix}
    $$
    What is the rank of this matrix?


    \item (5 points each) Let $L: \mathbb{R}^2 \rightarrow \mathbb{R}^3$ be a function given by $L(x,y)=(x-y, x+2y, 3x-2y)$.
      \begin{itemize}
        \item Show that $L$ is a linear transformation.
        \item Find the kernel of $L$.
        \item Find a basis for the range of the transformation $L$.
        \item Find the standard matrix representation of the transformation $L$. (Use the standard bases for $\mathbb{R}^2 \rightarrow \mathbb{R}^3$.)
      \end{itemize}


    \item (15 points) (The Pythagorean Law) Prove that if $\mathbf{u}$ and $\mathbf{v}$ are orthogonal vectors in an inner
    product space $V$, then $||\mathbf{u}+\mathbf{v}||^2=||\mathbf{u}||^2+||\mathbf{v}||^2$.


    \item (15 points) Write the explicit forms of the Pythagorean Law and the orthogonality relation
    in the vector space of continuous functions $\mathsf{C}\left[a, b\right]$  with inner product defined by:

    $(f,g)=\bigints\limits_{a}^{b} f(x) g(x) dx$.



    \item (15 points) Let $A$ be an $n \times n$ invertible matrix, let $\lambda \neq 0$ be an eigenvalue of $A$,
    and let $x$ be an eigenvector belonging to $\lambda$: $A x=\lambda x$. Use mathematical induction to show that
    $\lambda^{-m}$ is an eigenvalue of $A^{-m}=(A^{-1})^m$ and $x$ is an eigenvector of $A^{-m}$ belonging to $\lambda^{-m}$.  

  \end{enumerate}

\end{document}
