\documentclass[fleqn]{article}
\oddsidemargin 0.0in
\textwidth 6.0in
\thispagestyle{empty}
\usepackage{import}
\usepackage{amsmath}
\usepackage{graphicx}
\usepackage{flexisym}
\usepackage{amssymb}
\usepackage{bigints} 
\usepackage[english]{babel}
\usepackage[utf8x]{inputenc}
\usepackage{float}
\usepackage[colorinlistoftodos]{todonotes}

\definecolor{hwColor}{HTML}{7E368D}

\begin{document}

  \begin{titlepage}

    \newcommand{\HRule}{\rule{\linewidth}{0.5mm}}

    \center


    \textsc{\LARGE Arizona State University}\\[1.5cm]

    \textsc{\LARGE Linear Algebra }\\[1.5cm]


    \begin{figure}
      \includegraphics[width=\linewidth]{asu.png}
    \end{figure}


    \HRule \\[0.4cm]
    { \huge \bfseries Homework Five}\\[0.4cm] 
    \HRule \\[1.5cm]

    \textbf{Behnam Amiri}

    \bigbreak

    \textbf{Prof: Sergei Suslov}

    \bigbreak


    \textbf{{\large \today}\\[2cm]}

    \vfill

  \end{titlepage}

  \begin{enumerate}
    \item (15 points) Solve the system of linear equations 
    $$
      \begin{cases}
        x+y+z-w=2
        \\
        x-y+z+w=6
        \\
        x+y+3z-2w=4
        \\
        -x+y-2z+w=-1
      \end{cases}
    $$

      \textcolor{hwColor}{
        We tackle this problem by finding the augmented matrix.
        \\
        \\
        $
          A=\left(\begin{array}{cccc|c}  
            1 & 1 & 1 & -1 & 2 \\  
            1 & -1 & 1 & 1 & 6 \\
            1 & 1 & 3 & -2 & 4 \\
            -1 & 1 & -2 & 1 & -1 
          \end{array}\right)
          \\
          \\
          \rule{15cm}{1pt}
          \\
          \\
          -R_1+R_2 \rightarrow R_2 
          \\
          \\
          =\left(\begin{array}{cccc|c}  
            1 & 1 & 1 & -1 & 2 \\  
            0 & -2 & 0 & 2 & 4 \\
            1 & 1 & 3 & -2 & 4 \\
            -1 & 1 & -2 & 1 & -1 
          \end{array}\right)
          \\
          \\
          \rule{15cm}{1pt}
          \\
          \\
          -R_1+R_3 \rightarrow R_3 
          \\
          \\
          =\left(\begin{array}{cccc|c}  
            1 & 1 & 1 & -1 & 2 \\  
            0 & -2 & 0 & 2 & 4 \\
            0 & 0 & 2 & -1 & 2 \\
            -1 & 1 & -2 & 1 & -1 
          \end{array}\right)
          \\
          \\
          \rule{15cm}{1pt}
          \\
          \\
          R_1+R_4 \rightarrow R_4 
          \\
          \\
          =\left(\begin{array}{cccc|c}  
            1 & 1 & 1 & -1 & 2 \\  
            0 & -2 & 0 & 2 & 4 \\
            0 & 0 & 2 & -1 & 2 \\
            0 & 2 & -1 & 0 & 1 
          \end{array}\right)
          \\
          \\
          \rule{15cm}{1pt}
          \\
          \\
          -\dfrac{1}{2}R_2 \rightarrow R_2 
          \\
          \\
          =\left(\begin{array}{cccc|c}  
            1 & 1 & 1 & -1 & 2 \\  
            0 & 1 & 0 & -1 & -2 \\
            0 & 0 & 2 & -1 & 2 \\
            0 & 2 & -1 & 0 & 1 
          \end{array}\right)
          \\
          \\
          \rule{15cm}{1pt}
          \\
          \\
          -2R_2+R_4 \rightarrow R_4 
          \\
          \\
          =\left(\begin{array}{cccc|c}  
            1 & 1 & 1 & -1 & 2 \\  
            0 & 1 & 0 & -1 & -2 \\
            0 & 0 & 2 & -1 & 2 \\
            0 & 0 & -1 & 2 & 5 
          \end{array}\right)
          \\
          \\
          \rule{15cm}{1pt}
          \\
          \\
          \dfrac{1}{2}R_3 \rightarrow R_3 
          \\
          \\
          =\left(\begin{array}{cccc|c}  
            1 & 1 & 1 & -1 & 2 \\  
            0 & 1 & 0 & -1 & -2 \\
            0 & 0 & 1 & -\dfrac{1}{2} & 1 \\
            0 & 0 & -1 & 2 & 5 
          \end{array}\right)
          \\
          \\
          \rule{15cm}{1pt}
          \\
          \\
          R_3+R_4 \rightarrow R_4 
          \\
          \\
          =\left(\begin{array}{cccc|c}  
            1 & 1 & 1 & -1 & 2 \\  
            0 & 1 & 0 & -1 & -2 \\
            0 & 0 & 1 & -\dfrac{1}{2} & 1 \\
            0 & 0 & 0 & \dfrac{3}{2} & 6 
          \end{array}\right)
          \\
          \\
          \rule{15cm}{1pt}
          \\
          \\
          \dfrac{2}{3} R_4 \rightarrow R_4 
          \\
          \\
          =\left(\begin{array}{cccc|c}  
            1 & 1 & 1 & -1 & 2 \\  
            0 & 1 & 0 & -1 & -2 \\
            0 & 0 & 1 & -\dfrac{1}{2} & 1 \\
            0 & 0 & 0 & 1 & 4 
          \end{array}\right)
          \\
          \\
          \rule{15cm}{1pt}
          \\
          \\
          \dfrac{1}{2} R_4+R_3 \rightarrow R_3 
          \\
          \\
          =\left(\begin{array}{cccc|c}  
            1 & 1 & 1 & -1 & 2 \\  
            0 & 1 & 0 & -1 & -2 \\
            0 & 0 & 1 & 0 & 3 \\
            0 & 0 & 0 & 1 & 4 
          \end{array}\right)
          \\
          \\
          \rule{15cm}{1pt}
          \\
          \\
          R_4+R_2 \rightarrow R_2 
          \\
          \\
          =\left(\begin{array}{cccc|c}  
            1 & 1 & 1 & -1 & 2 \\  
            0 & 1 & 0 & 0 & 2 \\
            0 & 0 & 1 & 0 & 3 \\
            0 & 0 & 0 & 1 & 4 
          \end{array}\right)
          \\
          \\
          \rule{15cm}{1pt}
          \\
          \\
          R_4+R_1 \rightarrow R_1 
          \\
          \\
          =\left(\begin{array}{cccc|c}  
            1 & 1 & 1 & 0 & 6 \\  
            0 & 1 & 0 & 0 & 2 \\
            0 & 0 & 1 & 0 & 3 \\
            0 & 0 & 0 & 1 & 4 
          \end{array}\right)
          \\
          \\
          \rule{15cm}{1pt}
          \\
          \\
          -R_3+R_1 \rightarrow R_1 
          \\
          \\
          =\left(\begin{array}{cccc|c}  
            1 & 1 & 0 & 0 & 3 \\  
            0 & 1 & 0 & 0 & 2 \\
            0 & 0 & 1 & 0 & 3 \\
            0 & 0 & 0 & 1 & 4 
          \end{array}\right)
          \\
          \\
          \rule{15cm}{1pt}
          \\
          \\
          -R_2+R_1 \rightarrow R_1 
          \\
          \\
          =\left(\begin{array}{cccc|c}  
            1 & 0 & 0 & 0 & 1 \\  
            0 & 1 & 0 & 0 & 2 \\
            0 & 0 & 1 & 0 & 3 \\
            0 & 0 & 0 & 1 & 4 
          \end{array}\right)
          \\
          \\
          \\
          \therefore ~~~~~ \begin{cases}
            x=1
            \\
            y=2
            \\
            z=3
            \\
            w=4
          \end{cases} ~~~~~ \checkmark
        $
      }

    \pagebreak

    \item Let
    $$
      A=\begin{pmatrix}
        1 & 0 & 0
        \\
        1 & -1 & 0
        \\
        1 & 3 & 2
      \end{pmatrix} ~~~
      and ~~~
      B=\begin{pmatrix}
        2 & -1 & 2
        \\
        2 & -2 & 1
        \\
        1 & -1 & 3
      \end{pmatrix}
    $$
    
      \begin{itemize}
        \item (10 points) Find $AB, BA$ and $AB-BA$.
        
          \textcolor{hwColor}{
            \\
            \\
            $
              AB=
              \begin{pmatrix}
                1 & 0 & 0
                \\
                1 & -1 & 0
                \\
                1 & 3 & 2
              \end{pmatrix}
              \begin{pmatrix}
                2 & -1 & 2
                \\
                2 & -2 & 1
                \\
                1 & -1 & 3
              \end{pmatrix}
              \\
              \\
              \\
              =\begin{pmatrix}
                (1 \times 2 )+( 0 \times 2 )+( 0 \times 1 ) & ( 1 \times -1 )+( 0 \times -2 )+( 0 \times -1 ) & ( 1 \times 2 )+( 0 \times 1 )+( 0 \times 3 )
                \\
                \\
                ( 1 \times 2 )+( -1 \times 2 )+( 0 \times 1 ) & ( 1 \times -1 )+( -1 \times -2 )+( 0 \times -1 ) & ( 1 \times 2 )+( -1 \times 1 )+( 0 \times 3 )
                \\
                \\
                ( 1 \times 2 )+( 3 \times 2 )+( 2 \times 1 ) & ( 1 \times -1 )+( 3 \times -2 )+( 2 \times -1 ) & ( 1 \times 2 )+( 3 \times 1)+( 2 \times 3 )
              \end{pmatrix}
              \\
              \\
              \\
              \therefore ~~~~ AB=\begin{pmatrix}
                2 & -1 & 2
                \\
                0 & 1 & 1
                \\
                10 & -9 & 11
              \end{pmatrix}
              \\
              \\
              \rule{15cm}{1pt}
              \\
              \\
              BA=\begin{pmatrix}
                2 & -1 & 2
                \\
                2 & -2 & 1
                \\
                1 & -1 & 3
              \end{pmatrix}\begin{pmatrix}
                1 & 0 & 0
                \\
                1 & -1 & 0
                \\
                1 & 3 & 2
              \end{pmatrix}
              \\
              \\
              \\
              =\begin{pmatrix}
                ( 2 \times 1 )+( -1 \times 1 )+( 2 \times 1 ) & ( 2 \times 0 )+( -1 \times -1 )+( 2 \times 3 ) & ( 2 \times 0 )+( -1 \times 0 )+( 2 \times 2 )
                \\
                \\
                ( 2 \times 1 )+( -2 \times 1 )+( 1 \times 1 ) & ( 2 \times 0 )+( -2 \times -1 )+( 1 \times 3 ) & ( 2 \times 0 )+( -2 \times 0 )+( 1 \times 2 )
                \\
                \\
                ( 1 \times 1 )+( -1 \times 1 )+( 3 \times 1 ) & ( 1 \times 0 )+( -1 \times -1 )+( 3 \times 3 ) & ( 1 \times 0 )+( -1 \times 0 )+( 3 \times 2 )
              \end{pmatrix}
              \\
              \\
              \\
              \therefore ~~~~ BA=\begin{pmatrix}
                3 & 7 & 4
                \\
                1 & 5 & 5 
                \\
                3 & 10 & 6
              \end{pmatrix}
              \\
              \\
              \rule{15cm}{1pt}
              \\
              \\
              AB-BA=\begin{pmatrix}
                2 & -1 & 2
                \\
                0 & 1 & 1
                \\
                10 & -9 & 11
              \end{pmatrix}-\begin{pmatrix}
                3 & 7 & 4
                \\
                1 & 5 & 5 
                \\
                3 & 10 & 6
              \end{pmatrix}
              =\begin{pmatrix}
                2-3 & -1-7 & 2-4
                \\
                0-1 & 1-5 & 1-5
                \\
                10-3 & -9-10 & 11-6
              \end{pmatrix}
              \\
              \\
              \\
              \therefore ~~~~ AB-BA=\begin{pmatrix}
                -1 & -8 & -2
                \\
                -1 & -4 & -4
                \\
                7 & -19 & 5
              \end{pmatrix}
              \\
              \\
            $
          }

        \item (15 points) Does $A^{-1}$ exist? Does $B^{-1}$ exist? If so, find them.

          \textcolor{hwColor}{
            \\
            If the determinant of a matrix is \textbf{NOT} zero, then the matrix has an inverse matrix. So let's find the 
            determinant of matrices $A$ and $B$.
            \\
            \\
            $
              A^{-1}=\dfrac{1}{det(A)} adj(A)
              \\
              \\
              \\
              det(A)=\begin{vmatrix}
                1 & 0 & 0
                \\
                1 & -1 & 0
                \\
                1 & 3 & 2
              \end{vmatrix}=1 (-1)^{1+1} \begin{vmatrix}
                -1 & 0
                \\
                3 & 2
              \end{vmatrix}
              +0(-1)^{1+2} \begin{vmatrix}
                1 & 0
                \\
                1 & 2
              \end{vmatrix}
              +0(-1)^{1+2} \begin{vmatrix}
                1 & -1
                \\
                1 & 3
              \end{vmatrix}
              \\
              \\
              =\left[(-1 \times 2)-(0 \times 3)\right]
              \\
              \\
              \\
              \therefore ~~~~ det(A)=-2 
              \\
              \\
              adj(A)=(A_{ij})^T=\begin{pmatrix}
                +\begin{vmatrix}
                  -1 & 0
                  \\
                  3 & 2
                \end{vmatrix} & -\begin{vmatrix}
                  1 & 0 
                  \\
                  1 & 2
                \end{vmatrix} & +\begin{vmatrix}
                  1 & -1 
                  \\
                  1 & 3
                \end{vmatrix}
                \\
                \\
                -\begin{vmatrix}
                  0 & 0
                  \\
                  3 & 2
                \end{vmatrix} & +\begin{vmatrix}
                  1 & 0 
                  \\
                  1 & 2
                \end{vmatrix} & -\begin{vmatrix}
                  1 & 0
                  \\
                  1 & 3
                \end{vmatrix}
                \\
                \\
                +\begin{vmatrix}
                  0 & 0
                  \\
                  -1 & 0
                \end{vmatrix} & -\begin{vmatrix}
                  1 & 0
                  \\
                  1 & 0
                \end{vmatrix} & +\begin{vmatrix}
                  1 & 0
                  \\
                  1 & -1
                \end{vmatrix}
              \end{pmatrix}^T
              \\
              \\
              \\
              =\begin{pmatrix}
                -2 & -2 & 4
                \\
                0 & 2 & -3
                \\
                0 & 0 & -1
              \end{pmatrix}^T \Longrightarrow adj(A)=\begin{pmatrix}
                -2 & 0 & 0
                \\
                -2 & 2 & 0
                \\
                4 & -3 & -1
              \end{pmatrix}
              \\
              \\
              \\
              A^{-1}=\dfrac{1}{det(A)} adj(A)=-\dfrac{1}{2} \begin{pmatrix}
                -2 & 0 & 0
                \\
                -2 & 2 & 0
                \\
                4 & -3 & -1
              \end{pmatrix}
              \\
              \\
              \\
              \therefore ~~~~ A^{-1}=\begin{pmatrix}
                1 & 0 & 0
                \\
                1 & -1 & 0
                \\
                -2 & \dfrac{3}{2} & \dfrac{1}{2}
              \end{pmatrix}
            $
            \\
            \\
            \textbf{Check:}
            \\
            \\
            $
              A^{-1}A=I \Longrightarrow \begin{pmatrix}
                1 & 0 & 0
                \\
                1 & -1 & 0
                \\
                -2 & \dfrac{3}{2} & \dfrac{1}{2}
              \end{pmatrix}\begin{pmatrix}
                1 & 0 & 0
                \\
                1 & -1 & 0
                \\
                1 & 3 & 2
              \end{pmatrix}=\begin{pmatrix}
                1 & 0 & 0
                \\
                0 & 1 & 0
                \\
                0 & 0 & 1
              \end{pmatrix} ~~~~~ \checkmark
            $
            \\
            \\
            $
              B^{-1}=\dfrac{1}{det(B)} adj(B)
              \\
              \\
              \\
              det(B)=\begin{vmatrix}
                2 & -1 & 2
                \\
                2 & -2 & 1
                \\
                1 & -1 & 3
              \end{vmatrix}
              =2(-1)^{1+1} \begin{vmatrix}
                -2 & 1
                \\
                -1 & 3
              \end{vmatrix}
              +(-1)(-1)^{1+2} \begin{vmatrix}
                2 & 1
                \\
                1 & 3
              \end{vmatrix}
              +2(-1)^{1+3} \begin{vmatrix}
                2 & -2
                \\
                1 & -1
              \end{vmatrix}
              \\
              \\
              =(2 \times -5)+( 1 \times 5 )+( 2 \times 0 )
              \\
              \\
              \\
              \therefore ~~~~ det(B)=-5
              \\
              \\
              \\
              adj(B)=(B_{ij})^T=\begin{pmatrix}
                +\begin{vmatrix}
                  -2 & 1
                  \\
                  -1 & 3
                \end{vmatrix} & -\begin{vmatrix}
                  2 & 1
                  \\
                  1 & 3
                \end{vmatrix} & +\begin{vmatrix}
                  2 & -2
                  \\
                  1 & -1
                \end{vmatrix}
                \\
                \\
                -\begin{vmatrix}
                  -1 & 2
                  \\
                  -1 & 3
                \end{vmatrix} & +\begin{vmatrix}
                  2 & 2
                  \\
                  1 & 3
                \end{vmatrix} & -\begin{vmatrix}
                  2 & -1
                  \\
                  1 & -1
                \end{vmatrix}
                \\
                \\
                +\begin{vmatrix}
                  -1 & 2
                  \\
                  -2 & 1
                \end{vmatrix} & -\begin{vmatrix}
                  2 & 2
                  \\
                  2 & 1
                \end{vmatrix} & +\begin{vmatrix}
                  2 & -1
                  \\
                  2 & -2
                \end{vmatrix}
              \end{pmatrix}^T
              \\
              \\
              \\
              =\begin{pmatrix}
                -5 & -5 & 0
                \\
                1 & 4 & 1
                \\
                3 & 2 & -2
              \end{pmatrix}^T \Longrightarrow adj(B)=\begin{pmatrix}
                -5 & 1 & 3
                \\
                -5 & 4 & 2
                \\
                0 & 1 & -2
              \end{pmatrix}
              \\
              \\
              \\
              B^{-1}=\dfrac{1}{det(B)} adj(B)=-\dfrac{1}{5} \begin{pmatrix}
                -5 & 1 & 3
                \\
                -5 & 4 & 2
                \\
                0 & 1 & -2
              \end{pmatrix} 
              \\
              \\
              \\
              \therefore ~~~~ B^{-1}=\begin{pmatrix}
                1 & -\dfrac{1}{5} & -\dfrac{3}{5}
                \\
                \\
                1 & -\dfrac{4}{5} & -\dfrac{2}{5}
                \\
                \\
                0 & -\dfrac{1}{5} & \dfrac{2}{5}
              \end{pmatrix}
            $
            \\
            \textbf{Check:}
            \\
            $
              B^{-1}B=I \Longrightarrow \begin{pmatrix}
                1 & -\dfrac{1}{5} & -\dfrac{3}{5}
                \\
                \\
                1 & -\dfrac{4}{5} & -\dfrac{2}{5}
                \\
                \\
                0 & -\dfrac{1}{5} & \dfrac{2}{5}
              \end{pmatrix}\begin{pmatrix}
                2 & -1 & 2
                \\
                2 & -2 & 1
                \\
                1 & -1 & 3
              \end{pmatrix}=\begin{pmatrix}
                1 & 0 & 0
                \\
                0 & 1 & 0
                \\
                0 & 0 & 1
              \end{pmatrix} ~~~~~ \checkmark
            $
          }

      \end{itemize}

    \item (10 points) Find
    $$
      det \begin{pmatrix}
        1 & -3 & 2 & 0
        \\
        -1 & 4 & 3 & 1
        \\
        1 & 3 & -3 & 1
        \\
        0 & 2 & 1 & -1
      \end{pmatrix}
    $$

      \textcolor{hwColor}{
        $
          \begin{vmatrix}
            1 & -3 & 2 & 0
            \\
            -1 & 4 & 3 & 1
            \\
            1 & 3 & -3 & 1
            \\
            0 & 2 & 1 & -1
          \end{vmatrix}
          \\
          \\
          \\
          =1(-1)^{1+1} \begin{vmatrix}
            4 & 3 & 1
            \\
            3 & -3 & 1
            \\
            2 & 1 & -1
          \end{vmatrix}  
          +(-3)(-1)^{1+2} \begin{vmatrix}
            -1 & 3 & 1
            \\
            1 & -3 & 1
            \\
            0 & 1 & -1
          \end{vmatrix}  
          +2(-1)^{1+3} \begin{vmatrix}
            -1 & 4 & 1
            \\
            1 & 3 & 1
            \\
            0 & 2 & -1
          \end{vmatrix}    
          +0(-1)^{1+4} \begin{vmatrix}
            -1 & 4 & 3
            \\
            1 & 3 & -3
            \\
            0 & 2 & 1
          \end{vmatrix}
          \\
          \\
          \\
          =1(-1)^{1+1} A+(-3)(-1)^{1+2}B+2(-1)^{1+3} C
          \\
          \\
          \rule{15cm}{1pt}
          \\
          \\
          A=\begin{vmatrix}
            4 & 3 & 1
            \\
            3 & -3 & 1
            \\
            2 & 1 & -1
          \end{vmatrix}
          =4(-1)^{1+1} \begin{vmatrix}
            -3 & 1
            \\
            1 & -1
          \end{vmatrix}
          +3(-1)^{1+2} \begin{vmatrix}
            3 & 1
            \\
            2 & -1
          \end{vmatrix}
          +1(-1)^{1+3} \begin{vmatrix}
            3 & -3
            \\
            2 & 1
          \end{vmatrix}
          \\
          \\
          \\
          =( 4 \times 2 )+ ( -3 \times -5 )+( 1 \times 9 )=32
          \\
          \\
          \rule{15cm}{1pt}
          \\
          \\
          B=\begin{vmatrix}
            -1 & 3 & 1
            \\
            1 & -3 & 1
            \\
            0 & 1 & -1
          \end{vmatrix}
          =(-1)(-1)^{1+1} \begin{vmatrix}
            -3 & 1
            \\
            1 & -1
          \end{vmatrix}
          +3(-1)^{1+2} \begin{vmatrix}
            1 & 1
            \\
            0 & -1
          \end{vmatrix}
          +1(-1)^{1+3} \begin{vmatrix}
            1 & -3
            \\
            0 & 1
          \end{vmatrix}
          \\
          \\
          \\
          =( -1 \times 2 )+( -3 \times -1 )+( 1 \times 1 )=2
          \\
          \\
          \rule{15cm}{1pt}
          \\
          \\
          C=\begin{vmatrix}
            -1 & 4 & 1
            \\
            1 & 3 & 1
            \\
            0 & 2 & -1
          \end{vmatrix}
          =-1(-1)^{1+1} \begin{vmatrix}
            3 & 1
            \\
            2 & -1
          \end{vmatrix}
          +4(-1)^{1+2} \begin{vmatrix}
            1 & 1
            \\
            0 & -1
          \end{vmatrix}
          +1(-1)^{1+3} \begin{vmatrix}
            1 & 3
            \\
            0 & 2
          \end{vmatrix}
          \\
          \\
          \\
          =( -1 \times -5 )+( -4 \times -1 )+( 1 \times 2 )=11
          \\
          \\
          \rule{15cm}{1pt}
          \\
          \\
          \Longrightarrow 1(-1)^{1+1} A+(-3)(-1)^{1+2}B+2(-1)^{1+3} C=(1 \times 32)+(3 \times 2)+(2 \times 11) 
          \\
          \\
          \\
          \therefore ~~~~ det \begin{pmatrix}
            1 & -3 & 2 & 0
            \\
            -1 & 4 & 3 & 1
            \\
            1 & 3 & -3 & 1
            \\
            0 & 2 & 1 & -1
          \end{pmatrix}=60 ~~~~ \checkmark
        $
      }

    \item (10 points) Are the vectors $v_1=(2, -1, 2), ~~ v_2=(2, -2, 1),$ and $v_3=(1, -1, 3)$ linearly independent?

      \textcolor{hwColor}{
        A collection of vectors is either linearly independent or linearly dependent. The vectors
        $v_1, v_2,$ and $v_3$ are linearly independent if the equation involving linear combinations
        $$c_1 v_1+c_2 v_2+c_3 v_3=0$$
        The above equation is true only when the scalars $c_i$ are all equal to zero. The vectors are linearly dependent if the
        equation has a solution when at least one of the scalars is not zero.
        \\
        \\
        $
          c_1 (2, -1, 2)+c_2 (2, -2, 1)+c_3 (1, -1, 3)=0
          \\
          \\
          \\
          \Longrightarrow \left(\begin{array}{ccc|c}  
            2 & -1 & 2 & 0
            \\  
            2 & -2 & 1 & 0
            \\
            1 & -1 & 3 & 0
          \end{array}\right)
        $
        \\
        \\
        Let’s reduce this augmented matrix to the reduced row echelon form
        \\
        \\
        $
          \rule{15cm}{1pt}
          \\
          \\
          \dfrac{1}{2}R_1 \rightarrow R_1
          \\
          \\
          \left(\begin{array}{ccc|c}  
            1 & -\dfrac{1}{2} & 1 & 0
            \\  
            2 & -2 & 1 & 0
            \\
            1 & -1 & 3 & 0
          \end{array}\right)
          \\
          \\
          \rule{15cm}{1pt}
          \\
          \\
          -2R_1+R_2 \rightarrow R_2
          \\
          \\
          \left(\begin{array}{ccc|c}  
            1 & -\dfrac{1}{2} & 1 & 0
            \\  
            0 & -1 & -1 & 0
            \\
            1 & -1 & 3 & 0
          \end{array}\right)
          \\
          \\
          \rule{15cm}{1pt}
          \\
          \\
          -R_1+R_3 \rightarrow R_3
          \\
          \\
          \left(\begin{array}{ccc|c}  
            1 & -\dfrac{1}{2} & 1 & 0
            \\  
            0 & -1 & -1 & 0
            \\
            0 & -\dfrac{1}{2} & 2 & 0
          \end{array}\right)
          \\
          \\
          \rule{15cm}{1pt}
          \\
          \\
          -R_2 \rightarrow R_2
          \\
          \\
          \left(\begin{array}{ccc|c}  
            1 & -\dfrac{1}{2} & 1 & 0
            \\  
            0 & 1 & 1 & 0
            \\
            0 & -\dfrac{1}{2} & 2 & 0
          \end{array}\right)
          \\
          \\
          \rule{15cm}{1pt}
          \\
          \\
          \dfrac{1}{2}R_2+R_3 \rightarrow R_3
          \\
          \\
          \left(\begin{array}{ccc|c}  
            1 & -\dfrac{1}{2} & 1 & 0
            \\  
            0 & 1 & 1 & 0
            \\
            0 & 0 & \dfrac{5}{2} & 0
          \end{array}\right)
          \\
          \\
          \rule{15cm}{1pt}
          \\
          \\
          \dfrac{2}{5}R_3 \rightarrow R_3
          \\
          \\
          \left(\begin{array}{ccc|c}  
            1 & -\dfrac{1}{2} & 1 & 0
            \\  
            0 & 1 & 1 & 0
            \\
            0 & 0 & 1 & 0
          \end{array}\right)
          \\
          \\
          \rule{15cm}{1pt}
          \\
          \\
          -R_3+R_2 \rightarrow R_2
          \\
          \\
          \left(\begin{array}{ccc|c}  
            1 & -\dfrac{1}{2} & 1 & 0
            \\  
            0 & 1 & 0 & 0
            \\
            0 & 0 & 1 & 0
          \end{array}\right)
          \\
          \\
          \rule{15cm}{1pt}
          \\
          \\
          -R_3+R_1 \rightarrow R_1
          \\
          \\
          \left(\begin{array}{ccc|c}  
            1 & -\dfrac{1}{2} & 0 & 0
            \\  
            0 & 1 & 0 & 0
            \\
            0 & 0 & 1 & 0
          \end{array}\right)
          \\
          \\
          \rule{15cm}{1pt}
          \\
          \\
          \dfrac{1}{2}R_2+R_1 \rightarrow R_1
          \\
          \\
          \left(\begin{array}{ccc|c}  
            1 & 0 & 0 & 0
            \\  
            0 & 1 & 0 & 0
            \\
            0 & 0 & 1 & 0
          \end{array}\right)
        $
        \\
        \\
        \\
        The reduced row echelon form of the coefficient matrix of the homogeneous system 
        \\
        \\
        meaning 
        $
          \begin{cases}
            c_1=0
            \\
            c_2=0
            \\
            c_3=0
          \end{cases}
        $
        \\
        \\
        This system has only the trivial solution, so that the only solution is $c_1=c_2=c_3=0$.
        \\
        \\
        \\
        $
          \therefore ~~~~ 
        $ The vectors $v_1, v_2,$ and $v_3$ are linearly independent. $\blacksquare$ 
      }

    \item (10 points) Use Cramer’s rule to determine the solutions, if any exist, to the following system
    of equations.
    $$
      \begin{cases}
        2x-y+z=1
        \\
        x+2y-z=2
        \\
        -x+y+z=3
      \end{cases}
    $$

      \textcolor{hwColor}{
        From the given system of equations we can write the following:
        \\
        \\ 
        $
          Z=\begin{pmatrix}
            2 & -1 & 1
            \\
            1 & 2 & -1
            \\
            -1 & 1 & 1
          \end{pmatrix}
          \\
          \\
          \\
          det(Z)=\begin{vmatrix}
            2 & -1 & 1
            \\
            1 & 2 & -1
            \\
            -1 & 1 & 1
          \end{vmatrix}=
          2(-1)^{1+1} \begin{vmatrix}
            2 & -1
            \\
            1 & 1
          \end{vmatrix}
          +(-1)(-1)^{1+2} \begin{vmatrix}
            1 & -1
            \\
            -1 & 1
          \end{vmatrix}
          +1(-1)^{1+3} \begin{vmatrix}
            1 & 2
            \\
            -1 & 1
          \end{vmatrix}
          \\
          \\
          \\
          =(2 \times 3)+(1 \times 0)+(1 \times 3)
          \\
          \\
          \\
          \therefore ~~~~ det(Z)=9
          \\
          \\
          \rule{15cm}{1pt}
          \\
          \\
          D_x=\begin{vmatrix}
            1 & -1 & 1
            \\
            2 & 2 & -1
            \\
            3 & 1 & 1
          \end{vmatrix}=
          1(-1)^{1+1} \begin{vmatrix}
            2 & -1
            \\
            1 & 1
          \end{vmatrix}
          +(-1)(-1)^{1+2} \begin{vmatrix}
            2 & -1
            \\
            3 & 1
          \end{vmatrix}
          +1(-1)^{1+3} \begin{vmatrix}
            2 & 2
            \\
            3 & 1
          \end{vmatrix}=4
          \\
          \\
          \rule{15cm}{1pt}
          \\
          \\
          D_y=\begin{vmatrix}
            2 & 1 & 1
            \\
            1 & 2 & -1
            \\
            -1 & 3 & 1
          \end{vmatrix}
          =2 (-1)^{1+1} \begin{vmatrix}
            2 & -1
            \\
            3 & 1
          \end{vmatrix}
          +1 (-1)^{1+2} \begin{vmatrix}
            1 & -1
            \\
            -1 & 1
          \end{vmatrix}
          +1 (-1)^{1+3} \begin{vmatrix}
            1 & 2
            \\
            -1 & 3
          \end{vmatrix}=15
          \\
          \\
          \rule{15cm}{1pt}
          \\
          \\
          D_z=\begin{vmatrix}
            2 & -1 & 1
            \\
            1 & 2 & 2
            \\
            -1 & 1 & 3 
          \end{vmatrix}
          =2(-1)^{1+1} \begin{vmatrix}
            2 & 2
            \\
            1 & 3 
          \end{vmatrix}
          +(-1)(-1)^{1+2} \begin{vmatrix}
            1 & 2
            \\
            -1 & 3 
          \end{vmatrix}
          +1(-1)^{1+3} \begin{vmatrix}
            1 & 2
            \\
            -1 & 1
          \end{vmatrix}=16
          \\
          \\
          \rule{15cm}{1pt}
          \\
          \\
          \begin{cases}
            x=\dfrac{D_x}{det(Z)}=\dfrac{4}{9}
            \\
            \\
            y=\dfrac{D_y}{det(Z)}=\dfrac{15}{9}=\dfrac{5}{3}
            \\
            \\
            z=\dfrac{D_z}{det(Z)}=\dfrac{16}{9}
          \end{cases} ~~~~ \checkmark
        $
        \\
        \\
      }
    

    \item (10 points) Verify the Cauchy–Schwarz inequality $|u.v| \leq ||u||.||v||$ for the 
    vectors $u=(1, -2, -3, 4)$ and $v=(-1, -2, 2, 1)$ using the standard inner product in $R^4$.
    Find the angle between these vectors.

      \textcolor{hwColor}{
        To determine whether the given vectors satisfy the Cauchy–Schwarz inequality we need to 
        evaluate the required values:
        \\
        \\
        $
          \left\langle u.v\right\rangle=|(1, -2, -3, 4).(-1, -2, 2, 1)|=|\left[(1)(-1)+(-2)(-2)+(-3)(2)+(4)(1)\right]|=1
          \\
          \\
          \begin{cases}
            ||u||=\sqrt{(1)^2+(-2)^2+(-3)^2+(4)^2}=\sqrt{30}
            \\
            \\
            ||v||=\sqrt{(-1)^2+(-2)^2+(2)^2+(1)^2}=\sqrt{10}
          \end{cases} \longrightarrow ||u||.||v||=\sqrt{30} \sqrt{10}=\sqrt{300}
        $
        \\
        \\
        Clearly, $1 \leq \sqrt{300}$, therefore $|u.v| \leq ||u||.||v||$ holds. $~~~~ \checkmark$
        \\
        \\
        As we know the standard inner product formula is $\left\langle u.v\right\rangle=\left\lVert u\right\rVert \left\lVert v\right\rVert \angle (u, v)$, therefore
        the angle between the two given vectors is:
        \\
        \\
        $
          cos\left(\angle (u, v)\right)=\dfrac{\left\langle u.v\right\rangle}{\left\lVert u\right\rVert \left\lVert v\right\rVert}=\dfrac{1}{\sqrt{300}}
          \Longrightarrow \angle (u, v)=cos^{-1} \left[\dfrac{1}{\sqrt{300}}\right] \approx 86.70^{\circ}
        $
      }

    \pagebreak

    \item (15 points) Convert the basis $v_1=(1, -1, 0), v_2=(0, 1, -1), v_3=(1, 0, -1)$ for $R^3$ into 
    an orthonormal basis, using the Gram–Schmidt process and the standard inner product in $R^3$.

      \textcolor{hwColor}{
        \\
        The Gram-Schmidt process (or procedure) is a sequence of operations that allow to transform a set of linearly 
        independent vectors into a set of orthonormal vectors that span the same space spanned by the original set.
        \\
        \\
        $
          \begin{cases}
            u_1=\dfrac{v_1}{||v_1||}
            \\
            \\
            u_2=\dfrac{v_2-(u_1.u_2)u_1}{||v_2-(u_1.u_2)u_1||}
            \\
            \\
            u_3=\dfrac{v_3-(u_2.v_3)u_2-(u_1.v_3)u_1}{||v_3-(u_2.v_3)u_2-(u_1.v_3)u_1||}
          \end{cases}
          \\
          \\
          \rule{15cm}{1pt}
          \\
          \\
          u_1: 
          \\
          \\
          u_1=\dfrac{(1, -1, 0)}{\sqrt{(1)^2+(-1)^2+(0)^2}}=\dfrac{(1, -1, 0)}{\sqrt{2}}=(\dfrac{1}{\sqrt{2}}, -\dfrac{1}{\sqrt{2}}, 0)
          \\
          \\
          \rule{15cm}{1pt}
          \\
          \\
          u_2:
          \\
          \\
          u_2=\dfrac{v_2-(u_1.v_2)u_1}{||v_2-(u_1.v_2)u_1||}=\dfrac{(0, 1, -1)-\left[(\dfrac{1}{\sqrt{2}}, -\dfrac{1}{\sqrt{2}}, 0).(0, 1, -1)\right](\dfrac{1}{\sqrt{2}}, -\dfrac{1}{\sqrt{2}}, 0)}{||(0, 1, -1)-\left[(\dfrac{1}{\sqrt{2}}, -\dfrac{1}{\sqrt{2}}, 0).(0, 1, -1)\right](\dfrac{1}{\sqrt{2}}, -\dfrac{1}{\sqrt{2}}, 0)||}
          \\
          \\
          \\
          u_2=\dfrac{(0, 1, -1)-(-\dfrac{1}{2}, \dfrac{1}{2}, 0)}{||(0, 1, -1)-(-\dfrac{1}{2}, \dfrac{1}{2}, 0)||}
          =\dfrac{(\dfrac{1}{2},\dfrac{1}{2},-1)}{||(\dfrac{1}{2},\dfrac{1}{2},-1)||}
          =\dfrac{(\dfrac{1}{2},\dfrac{1}{2},-1)}{\sqrt{(\dfrac{1}{2})^2+(\dfrac{1}{2})^2+(-1)^2}}
          \\
          \\
          \\
          u_2=\dfrac{(\dfrac{1}{2},\dfrac{1}{2},-1)}{\sqrt{\dfrac{3}{2}}}=\sqrt{\dfrac{2}{3}}(\dfrac{1}{2},\dfrac{1}{2},-1)=(\dfrac{1}{\sqrt{6}}, \dfrac{1}{\sqrt{6}},-\sqrt{\dfrac{2}{3}})
          \\
          \\
          \rule{15cm}{1pt}
          \\
          \\
          u_3:
          \\
          \\
          u_3=\dfrac{v_3-(u_2.v_3)u_2-(u_1.v_3)u_1}{||v_3-(u_2.v_3)u_2-(u_1.v_3)u_1||}
          \\
          \\
          =\dfrac{(1, 0, -1)-((\dfrac{1}{\sqrt{6}}, \dfrac{1}{\sqrt{6}},-\sqrt{\dfrac{2}{3}}).(1, 0, -1))(\dfrac{1}{\sqrt{6}}, \dfrac{1}{\sqrt{6}},-\sqrt{\dfrac{2}{3}})-((\dfrac{1}{\sqrt{2}}, -\dfrac{1}{\sqrt{2}}, 0).(1, 0, -1))(\dfrac{1}{\sqrt{2}}, -\dfrac{1}{\sqrt{2}}, 0)}{||(1, 0, -1)-((\dfrac{1}{\sqrt{6}}, \dfrac{1}{\sqrt{6}},-\sqrt{\dfrac{2}{3}}).(1, 0, -1))(\dfrac{1}{\sqrt{6}}, \dfrac{1}{\sqrt{6}},-\sqrt{\dfrac{2}{3}})-((\dfrac{1}{\sqrt{2}}, -\dfrac{1}{\sqrt{2}}, 0).(1, 0, -1))(\dfrac{1}{\sqrt{2}}, -\dfrac{1}{\sqrt{2}}, 0)||}
          \\
          \\
          \\
        $
        By doing some algebra we get $u_3=(0, 0, 0)$. Since the result is zero, it is linearly dependent so we can ignore it. Therefore:
        \\
        \\
        $
          \begin{cases}
            u_1=\dfrac{(1, -1, 0)}{\sqrt{2}}=(\dfrac{1}{\sqrt{2}}, -\dfrac{1}{\sqrt{2}}, 0)
            \\
            \\
            u_2=(\dfrac{1}{\sqrt{6}}, \dfrac{1}{\sqrt{6}},-\sqrt{\dfrac{2}{3}})
          \end{cases} ~~~~ \checkmark
        $
      }

    \item Find the eigenvalues and associated eigenvectors of a given matrix $A$.
      \begin{itemize}
        \item (10 points) $A=\begin{pmatrix}
          3 & -1
          \\
          1 & 4
        \end{pmatrix}$.

          \textcolor{hwColor}{
            To solve for the eigenvalues, $\lambda_i$, and the corresponding eigenvectors, $x$ of an $n \times n$ matrix 
            $A$, we should do the following: 
            \\
            \begin{itemize}
              \item Multiply an $n \times n$ identity matrix by the scalar $\lambda$.
              \item Subtract the identity matrix multiple from the matrix $A$.
              \item Find the determinant of the matrix and the difference.
              \item Solve for the values of $\lambda$ that satisfy the equation $det\left(A-\lambda I\right)=0$.
              \item Solve for the corresponding vector to each $\lambda$.
            \end{itemize}
          }

          \textcolor{hwColor}{
            $
              A=\begin{pmatrix}
                3 & -1
                \\
                1 & 4
              \end{pmatrix}
              \\
              \\
              \lambda I=\lambda \begin{pmatrix}
                1 & 0
                \\
                0 & 1
              \end{pmatrix}=\begin{pmatrix}
                \lambda & 0
                \\
                0 & \lambda
              \end{pmatrix}
              \\
              \\
              A-\lambda I=\begin{pmatrix}
                3 & -1
                \\
                1 & 4
              \end{pmatrix}-\begin{pmatrix}
                \lambda & 0
                \\
                0 & \lambda
              \end{pmatrix}=\begin{pmatrix}
                3-\lambda & -1
                \\
                1 & 4-\lambda
              \end{pmatrix}
              \\
              \\
              det(A-\lambda I)=\begin{vmatrix}
                3-\lambda & -1
                \\
                1 & 4-\lambda
              \end{vmatrix}=(3-\lambda)(4-\lambda)+1=13-7\lambda+\lambda^2
              \\
              \\
              \\
              det(A-\lambda I)=0 \Longrightarrow \begin{cases}
                \lambda=\dfrac{7}{2}-i \dfrac{\sqrt{3}}{2}
                \\
                \\
                \lambda=\dfrac{7}{2}+i \dfrac{\sqrt{3}}{2}
              \end{cases}
            $
            \\
            \\
            Now let’s find the corresponding eigenvectors:
            \\
            \\
            $
              \lambda=\dfrac{7}{2}-i \dfrac{\sqrt{3}}{2} \Longrightarrow A-\lambda I=\begin{pmatrix}
                3-\left[\dfrac{7}{2}-i \dfrac{\sqrt{3}}{2}\right] & -1
                \\
                1 & 4-\left[\dfrac{7}{2}-i \dfrac{\sqrt{3}}{2}\right]
              \end{pmatrix}
              \\
              \\
              \\
              (A-\lambda I)X=0 \longrightarrow \begin{pmatrix}
                3-\left[\dfrac{7}{2}-i \dfrac{\sqrt{3}}{2}\right] & -1
                \\
                1 & 4-\left[\dfrac{7}{2}-i \dfrac{\sqrt{3}}{2}\right]
              \end{pmatrix} \begin{pmatrix}
                x_1
                \\
                x_2
              \end{pmatrix}=\begin{pmatrix}
                0
                \\
                0
              \end{pmatrix}
            $
            \\
            \\
            \\
            After doing some algebra we have $X=\begin{pmatrix}
              -1-i \sqrt{3}
              \\
              2
            \end{pmatrix}$
            \\
            \\
            \rule{15cm}{1pt}
            \\
            \\
            $
              \lambda=\dfrac{7}{2}+i \dfrac{\sqrt{3}}{2} \Longrightarrow A-\lambda I=\begin{pmatrix}
                3-\left[\dfrac{7}{2}+i \dfrac{\sqrt{3}}{2}\right] & -1
                \\
                1 & 4-\left[\dfrac{7}{2}+i \dfrac{\sqrt{3}}{2}\right]
              \end{pmatrix}
              \\
              \\
              \\
              (A-\lambda I)X=0 \longrightarrow \begin{pmatrix}
                3-\left[\dfrac{7}{2}+i \dfrac{\sqrt{3}}{2}\right] & -1
                \\
                1 & 4-\left[\dfrac{7}{2}+i \dfrac{\sqrt{3}}{2}\right]
              \end{pmatrix} \begin{pmatrix}
                x_1
                \\
                x_2
              \end{pmatrix}=\begin{pmatrix}
                0
                \\
                0
              \end{pmatrix}
            $
            \\
            \\
            \\
            After doing some algebra we have $X=\begin{pmatrix}
              -1+i \sqrt{3}
              \\
              2
            \end{pmatrix}$
            \\
            \\
            \\
            $
              \therefore ~~~~ \begin{cases}
                \lambda=\dfrac{7}{2}-i \dfrac{\sqrt{3}}{2} \Longrightarrow X=\begin{pmatrix}
                  -1-i \sqrt{3}
                  \\
                  2
                \end{pmatrix}
                \\
                \\
                \lambda=\dfrac{7}{2}+i \dfrac{\sqrt{3}}{2} \Longrightarrow X=\begin{pmatrix}
                  -1+i \sqrt{3}
                  \\
                  2
                \end{pmatrix}
              \end{cases} ~~~~~ \checkmark
            $
          }

        \pagebreak

        \item (15 points) $A=\begin{pmatrix}
          1 & 2 & 1 
          \\
          0 & 3 & 1 
          \\
          0 & 5 & -1
        \end{pmatrix}$

          \textcolor{hwColor}{
            $
              \\
              A-\lambda I=\begin{pmatrix}
                1 & 2 & 1 
                \\
                0 & 3 & 1 
                \\
                0 & 5 & -1
              \end{pmatrix}-\begin{pmatrix}
                \lambda & 0 & 0
                \\
                0 & \lambda & 0
                \\
                0 & 0 & \lambda
              \end{pmatrix}=\begin{pmatrix}
                1-\lambda & 2 & 1 
                \\
                0 & 3-\lambda & 1 
                \\
                0 & 5 & -1-\lambda
              \end{pmatrix}
              \\
              \\
              \\
              det(A-\lambda I)=\begin{vmatrix}
                1-\lambda & 2 & 1 
                \\
                0 & 3-\lambda & 1 
                \\
                0 & 5 & -1-\lambda
              \end{vmatrix}
              =(1-\lambda)(-1)^{1+1} \begin{vmatrix}
                3-\lambda & 1 
                \\
                5 & -1-\lambda
              \end{vmatrix}
              \\
              \\
              =(1-\lambda)(3-\lambda)(-1-\lambda)-5
              \\
              \\
              \\
              det(A-\lambda I)=0 \longleftrightarrow \begin{cases}
                \lambda=-2
                \\
                \lambda=1
                \\
                \lambda=4
              \end{cases}
            $
            \\
            \\
            Now we need to find the corresponding eigenvectors for each of the above eigenvalues.
            \\
            \\
            \rule{15cm}{3pt}
            \\
            \\
            $
              \lambda=-2 \longrightarrow A-\lambda I=\begin{pmatrix}
                3 & 2 & 1 
                \\
                0 & 5 & 1 
                \\
                0 & 5 & 1
              \end{pmatrix}
              \\
              \\
              \left(\begin{array}{ccc|c}  
                3 & 2 & 1 & 0
                \\
                0 & 5 & 1 & 0
                \\ 
                0 & 5 & 1 & 0
              \end{array}\right)
              \\
              \\
              \therefore ~~~~ X=\begin{pmatrix}
                -\dfrac{1}{5}
                \\
                \\
                -\dfrac{1}{5}
                \\
                \\
                1
              \end{pmatrix}
              \\
              \\
              \rule{15cm}{3pt}
              \\
              \\
              \lambda=1 \longrightarrow A-\lambda I=\begin{pmatrix}
                0 & 2 & 1 
                \\
                0 & 2 & 1 
                \\
                0 & 5 & -2
              \end{pmatrix}
              \\
              \\
              \left(\begin{array}{ccc|c}  
                0 & 2 & 1 & 0 
                \\
                0 & 2 & 1 & 0
                \\
                0 & 5 & -2 & 0
              \end{array}\right)
              \\
              \\
              \therefore ~~~~ X=\begin{pmatrix}
                1
                \\
                0
                \\
                0
              \end{pmatrix}
              \\
              \\
              \rule{15cm}{3pt}
              \\
              \\
              \lambda=4 \longrightarrow A-\lambda I=\begin{pmatrix}
                -3 & 2 & 1 
                \\
                0 & -1 & 1 
                \\
                0 & 5 & -5
              \end{pmatrix}
              \\
              \\
              \left(\begin{array}{ccc|c}  
                -3 & 2 & 1 & 0
                \\
                0 & -1 & 1 & 0
                \\
                0 & 5 & -5 & 0
              \end{array}\right)
              \\
              \\
              \therefore ~~~~ X=\begin{pmatrix}
                1
                \\
                1
                \\
                1
              \end{pmatrix}
            $
          }

      \end{itemize}


    \item (15 points) Find both a basis for the row space and a basis for the column space of the
    matrix
    $$
      \begin{pmatrix}
        1 & 3 & -1 & -1
        \\
        2 & 1 & -2 & 0
        \\
        3 & 4 & -3 & 2
      \end{pmatrix}
    $$
    What is the rank of this matrix?

      \textcolor{hwColor}{
        \\
        We start by putting the matrix in row echelon form.
        \\
        \\
        $
          A=\begin{pmatrix}
            1 & 3 & -1 & -1
            \\
            2 & 1 & -2 & 0
            \\
            3 & 4 & -3 & 2
          \end{pmatrix}
          \\
          \\
          \rule{15cm}{1pt}
          \\
          \\
          -2R_1+R_2 \rightarrow R_2
          \\
          \\
          \begin{pmatrix}
            1 & 3 & -1 & -1
            \\
            0 & -5 & 0 & 2
            \\
            3 & 4 & -3 & 2
          \end{pmatrix}
          \\
          \\
          \rule{15cm}{1pt}
          \\
          \\
          -3R_1+R_3 \rightarrow R_3
          \\
          \\
          \begin{pmatrix}
            1 & 3 & -1 & -1
            \\
            0 & -5 & 0 & 2
            \\
            0 & -5 & 0 & 5
          \end{pmatrix}
          \\
          \\
          \rule{15cm}{1pt}
          \\
          \\
          -\dfrac{1}{5}R_2 \rightarrow R_2
          \\
          \\
          \begin{pmatrix}
            1 & 3 & -1 & -1
            \\
            0 & 1 & 0 & -\dfrac{2}{5}
            \\
            0 & -5 & 0 & 5
          \end{pmatrix}
          \\
          \\
          \rule{15cm}{1pt}
          \\
          \\
          5R_2+R_3 \rightarrow R_3
          \\
          \\
          \begin{pmatrix}
            1 & 3 & -1 & -1
            \\
            0 & 1 & 0 & -\dfrac{2}{5}
            \\
            0 & 0 & 0 & 3
          \end{pmatrix}
          \\
          \\
          \rule{15cm}{1pt}
          \\
          \\
          \dfrac{1}{3}R_3 \rightarrow R_3
          \\
          \\
          \begin{pmatrix}
            1 & 3 & -1 & -1
            \\
            0 & 1 & 0 & -\dfrac{2}{5}
            \\
            0 & 0 & 0 & 1
          \end{pmatrix}
          \\
          \\
          \rule{15cm}{1pt}
          \\
          \\
          \dfrac{2}{5}R_3+R_2 \rightarrow R_2
          \\
          \\
          \begin{pmatrix}
            1 & 3 & -1 & -1
            \\
            0 & 1 & 0 & 0
            \\
            0 & 0 & 0 & 1
          \end{pmatrix}
          \\
          \\
          \rule{15cm}{1pt}
          \\
          \\
          R_3+R_1 \rightarrow R_1
          \\
          \\
          \begin{pmatrix}
            1 & 3 & -1 & 0
            \\
            0 & 1 & 0 & 0
            \\
            0 & 0 & 0 & 1
          \end{pmatrix}
          \\
          \\
          \rule{15cm}{1pt}
          \\
          \\
          -3R_2+R_1 \rightarrow R_1
          \\
          \\
          \\
          \therefore ~~~~~ A=\begin{pmatrix}
            1 & 0 & -1 & 0
            \\
            0 & 1 & 0 & 0
            \\
            0 & 0 & 0 & 1
          \end{pmatrix}
        $
        \\
        \\
        Row operations do not change the row space, so the rows of the matrix at the end have
        the same span as our given matrix. Furthermore, the nonzero rows of a matrix in row echelon
        form are linearly independent. Therefore, the row space has a basis:
        \\
        \\
        $
          \{ \left[1 , 0 , -1 , 0\right], \left[0 , 1 , 0 , 0\right], \left[0 , 0 , 0 , 1\right] \} 
        $
        \\
        \\
        Recall that a leading one is the first nonzero entry in a row. The columns containing leading ones 
        are the pivot columns. To obtain a basis for the column space, we just use the pivot columns from 
        the original matrix:
        \\
        \\
        $
          \{ 
            \begin{bmatrix}
              1
              \\
              2
              \\
              3
            \end{bmatrix},
            \begin{bmatrix}
              3
              \\
              1
              \\
              4
            \end{bmatrix},
            \begin{bmatrix}
              -1
              \\
              0
              \\
              2
            \end{bmatrix}
          \} 
        $
        \\
        \\
        Finally, the row space and column space each have bases with three vectors, so they have
        dimension three. Therefore, the rank of A is 3.
      }


    \item (5 points each) Let $L: \mathbb{R}^2 \rightarrow \mathbb{R}^3$ be a function given by $L(x,y)=(x-y, x+2y, 3x-2y)$.
      \begin{itemize}
        \item Show that $L$ is a linear transformation.
        \item Find the kernel of $L$.
        \item Find a basis for the range of the transformation $L$.
        \item Find the standard matrix representation of the transformation $L$. (Use the standard bases for $\mathbb{R}^2 \rightarrow \mathbb{R}^3$.)
      \end{itemize}


    \item (15 points) (The Pythagorean Law) Prove that if $\mathbf{u}$ and $\mathbf{v}$ are orthogonal vectors in an inner
    product space $V$, then $||\mathbf{u}+\mathbf{v}||^2=||\mathbf{u}||^2+||\mathbf{v}||^2$.


    \item (15 points) Write the explicit forms of the Pythagorean Law and the orthogonality relation
    in the vector space of continuous functions $\mathsf{C}\left[a, b\right]$  with inner product defined by:

    $(f,g)=\bigints\limits_{a}^{b} f(x) g(x) dx$.



    \item (15 points) Let $A$ be an $n \times n$ invertible matrix, let $\lambda \neq 0$ be an eigenvalue of $A$,
    and let $x$ be an eigenvector belonging to $\lambda$: $A x=\lambda x$. Use mathematical induction to show that
    $\lambda^{-m}$ is an eigenvalue of $A^{-m}=(A^{-1})^m$ and $x$ is an eigenvector of $A^{-m}$ belonging to $\lambda^{-m}$.  

  \end{enumerate}

\end{document}
