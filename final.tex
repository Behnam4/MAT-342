\documentclass[fleqn]{article}
\oddsidemargin 0.0in
\textwidth 6.0in
\thispagestyle{empty}
\usepackage{import}
\usepackage{amsmath}
\usepackage{graphicx}
\usepackage{flexisym}
\usepackage{amssymb}
\usepackage{bigints} 
\usepackage[english]{babel}
\usepackage[utf8x]{inputenc}
\usepackage{float}
\usepackage[colorinlistoftodos]{todonotes}

\definecolor{hwColor}{HTML}{AD53BA}

\begin{document}

  \begin{titlepage}

    \newcommand{\HRule}{\rule{\linewidth}{0.5mm}}

    \center


    \textsc{\LARGE Arizona State University}\\[1.5cm]

    \textsc{\LARGE Linear Algebra }\\[1.5cm]


    \begin{figure}
      \includegraphics[width=\linewidth]{asu.png}
    \end{figure}


    \HRule \\[0.4cm]
    { \huge \bfseries Final}\\[0.4cm] 
    \HRule \\[1.5cm]

    \textbf{Behnam Amiri}

    \bigbreak

    \textbf{Prof: Sergei Suslov}

    \bigbreak


    \textbf{{\large \today}\\[2cm]}

    \vfill

  \end{titlepage}

  \begin{enumerate}
    \item Let 
    $$
      A=\begin{pmatrix}
        -7 & -3 & 1
        \\
        0 & 3 & -1 
        \\
        -3 & 4 & -2
      \end{pmatrix}
    $$
      \begin{itemize}
        \item (20 points) Find $A^{-1}$.

          \textcolor{hwColor}{
            $
              A^{-1}=\dfrac{1}{det(A)} adj(A) 
              \\
              \\
              \\
              adj(A)=\left(A_{ij}\right)^T=\begin{pmatrix}
                +\begin{vmatrix}
                  3 & -1 
                  \\
                  4 & -2
                \end{vmatrix} 
                & -\begin{vmatrix}
                  0 & -1 
                  \\
                  -3 & -2
                \end{vmatrix} 
                & +\begin{vmatrix}
                  0 & 3 
                  \\
                  -3 & 4
                \end{vmatrix}
                \\
                \\
                -\begin{vmatrix}
                  -3 & 1 
                  \\
                  4 & -2
                \end{vmatrix} 
                & +\begin{vmatrix}
                  -7 & 1 
                  \\
                  -3 & -2
                \end{vmatrix} 
                & -\begin{vmatrix}
                  -7 & -3
                  \\
                  -3 & 4
                \end{vmatrix}
                \\
                \\
                +\begin{vmatrix}
                  -3 & 1
                  \\
                  3 & -1
                \end{vmatrix} 
                & -\begin{vmatrix}
                  -7 & 1
                  \\
                  0 & -1
                \end{vmatrix} 
                & +\begin{vmatrix}
                  -7 & -3
                  \\
                  0 & 3
                \end{vmatrix}
              \end{pmatrix}^T=\begin{pmatrix}
                -2 & 3 & 9
                \\
                -2 & 17 & 37 
                \\
                0 & -7 & -21
              \end{pmatrix}^T
              \\
              \\
              \\
              \\
              \therefore ~~~~ adj(A)=\begin{pmatrix}
                -2 & -2 & 0
                \\
                3 & 17 & -7 
                \\
                9 & 37 & -21
              \end{pmatrix} ~~~~ \checkmark
              \\
              \\
              \rule{15cm}{2pt}
              \\
              \\
              det(A)=\begin{vmatrix}
                -7 & -3 & 1
                \\
                0 & 3 & -1 
                \\
                -3 & 4 & -2
              \end{vmatrix}
              =(-7)(-1)^{1+1} \begin{vmatrix}
                3 & -1 
                \\
                4 & -2
              \end{vmatrix}+0+(-3)(-1)^{3+1} \begin{vmatrix}
                -3 & 1
                \\
                3 & -1 
              \end{vmatrix}
              \\
              \\
              =(-7)(-2)+(-3)(0)
              \\
              \\
              \\
              \therefore ~~~~ det(A)=14 ~~~~ \checkmark
            $
            \\
            \\
            \\
            Now we can find the inverse matrix:
            \\
            \\
            $
              A^{-1}=\dfrac{1}{det(A)} adj(A)=\dfrac{1}{14} \begin{pmatrix}
                -2 & -2 & 0
                \\
                3 & 17 & -7 
                \\
                9 & 37 & -21
              \end{pmatrix}
              \\
              \\
              \\
              A^{-1}=\begin{pmatrix}
                -\dfrac{1}{7} & -\dfrac{1}{7} & 0
                \\
                \\
                \dfrac{3}{14} & \dfrac{17}{14} & -\dfrac{1}{2}
                \\
                \\
                \dfrac{9}{14} & \dfrac{37}{14} & -\dfrac{3}{2}
              \end{pmatrix} ~~~~ \checkmark
            $
            \\
            \\
            \\
            \\
            \textbf{Testing the solution:}
            \\
            \\
            $
              A.A^{-1}=\begin{pmatrix}
                -7 & -3 & 1
                \\
                0 & 3 & -1 
                \\
                -3 & 4 & -2
              \end{pmatrix}.\begin{pmatrix}
                -\dfrac{1}{7} & -\dfrac{1}{7} & 0
                \\
                \\
                \dfrac{3}{14} & \dfrac{17}{14} & -\dfrac{1}{2}
                \\
                \\
                \dfrac{9}{14} & \dfrac{37}{14} & -\dfrac{3}{2}
              \end{pmatrix}=\begin{pmatrix}
                1 & 0 & 0
                \\
                0 & 1 & 0
                \\
                0 & 0 & 1
              \end{pmatrix}=I_{2 \times 2} ~~~~ \checkmark
            $ 
            \\
            \\
            \\
            The above justifies the truthy of our result.
            \\
            \\
          }


        \item (10 points) Solve the system $Ax=(1, -1, 1)^T$.

          \textcolor{hwColor}{
            $
              \\
              Ax=(1, -1, 1)^T
              \Longrightarrow
              A^{-1}.Ax=A^{-1}\begin{pmatrix}
                1
                \\
                -1
                \\
                1
              \end{pmatrix}
              \Longrightarrow
              Ix=A^{-1}\begin{pmatrix}
                1
                \\
                -1
                \\
                1
              \end{pmatrix}
              \\
              \\
              \\
              x=\begin{pmatrix}
                -\dfrac{1}{7} & -\dfrac{1}{7} & 0
                \\
                \\
                \dfrac{3}{14} & \dfrac{17}{14} & -\dfrac{1}{2}
                \\
                \\
                \dfrac{9}{14} & \dfrac{37}{14} & -\dfrac{3}{2}
              \end{pmatrix}.\begin{pmatrix}
                1
                \\
                -1
                \\
                1
              \end{pmatrix}
              =\begin{pmatrix}
                (-\dfrac{1}{7})(1)+(-\dfrac{1}{7})(-1)+(0)(1)
                \\
                \\
                (\dfrac{3}{14})(1)+(\dfrac{17}{14})(-1)+(-\dfrac{1}{2})(1)
                \\
                \\
                (\dfrac{9}{14})(1)+(\dfrac{37}{14})(-1)+(-\dfrac{3}{2})(1)
              \end{pmatrix}
              \\
              \\
              \\
              \\
              \therefore ~~~~ x=\begin{pmatrix}
                0
                \\
                \\
                -\dfrac{3}{2}
                \\
                \\
                -\dfrac{7}{2}
              \end{pmatrix} ~~~~ \checkmark
            $
          }
      \end{itemize}

    \item (10 points) Evaluate the following determinant:
    $$
      det\begin{pmatrix}
        -1 & 2 & 3
        \\
        1 & -1 & 1
        \\
        3 & 2 & 1
      \end{pmatrix}
    $$

      \textcolor{hwColor}{
        $
          \begin{vmatrix}
            -1 & 2 & 3
            \\
            1 & -1 & 1
            \\
            3 & 2 & 1
          \end{vmatrix}
          =(-1)(-1)^{1+1} \begin{vmatrix}
            -1 & 1
            \\
            2 & 1
          \end{vmatrix}
          +2(-1)^{1+2} \begin{vmatrix}
            1 & 1
            \\
            3 & 1
          \end{vmatrix}
          +3(-1)^{1+3} \begin{vmatrix}
            1 & -1
            \\
            3 & 2
          \end{vmatrix}
          \\
          \\
          \\
          =(-1)(-3)+(-2)(-2)+(3)(5)=3+4+15
          \\
          \\
          \\
          \therefore ~~~~~   det\begin{pmatrix}
            -1 & 2 & 3
            \\
            1 & -1 & 1
            \\
            3 & 2 & 1
          \end{pmatrix}=22  ~~~~ \checkmark
        $
      }

    \item (25 points) Convert the basis $v_1=(1,-1, 0), ~ v_2=(0, 1, -1),$ and $v_3=(1, 1, 1)$ for
    $\mathbb{R}^3$ into an orthonormal basis, using the Gram-Schmidt process and the standard inner product in $\mathbb{R}^3$. 


      \textcolor{hwColor}{
        \\
        The Gram-Schmidt process (or procedure) is a sequence of operations that allow to transform a set of linearly 
        independent vectors into a set of orthonormal vectors that span the same space spanned by the original set.
        \\
        \\
        $
          \begin{cases}
            u_1=\dfrac{v_1}{||v_1||}
            \\
            \\
            u_2=\dfrac{v_2-(u_1.u_2)u_1}{||v_2-(u_1.u_2)u_1||}
            \\
            \\
            u_3=\dfrac{v_3-(u_2.v_3)u_2-(u_1.v_3)u_1}{||v_3-(u_2.v_3)u_2-(u_1.v_3)u_1||}
          \end{cases}
          \\
          \\
          \rule{15cm}{2pt}
          \\
          \\
          u_1=\dfrac{(1, -1, 0)}{\sqrt{(1)^2+(-1)^2+(0)^2}}=\dfrac{(1, -1, 0)}{\sqrt{2}}=(\dfrac{1}{\sqrt{2}}, -\dfrac{1}{\sqrt{2}}, 0)
          \\
          \\
          \rule{15cm}{2pt}
          \\
          \\
          u_2=\dfrac{v_2-(u_1.v_2)u_1}{||v_2-(u_1.v_2)u_1||}=\dfrac{(0, 1, -1)-\left[(\dfrac{1}{\sqrt{2}}, -\dfrac{1}{\sqrt{2}}, 0).(0, 1, -1)\right](\dfrac{1}{\sqrt{2}}, -\dfrac{1}{\sqrt{2}}, 0)}{||(0, 1, -1)-\left[(\dfrac{1}{\sqrt{2}}, -\dfrac{1}{\sqrt{2}}, 0).(0, 1, -1)\right](\dfrac{1}{\sqrt{2}}, -\dfrac{1}{\sqrt{2}}, 0)||}
          \\
          \\
          \\
          u_2=\dfrac{(0, 1, -1)-(-\dfrac{1}{2}, \dfrac{1}{2}, 0)}{||(0, 1, -1)-(-\dfrac{1}{2}, \dfrac{1}{2}, 0)||}
          =\dfrac{(\dfrac{1}{2},\dfrac{1}{2},-1)}{||(\dfrac{1}{2},\dfrac{1}{2},-1)||}
          =\dfrac{(\dfrac{1}{2},\dfrac{1}{2},-1)}{\sqrt{(\dfrac{1}{2})^2+(\dfrac{1}{2})^2+(-1)^2}}
          \\
          \\
          \\
          u_2=\dfrac{(\dfrac{1}{2},\dfrac{1}{2},-1)}{\sqrt{\dfrac{3}{2}}}=\sqrt{\dfrac{2}{3}}(\dfrac{1}{2},\dfrac{1}{2},-1)=(\dfrac{1}{\sqrt{6}}, \dfrac{1}{\sqrt{6}},-\sqrt{\dfrac{2}{3}})
          \\
          \\
          \rule{15cm}{2pt}
          \\
          \\
          u_3=\dfrac{v_3-\left[u_2.v_3\right]u_2-\left[u_1.v_3\right]u_1}{||v_3-\left[u_2.v_3\right] u_2-\left[u_1.v_3\right]u_1||}
          \\
          \\
          =\dfrac{(1, 1, 1)-\left[(\dfrac{1}{\sqrt{6}}, \dfrac{1}{\sqrt{6}},-\sqrt{\dfrac{2}{3}}).(1, 1, 1)\right](\dfrac{1}{\sqrt{6}}, \dfrac{1}{\sqrt{6}},-\sqrt{\dfrac{2}{3}})-\left[(\dfrac{1}{\sqrt{2}}, -\dfrac{1}{\sqrt{2}}, 0).(1, 1, 1)\right](\dfrac{1}{\sqrt{2}}, -\dfrac{1}{\sqrt{2}}, 0)}{||(1, 1, 1)-\left[(\dfrac{1}{\sqrt{6}}, \dfrac{1}{\sqrt{6}},-\sqrt{\dfrac{2}{3}}).(1, 1, 1)\right] (\dfrac{1}{\sqrt{6}}, \dfrac{1}{\sqrt{6}},-\sqrt{\dfrac{2}{3}})-\left[(\dfrac{1}{\sqrt{2}}, -\dfrac{1}{\sqrt{2}}, 0).(1, 1, 1)\right](\dfrac{1}{\sqrt{2}}, -\dfrac{1}{\sqrt{2}}, 0)||}
          \\
          \\
          \\
          \Longrightarrow =(\dfrac{1}{\sqrt{3}}, \dfrac{1}{\sqrt{3}}, \dfrac{1}{\sqrt{3}})
          \\
          \\
          \rule{15cm}{2pt}
          \\
          \\
          \therefore ~~~~ \begin{cases}
            u_1=\left(\dfrac{1}{\sqrt{2}}, -\dfrac{1}{\sqrt{2}}, 0\right)
            \\
            \\
            u_2=\left(\dfrac{1}{\sqrt{6}}, \dfrac{1}{\sqrt{6}},-\sqrt{\dfrac{2}{3}}\right)
            \\
            \\
            u_3=\left(\dfrac{1}{\sqrt{3}}, \dfrac{1}{\sqrt{3}}, \dfrac{1}{\sqrt{3}}\right)
          \end{cases} ~~~~ \checkmark
        $
      }

    \pagebreak

    \item Find the eigenvalues and associated eigenvectors of a given matrix $A$.
      \begin{itemize}
        \item (15 points) 
        $A=\begin{pmatrix}
          2 & 1
          \\
          6 & -3
        \end{pmatrix}$

          \textcolor{hwColor}{
            \\
            To solve for the eigenvalues, $\lambda_i$, and the corresponding eigenvectors, $x$ of an $n \times n$ matrix 
            $A$, we should do the following: 
            \\
            \begin{itemize}
              \item Multiply an $n \times n$ identity matrix by the scalar $\lambda$.
              \item Subtract the identity matrix multiple from the matrix $A$.
              \item Find the determinant of the matrix and the difference.
              \item Solve for the values of $\lambda$ that satisfy the equation $det\left(A-\lambda I\right)=0$.
              \item Solve for the corresponding vector to each $\lambda$.
            \end{itemize}
          }

          \textcolor{hwColor}{
            $
              \\
              A=\begin{pmatrix}
                2 & 1 
                \\
                6 & -3
              \end{pmatrix}
              \\
              \\
              \\
              \lambda I=\lambda \begin{pmatrix}
                1 & 0 
                \\
                0 & 1
              \end{pmatrix}=\begin{pmatrix}
                \lambda & 0
                \\
                0 & \lambda
              \end{pmatrix}
              \\
              \\
              \\
              A-\lambda I=\begin{pmatrix}
                2 & 1 
                \\
                6 & -3
              \end{pmatrix}-\begin{pmatrix}
                \lambda & 0
                \\
                0 & \lambda
              \end{pmatrix}=\begin{pmatrix}
                2-\lambda & 1
                \\
                6 & -3-\lambda
              \end{pmatrix}
              \\
              \\
              \\
              det\left(A-\lambda I\right)=\begin{vmatrix}
                2-\lambda & 1
                \\
                6 & -3-\lambda
              \end{vmatrix}=\left(2-\lambda\right) \left(-3-\lambda\right)-6=\lambda^2+\lambda-12
              \\
              \\
              \\
              det\left(A-\lambda I\right)=\lambda^2+\lambda-12=0 \rightarrow \begin{cases}
                \lambda=-4 
                \\
                \lambda=3
              \end{cases}
            $
            \\
            \\
            Now let's find the corresponding eigenvectors: 
            \\
            \\
            $
              \lambda=-4 \Rightarrow A-\lambda I=\begin{pmatrix}
                6 & 1
                \\
                6 & 1
              \end{pmatrix}
              \\
              \\
              \\
              \left(A-\lambda I\right) X=0 \Rightarrow \begin{pmatrix}
                6 & 1
                \\
                6 & 1
              \end{pmatrix} \begin{pmatrix}
                x_1
                \\
                x_2
              \end{pmatrix}=\begin{pmatrix}
                0
                \\
                0
              \end{pmatrix} \Rightarrow \left(\begin{array}{cc|c}  
                6 & 1 & 0
                \\  
                6 & 1 & 0
              \end{array}\right)
              \\
              \\
              \rule{15cm}{2pt}
              \\
              \\
              \dfrac{1}{6}R_1 \rightarrow R_1
              \\
              \\
              \left(\begin{array}{cc|c}  
                1 & \dfrac{1}{6} & 0
                \\  
                6 & 1 & 0
              \end{array}\right)
              \\
              \\
              \rule{15cm}{2pt}
              \\
              \\
              R_2-6R_1 \rightarrow R_2
              \\
              \\
              \left(\begin{array}{cc|c}  
                1 & \dfrac{1}{6} & 0
                \\  
                0 & 0 & 0
              \end{array}\right)
              \\
              \\
              \rule{15cm}{2pt}
              \\
              \\
              \Longrightarrow x_1+\dfrac{1}{6}x_2=0 \Longrightarrow X=\begin{pmatrix}
                -\dfrac{1}{6} x_2
                \\
                x_2
              \end{pmatrix}
            $
            \\
            \\
            Now let's set $x_2=1$ then we have the following eigenvector:
            \\
            \\
            $
              X=\begin{pmatrix}
                -\dfrac{1}{6}
                \\
                1
              \end{pmatrix}
            $
            \\
            \\
            \rule{15cm}{2pt}
            \\
            \\
            $
              \lambda=3 \Rightarrow A-\lambda I=\begin{pmatrix}
                -1 & 1
                \\
                6 & -6
              \end{pmatrix}
              \\
              \\
              \left(A-\lambda I\right) X=0 \Rightarrow \begin{pmatrix}
                -1 & 1
                \\
                6 & -6
              \end{pmatrix} \begin{pmatrix}
                x_1
                \\
                x_2
              \end{pmatrix}=\begin{pmatrix}
                0
                \\
                0
              \end{pmatrix} \Rightarrow \left(\begin{array}{cc|c}  
                -1 & 1 & 0
                \\  
                6 & -6 & 0
              \end{array}\right)
              \\
              \\
              \rule{15cm}{2pt}
              \\
              \\
              -R_1 \rightarrow R_1
              \\
              \\
              \left(\begin{array}{cc|c}  
                1 & -1 & 0
                \\  
                6 & -6 & 0
              \end{array}\right)
              \\
              \\
              \rule{15cm}{2pt}
              \\
              \\
              R_2-6R_1 \rightarrow R_2
              \\
              \\
              \left(\begin{array}{cc|c}  
                1 & -1 & 0
                \\  
                0 & 0 & 0
              \end{array}\right)
              \\
              \\
              \rule{15cm}{2pt}
              \\
              \\
              \Longrightarrow x_1-x_2=0 \Longrightarrow X=\begin{pmatrix}
                x_2
                \\
                x_2
              \end{pmatrix}
              \\
              \\
            $
            Let's set $x_2=1$ then $X=\begin{pmatrix}
              1
              \\
              1
            \end{pmatrix}$
            \\
            \\
            \\
            $
              \therefore ~~~~ \begin{cases}
                \lambda=-4 \Longrightarrow X=\begin{pmatrix}
                  -\dfrac{1}{6} 
                  \\
                  \\
                  1
                \end{pmatrix}
                \\
                \\
                \lambda=3 \longrightarrow X=\begin{pmatrix}
                  1
                  \\
                  1
                \end{pmatrix}
              \end{cases} ~~~~ \checkmark
            $
            \\
            \\
            \\
            \\
          }

        \item (25 points) $A=\begin{pmatrix}
          -2 & 0 & 0
          \\
          3 & 7 & 2 
          \\
          1 & -2 & 2
        \end{pmatrix}$
      \end{itemize}

        \textcolor{hwColor}{
          $
            A=\begin{pmatrix}
              -2 & 0 & 0 
              \\
              3 & 7 & 2 
              \\
              1 & -2 & 2
            \end{pmatrix}
            \\
            \\
            \\
            \lambda I=\lambda \begin{pmatrix}
              1 & 0 & 0
              \\
              0 & 1 & 0
              \\
              0 & 0 & 1
            \end{pmatrix}=\begin{pmatrix}
              \lambda & 0 & 0
              \\
              0 & \lambda & 0
              \\
              0 & 0 & \lambda
            \end{pmatrix}
            \\
            \\
            \\
            A-\lambda I=\begin{pmatrix}
              -2 & 0 & 0 
              \\
              3 & 7 & 2 
              \\
              1 & -2 & 2
            \end{pmatrix}-\begin{pmatrix}
              \lambda & 0 & 0
              \\
              0 & \lambda & 0
              \\
              0 & 0 & \lambda
            \end{pmatrix}=\begin{pmatrix}
              -2-\lambda & 0 & 0
              \\
              3 & 7-\lambda & 2
              \\
              1 & -2 & 2-\lambda
            \end{pmatrix}
            \\
            \\
            \\
            \\
            det\left(A-\lambda I\right)=\begin{vmatrix}
              -2-\lambda & 0 & 0
              \\
              3 & 7-\lambda & 2
              \\
              1 & -2 & 2-\lambda
            \end{vmatrix}=\left(-2-\lambda\right) (-1)^{1+1} \begin{vmatrix}
              7-\lambda & 2
              \\
              -2 & 2-\lambda
            \end{vmatrix}
            \\
            \\
            \\
            =\left(-2-\lambda\right) \left(7-\lambda\right) \left(2-\lambda\right)+4
            =-\lambda^3+7\lambda^2-36=\left(-\lambda-2\right) \left(\lambda-3\right) \left(\lambda-6\right)
            \\
            \\
            \\
            det\left(A-\lambda I\right)=\left(-\lambda-2\right) \left(\lambda-3\right) \left(\lambda-6\right)=0
            \\
            \\
            \begin{cases}
              \lambda=-2
              \\
              \lambda=3
              \\
              \lambda=6
            \end{cases}
          $
          \\
          \\
          Now we need to find the corresponding eigenvectors for each of the above eigenvalues.
          \\
          \\
          \rule{15cm}{3pt}
          \\
          \\
          $
            \lambda=-2 \Longrightarrow A-\lambda I=\begin{pmatrix}
              0 & 0 & 0
              \\
              3 & 9 & 2
              \\
              1 & -2 & 4
            \end{pmatrix}
            \\
            \\
            \\
            \left(\begin{array}{ccc|c}  
              0 & 0 & 0 & 0
              \\  
              3 & 9 & 2 & 0
              \\
              1 & -2 & 4 & 0
            \end{array}\right)
            \\
            \\
            \rule{15cm}{2pt}
            \\
            \\
            R_2 \leftrightarrow  R_1
            \\
            \\
            \left(\begin{array}{ccc|c}
              3 & 9 & 2 & 0
              \\
              0 & 0 & 0 & 0
              \\
              1 & -2 & 4 & 0
            \end{array}\right)
            \\
            \\
            \rule{15cm}{2pt}
            \\
            \\
            \dfrac{R_1}{3} \rightarrow R_1
            \\
            \\
            \left(\begin{array}{ccc|c}
              1 & 3 & \dfrac{2}{3} & 0
              \\
              0 & 0 & 0 & 0
              \\
              1 & -2 & 4 & 0
            \end{array}\right)
            \\
            \\
            \rule{15cm}{2pt}
            \\
            \\
            R_3-R_1 \rightarrow R_3
            \\
            \\
            \left(\begin{array}{ccc|c}
              1 & 3 & \dfrac{2}{3} & 0
              \\
              0 & 0 & 0 & 0
              \\
              1 & -5 & \dfrac{10}{3} & 0
            \end{array}\right)
            \\
            \\
            \rule{15cm}{2pt}
            \\
            \\
            -\dfrac{R_2}{5} \rightarrow R_2
            \\
            \\
            \left(\begin{array}{ccc|c}
              1 & 3 & \dfrac{2}{3} & 0
              \\
              0 & 1 & -\dfrac{2}{3} & 0
              \\
              0 & 0 & 0 & 0
            \end{array}\right)
            \\
            \\
            \rule{15cm}{2pt}
            \\
            \\
            R_1-3R_2 \rightarrow R_1
            \\
            \\
            \left(\begin{array}{ccc|c}
              1 & 0 & \dfrac{8}{3} & 0
              \\
              0 & 1 & -\dfrac{2}{3} & 0
              \\
              0 & 0 & 0 & 0
            \end{array}\right)
            \\
            \\
            \\
            \Longrightarrow \begin{cases}
              x_1+\dfrac{8}{3}x_3=0
              \\
              \\
              x_2-\dfrac{2}{3}x_3=0
            \end{cases} \Longrightarrow \begin{cases}
              x_1=-\dfrac{8}{3} x_3
              \\
              \\
              x_2=\dfrac{2}{3} x_3
              \\
              \\
              x_3=x_3
            \end{cases}
          $
          \\
          \\
          Let's set $x_3=1$ then we have $X=\begin{cases}
            -\dfrac{8}{3} 
            \\
            \\
            \dfrac{2}{3}
            \\
            \\
            1
          \end{cases}$
          \\
          \\
          \rule{15cm}{3pt}
          \\
          \\
          $
            \lambda=3 \Longrightarrow A-\lambda I=\begin{pmatrix}
              -5 & 0 & 0
              \\
              3 & 4 & 2
              \\
              1 & -2 & -1
            \end{pmatrix}
            \\
            \\
            \left(\begin{array}{ccc|c}
              -5 & 0 & 0 & 0
              \\
              3 & 4 & 2 & 0
              \\
              1 & -2 & -1 & 0
            \end{array}\right)
            \\
            \\
            \rule{15cm}{2pt}
            \\
            \\
            -\dfrac{R_1}{5} \rightarrow R_1
            \\
            \\
            \left(\begin{array}{ccc|c}
              1 & 0 & 0 & 0
              \\
              3 & 4 & 2 & 0
              \\
              1 & -2 & -1 & 0
            \end{array}\right)
            \\
            \\
            \rule{15cm}{2pt}
            \\
            \\
            R_2-3R_1 \rightarrow 
            \\
            \\
            \left(\begin{array}{ccc|c}
              1 & 0 & 0 & 0
              \\
              0 & 4 & 2 & 0
              \\
              1 & -2 & -1 & 0
            \end{array}\right)
            \\
            \\
            \rule{15cm}{2pt}
            \\
            \\
            R_3-R_1 \rightarrow R_3
            \\
            \\
            \left(\begin{array}{ccc|c}
              1 & 0 & 0 & 0
              \\
              0 & 4 & 2 & 0
              \\
              0 & -2 & -1 & 0
            \end{array}\right)
            \\
            \\
            \rule{15cm}{2pt}
            \\
            \\
            R_3+R_2 \rightarrow R_3
            \\
            \\
            \left(\begin{array}{ccc|c}
              1 & 0 & 0 & 0
              \\
              0 & 1 & \dfrac{1}{2} & 0
              \\
              0 & 0 & 0 & 0
            \end{array}\right)
            \\
            \\
            \\
            \Longrightarrow \begin{cases}
              x_1=0
              \\
              \\
              x_2+\dfrac{1}{2}x_3=0
            \end{cases} \Longrightarrow X=\begin{pmatrix}
              0
              \\
              -\dfrac{1}{2} x_3
              \\
              x_3
            \end{pmatrix}
          $
          \\
          \\
          Let's set $x_3=1$ then we have $X=\begin{pmatrix}
            0
            \\
            \\
            -\dfrac{1}{2}
            \\
            \\
            1
          \end{pmatrix}$
          \\
          \\
          \rule{15cm}{3pt}
          \\
          \\
          $
            \lambda=6 \Longrightarrow A-\lambda I=\begin{pmatrix}
              -8 & 0 & 0
              \\
              3 & 1 & 2
              \\
              1 & -2 & -4
            \end{pmatrix}
            \\
            \\
            \rule{15cm}{2pt}
            \\
            \\
            -\dfrac{1}{8}R_1 \rightarrow R_1
            \\
            \\
            \left(\begin{array}{ccc|c}
              1 & 0 & 0 & 0
              \\
              3 & 1 & 2 & 0
              \\
              1 & -2 & -4 & 0
            \end{array}\right)
            \\
            \\
            \rule{15cm}{2pt}
            \\
            \\
            R_2-3R_1 \rightarrow R_2
            \\
            \\
            \left(\begin{array}{ccc|c}
              1 & 0 & 0 & 0
              \\
              0 & 1 & 2 & 0
              \\
              1 & -2 & -4 & 0
            \end{array}\right)
            \\
            \\
            \rule{15cm}{2pt}
            \\
            \\
            R_3+2R_2 \rightarrow R_3
            \\
            \\
            \left(\begin{array}{ccc|c}
              1 & 0 & 0 & 0
              \\
              0 & 1 & 2 & 0
              \\
              0 & 0 & 0 & 0
            \end{array}\right)
            \\
            \\
            \\
            \Longrightarrow \begin{cases}
              x_1=0
              \\
              x_2+2x_3=0
            \end{cases} \Longrightarrow X=\begin{pmatrix}
              0
              \\
              -2x_3
              \\
              x_3
            \end{pmatrix}
          $
          \\
          \\
          Let's set $x_3=1$ then we have $X=\begin{pmatrix}
            0
            \\
            -2
            \\
            1
          \end{pmatrix}$
          \\
          \\
          \\
          \\
          $
            \therefore ~~~~~ \begin{cases}
              \lambda=-2 \Longrightarrow X=\begin{pmatrix}
                -\dfrac{8}{3}
                \\
                \\
                \dfrac{3}{2}
                \\
                \\
                1
              \end{pmatrix}
              \\
              \\
              \lambda=3 \Longrightarrow X=\begin{pmatrix}
                0
                \\
                \\
                -\dfrac{1}{2}
                \\
                \\
                1
              \end{pmatrix}
              \\
              \\
              \lambda=6 \Longrightarrow X=\begin{pmatrix}
                0
                \\
                \\
                -2
                \\
                \\
                1
              \end{pmatrix} 
            \end{cases} ~~~~ \checkmark
          $
        }

    \pagebreak

    \item (25 points) Solve the following system of linear equations:
    $$
      \begin{cases}
        x+y-z-w=2
        \\
        x-y-z+w=6
        \\
        x+y-3z-2w=4
        \\
        -x+y+2z+w=-1
      \end{cases}
    $$

      \textcolor{hwColor}{
        $
          A=\left(\begin{array}{cccc|c}  
            1 & 1 & -1 & -1 & 2 \\  
            1 & -1 & -1 & 1 & 6 \\
            1 & 1 & -3 & -2 & 4 \\
            -1 & 1 & 2 & 1 & -1 
          \end{array}\right)
          \\
          \\
          \\
          \rule{15cm}{2pt}
          \\
          \\
          -R_1+R_2 \rightarrow R_2
          \\
          \\
          =\left(\begin{array}{cccc|c}  
            1 & 1 & -1 & -1 & 2 \\  
            0 & -2 & 0 & 2 & 4 \\
            1 & 1 & -3 & -2 & 4 \\
            -1 & 1 & 2 & 1 & -1 
          \end{array}\right)
          \\
          \\
          \rule{15cm}{2pt}
          \\
          \\
          -R_1+R_3 \rightarrow R_3
          \\
          \\
          =\left(\begin{array}{cccc|c}  
            1 & 1 & -1 & -1 & 2 \\  
            0 & -2 & 0 & 2 & 4 \\
            0 & 0 & -2 & -1 & 2 \\
            -1 & 1 & 2 & 1 & -1 
          \end{array}\right)
          \\
          \\
          \rule{15cm}{2pt}
          \\
          \\
          -R_1+R_4 \rightarrow R_4
          \\
          \\
          =\left(\begin{array}{cccc|c}  
            1 & 1 & -1 & -1 & 2 \\  
            0 & -2 & 0 & 2 & 4 \\
            0 & 0 & -2 & -1 & 2 \\
            0 & 2 & 1 & 0 & 1 
          \end{array}\right)
          \\
          \\
          \rule{15cm}{2pt}
          \\
          \\
          -\dfrac{1}{2}R_2 \rightarrow R_2
          \\
          \\
          =\left(\begin{array}{cccc|c}  
            1 & 1 & -1 & -1 & 2 \\  
            0 & 1 & 0 & -1 & -2 \\
            0 & 0 & -2 & -1 & 2 \\
            0 & 2 & 1 & 0 & 1 
          \end{array}\right)
          \\
          \\
          \rule{15cm}{2pt}
          \\
          \\
          -2R_2+R_4 \rightarrow R_4
          \\
          \\
          =\left(\begin{array}{cccc|c}  
            1 & 1 & -1 & -1 & 2 \\  
            0 & 1 & 0 & -1 & -2 \\
            0 & 0 & -2 & -1 & 2 \\
            0 & 0 & 1 & 2 & 5 
          \end{array}\right)
          \\
          \\
          \rule{15cm}{2pt}
          \\
          \\
          -\dfrac{1}{2}R_3 \rightarrow R_3
          \\
          \\
          =\left(\begin{array}{cccc|c}  
            1 & 1 & -1 & -1 & 2 \\  
            0 & 1 & 0 & -1 & -2 \\
            0 & 0 & 1 & \dfrac{1}{2} & -1 \\
            0 & 0 & 1 & 2 & 5 
          \end{array}\right)
          \\
          \\
          \rule{15cm}{2pt}
          \\
          \\
          -R_3+R_4 \rightarrow R_4
          \\
          \\
          =\left(\begin{array}{cccc|c}  
            1 & 1 & -1 & -1 & 2 \\  
            0 & 1 & 0 & -1 & -2 \\
            0 & 0 & 1 & \dfrac{1}{2} & -1 \\
            0 & 0 & 0 & \dfrac{3}{2} & 6 
          \end{array}\right)
          \\
          \\
          \rule{15cm}{2pt}
          \\
          \\
          \dfrac{2}{3}R_4 \rightarrow R_4
          \\
          \\
          =\left(\begin{array}{cccc|c}  
            1 & 1 & -1 & -1 & 2 \\  
            0 & 1 & 0 & -1 & -2 \\
            0 & 0 & 1 & \dfrac{1}{2} & -1 \\
            0 & 0 & 0 & 1 & 4 
          \end{array}\right)
          \\
          \\
          \rule{15cm}{2pt}
          \\
          \\
          -\dfrac{1}{2}R_4+R_3 \rightarrow R_3
          \\
          \\
          =\left(\begin{array}{cccc|c}  
            1 & 1 & -1 & -1 & 2 \\  
            0 & 1 & 0 & -1 & -2 \\
            0 & 0 & 1 & 0 & -3 \\
            0 & 0 & 0 & 1 & 4 
          \end{array}\right)
          \\
          \\
          \rule{15cm}{2pt}
          \\
          \\
          R_4+R_2 \rightarrow R_2
          \\
          \\
          =\left(\begin{array}{cccc|c}  
            1 & 1 & -1 & -1 & 2 \\  
            0 & 1 & 0 & 0 & 2 \\
            0 & 0 & 1 & 0 & -3 \\
            0 & 0 & 0 & 1 & 4 
          \end{array}\right)
          \\
          \\
          \rule{15cm}{2pt}
          \\
          \\
          R_4+R_1 \rightarrow R_1
          \\
          \\
          =\left(\begin{array}{cccc|c}  
            1 & 1 & -1 & 0 & 6 \\  
            0 & 1 & 0 & 0 & 2 \\
            0 & 0 & 1 & 0 & -3 \\
            0 & 0 & 0 & 1 & 4 
          \end{array}\right)
          \\
          \\
          \rule{15cm}{2pt}
          \\
          \\
          R_3+R_1 \rightarrow R_1
          \\
          \\
          =\left(\begin{array}{cccc|c}  
            1 & 1 & 0 & 0 & 3 \\  
            0 & 1 & 0 & 0 & 2 \\
            0 & 0 & 1 & 0 & -3 \\
            0 & 0 & 0 & 1 & 4 
          \end{array}\right)
          \\
          \\
          \rule{15cm}{2pt}
          \\
          \\
          -R_2+R_1 \rightarrow R_1
          \\
          \\
          =\left(\begin{array}{cccc|c}  
            1 & 0 & 0 & 0 & 1 \\  
            0 & 1 & 0 & 0 & 2 \\
            0 & 0 & 1 & 0 & -3 \\
            0 & 0 & 0 & 1 & 4 
          \end{array}\right) ~~~~ \checkmark
          \\
          \\
          \rule{15cm}{2pt}
          \\
          \\
          \\
          \therefore ~~~~ \begin{cases}
            x=1
            \\
            y=2
            \\
            z=-3
            \\
            w=4
          \end{cases}  ~~~~ \checkmark
        $
      }

    \pagebreak

    \item (20 points) Find both a basis for the row space and a basis for the column space of the matrix
    $$
      \begin{pmatrix}
        2 & -4 & 2 & -8
        \\
        -2 & 1 & 1 & 0
        \\
        0 & -3 & 3 & -8
      \end{pmatrix}
    $$
    What is the rank of this matrix? Find the nullspace basis of this matrix.

      \textcolor{hwColor}{
        We start by putting the matrix in row echelon form.
        \\
        \\
        $
          A=\begin{pmatrix}
            2 & -4 & 2 & -8
            \\
            -2 & 1 & 1 & 0
            \\
            0 & -3 & 3 & -8
          \end{pmatrix}
          \\
          \\
          \rule{15cm}{2pt}
          \\
          \\
          \dfrac{1}{2} R_1 \rightarrow R_1
          \\
          \\
          =\begin{pmatrix}
            1 & -2 & 1 & -4
            \\
            -2 & 1 & 1 & 0
            \\
            0 & -3 & 3 & -8
          \end{pmatrix}
          \\
          \\
          \rule{15cm}{2pt}
          \\
          \\
          2R_1+R_2 \rightarrow R_2
          \\
          \\
          =\begin{pmatrix}
            1 & -2 & 1 & -4
            \\
            0 & -3 & 3 & -8
            \\
            0 & -3 & 3 & -8
          \end{pmatrix}
          \\
          \\
          \rule{15cm}{2pt}
          \\
          \\
          -\dfrac{1}{3}R_2 \rightarrow R_2
          \\
          \\
          =\begin{pmatrix}
            1 & -2 & 1 & -4
            \\
            \\
            0 & 1 & -1 & \dfrac{8}{3}
            \\
            \\
            0 & -3 & 3 & -8
          \end{pmatrix}
          \\
          \\
          \rule{15cm}{2pt}
          \\
          \\
          3R_2+R_3 \rightarrow R_3
          \\
          \\
          =\begin{pmatrix}
            1 & -2 & 1 & -4
            \\
            \\
            0 & 1 & -1 & \dfrac{8}{3}
            \\
            \\
            0 & 0 & 0 & 0
          \end{pmatrix}
          \\
          \\
          \rule{15cm}{2pt}
          \\
          \\
          3R_2+R_3 \rightarrow R_3
          \\
          \\
          =\begin{pmatrix}
            1 & 0 & -1 & \dfrac{4}{3}
            \\
            \\
            0 & 1 & -1 & \dfrac{8}{3}
            \\
            \\
            0 & 0 & 0 & 0
          \end{pmatrix} ~~~~ \checkmark
        $
        \\
        \\
        \\
        \textbf{Basis of Row Space:}
        \\
        Row operations do not change the row space, so the rows of the matrix at the end have the
        same span as our given matrix. Furthermore, the nonzero rows of a matrix in row echelon
        form are linearly independent. The nonzero rows in the reduced row-echelon form are a basis for the row space.
        \\
        \\
        $
          \{
            \begin{bmatrix}
              1 & 0 & -1 & \dfrac{4}{3}
            \end{bmatrix},
            \begin{bmatrix}
              0 & 1 & -1 & \dfrac{8}{3}
            \end{bmatrix}
          \} ~~~~ \checkmark
        $
        \\
        \\
        \\
        \textbf{Basis of Column Space:}
        \\
        Recall that a leading one is the first nonzero entry in a row. The columns containing leading
        ones are the pivot columns. To obtain a basis for the column space, we just use the pivot
        columns from the original matrix.
        \\
        \\ 
        $
        \{
            \begin{bmatrix}
              2
              \\
              -2
              \\
              0
            \end{bmatrix},
            \begin{bmatrix}
              -4
              \\
              1
              \\
              3
            \end{bmatrix}  
        \}  ~~~~ \checkmark
        $
        \\
        \\
        \\
        Finally, the row space and column space each have bases with three vectors, so they have
        dimension two. Therefore, the rank of $A$ is 2.
        \\
        \\
        Based on the results we have found so far, the following system of equations can be written:
        \\
        \\
        $
          \begin{cases}
            x_1-x_3+\dfrac{4}{3}x_4=0
            \\
            \\
            x_2-x_3+\dfrac{8}{3}x_4=0
            \\
            \\
            x_3=arbitrary
            \\
            \\
            x_4=arbitrary
          \end{cases}
        $
        \\
        \\
        The system has infinitely many solutions! we can write the solution in the vector form:
        \\
        \\
        $
          \begin{bmatrix}
            1
            \\
            1
            \\
            1
            \\
            0
          \end{bmatrix}x_3+\begin{bmatrix}
            -\dfrac{4}{3}
            \\
            \\
            -\dfrac{8}{3}
            \\
            \\
            0
            \\
            \\
            1
          \end{bmatrix}x_4=\begin{bmatrix}
            x_1
            \\
            x_2
            \\
            x_3
            \\
            x_4
          \end{bmatrix}
        $
        \\
        \\
        Therefore the \textbf{null space} has a basis formed by the set $
          \{
            \begin{bmatrix}
              1
              \\
              1
              \\
              1
              \\
              0
            \end{bmatrix},
            \begin{bmatrix}
              -\dfrac{4}{3}
              \\
              \\
              -\dfrac{8}{3}
              \\
              \\
              0
              \\
              \\
              1
            \end{bmatrix}
          \}. ~~~~ \checkmark
          \\
          \\
        $
      }


    \item (20 points) The linear transformation, $L:\mathbb{R}^3 \rightarrow \mathbb{R}^2$ is given by 
    $$
      L(x_1, x_2, x_3)=\left(x_1+x_2-x_3, ~ x_1-x_2+x_3\right)
    $$ 
    Find the matrix representation of $L$ with respect to the canonical bases. What is the kernel
    of this transformation?

      \textcolor{hwColor}{
        \\
        \\
        $L:\mathbb{R}^3 \rightarrow \mathbb{R}^2$ is a linear transformation defined by 
        $L(x_1, x_2, x_3)=\left(x_1+x_2-x_3, ~ x_1-x_2+x_3\right)$. We know the basis vectors 
        for $\mathbb{R}^3$ are $e_1=(1, 0, 0), e_2=(0, 1, 0)$ and $e_3=(0, 0, 1)$.
        Based on the given transformation we have:
        \\
        \\
        $
          L\left(\begin{bmatrix}
            x_1 
            \\
            x_2
            \\
            x_3
          \end{bmatrix}\right)
          =L\left(
            \begin{bmatrix}
              x_1+x_2-x_3 
              \\
              x_1-x_2+x_3
            \end{bmatrix}
          \right)
          \\
          \\
          \\
          \Longrightarrow \begin{cases}
            L\left(\begin{bmatrix}
              1 
              \\
              0
              \\
              0
            \end{bmatrix}\right)=\begin{bmatrix}
              1+0-0 
              \\
              1-0+0
            \end{bmatrix}=\begin{bmatrix}
              1 
              \\
              1
            \end{bmatrix}=1\begin{bmatrix}
              1 
              \\
              0
            \end{bmatrix}+1\begin{bmatrix}
              0 
              \\
              1
            \end{bmatrix}
            \\
            \\
            L\left(\begin{bmatrix}
              0 
              \\
              1
              \\
              0
            \end{bmatrix}\right)=\begin{bmatrix}
              0+1-0 
              \\
              0-1+0
            \end{bmatrix}=\begin{bmatrix}
              1 
              \\
              -1
            \end{bmatrix}=1\begin{bmatrix}
              1 
              \\
              0
            \end{bmatrix}+(-1)\begin{bmatrix}
              0 
              \\
              1
            \end{bmatrix}
            \\
            \\
            L\left(\begin{bmatrix}
              0 
              \\
              0
              \\
              1
            \end{bmatrix}\right)=\begin{bmatrix}
              0+0-1 
              \\
              0-0+1
            \end{bmatrix}=\begin{bmatrix}
              -1 
              \\
              1
            \end{bmatrix}=(-1)\begin{bmatrix}
              1 
              \\
              0
            \end{bmatrix}+1\begin{bmatrix}
              0 
              \\
              1
            \end{bmatrix}
          \end{cases} ~~~~ \checkmark
        $
        \\
        \\
        \\
        Therefore, the matrix for $L$ relative to the standard bases is $\begin{pmatrix}
          1 & 1 & -1
          \\
          1 & -1 & 1
        \end{pmatrix}$.
        \\
        \\
        \\
        Recall that the kernel of a transformation is the subset of the domain which maps into the
        zero vector. Depending on what we pick for x3 we have the following:
        \\
        \\
        $
          L\left(
            \begin{bmatrix}
              x_1+x_2-x_3 
              \\
              x_1-x_2+x_3
            \end{bmatrix}
          \right)=\begin{bmatrix}
            0 
            \\
            0
          \end{bmatrix}
        $
        \\
        \\
        If we set $x_3=1$ then we have:
        \\
        \\
        $
          \begin{pmatrix}
            x_1+x_2=1
            \\
            x_1-x_2=-1
          \end{pmatrix} \Longrightarrow \begin{cases}
            x_1=0
            \\
            x_2=1
          \end{cases}
          \\
          \\
          \\
          \therefore ~~~~~ ker(L)=\left(0, 1, 1\right) ~~~~ \checkmark
        $
      }

    \item (20 points) Solve the following initial value problem:
    $$
      \dfrac{d}{dt}x=\begin{pmatrix}
        -2 & 0 & 0
        \\
        3 & 7 & 2
        \\
        1 & -2 & 2
      \end{pmatrix}x, ~~~~~ x(0)=\begin{pmatrix}
        0
        \\
        1
        \\
        -2
      \end{pmatrix}
    $$

      \textcolor{hwColor}{
        $
          \begin{pmatrix}
            -2 & 0 & 0
            \\
            3 & 7 & 2
            \\
            1 & -2 & 2
          \end{pmatrix} 
        $ matrix is the exact same matrix given in the previous question 4. The way we
        are going to solve this problem is to calculate the eigenvalues of the matrix 
        and the corresonding eigenvectors. We already did that in question 4. so we have:
        \\
        \\
        $
          \begin{cases}
            \lambda=-2 \Longrightarrow X=\begin{pmatrix}
              -\dfrac{8}{3}
              \\
              \\
              \dfrac{3}{2}
              \\
              \\
              1
            \end{pmatrix}
            \\
            \\
            \lambda=3 \Longrightarrow X=\begin{pmatrix}
              0
              \\
              \\
              -\dfrac{1}{2}
              \\
              \\
              1
            \end{pmatrix}
            \\
            \\
            \lambda=6 \Longrightarrow X=\begin{pmatrix}
              0
              \\
              \\
              -2
              \\
              \\
              1
            \end{pmatrix}
          \end{cases}
          \\
          \\
          \\
          \overrightarrow{X}=e^{\lambda t} \overrightarrow{v}, ~~~ \overrightarrow{v} ~~ constant \neq 0
          \\
          \\
          \\
        $
        \textbf{General solution:}
        \\
        \\
        $
          \overrightarrow{X}=C_1 e^{\lambda t} \begin{pmatrix}
            -\dfrac{8}{3}
            \\
            \\
            \dfrac{3}{2}
            \\
            \\
            1
          \end{pmatrix}+C_2 e^{\lambda t} \begin{pmatrix}
            0
            \\
            \\
            -\dfrac{1}{2}
            \\
            \\
            1
          \end{pmatrix}+C_3 e^{\lambda t} \begin{pmatrix}
            0
            \\
            \\
            -2
            \\
            \\
            1
          \end{pmatrix}
          \\
          \\
          \\
          =C_1 e^{-2 t} \begin{pmatrix}
            -\dfrac{8}{3}
            \\
            \\
            \dfrac{3}{2}
            \\
            \\
            1
          \end{pmatrix}+C_2 e^{3 t} \begin{pmatrix}
            0
            \\
            \\
            -\dfrac{1}{2}
            \\
            \\
            1
          \end{pmatrix}+C_3 e^{6 t} \begin{pmatrix}
            0
            \\
            \\
            -2
            \\
            \\
            1
          \end{pmatrix}
        $
        \\
        \\
        \\
        \\
        \\
        \textbf{Initial-value problem:}
        \\
        \\
        $
          x(0)=\begin{pmatrix}
            0
            \\
            1
            \\
            -2
          \end{pmatrix}=C_1 \begin{pmatrix}
            -\dfrac{8}{3}
            \\
            \\
            \dfrac{3}{2}
            \\
            \\
            1
          \end{pmatrix}+C_2 \begin{pmatrix}
            0
            \\
            \\
            -\dfrac{1}{2}
            \\
            \\
            1
          \end{pmatrix}+C_3 \begin{pmatrix}
            0
            \\
            \\
            -2
            \\
            \\
            1
          \end{pmatrix}
          \\
          \\
          \\
          \Longrightarrow \begin{pmatrix}
            -\dfrac{8}{3} C_1
            \\
            \\
            \dfrac{3}{2}C_1
            \\
            \\
            C_1
          \end{pmatrix}+\begin{pmatrix}
            0
            \\
            \\
            -\dfrac{1}{2}C_2
            \\
            \\
            C_2
          \end{pmatrix}+\begin{pmatrix}
            0
            \\
            \\
            -2C_3
            \\
            \\
            C_3
          \end{pmatrix}=\begin{pmatrix}
            0
            \\
            \\
            1
            \\
            \\
            -2
          \end{pmatrix}
          \\
          \\
          \\
          \Longrightarrow \begin{cases}
            -\dfrac{8}{3} C_1=0
            \\
            \\
            \dfrac{3}{2} C_1-\dfrac{1}{2} C_2-2C_3=1
            \\
            \\
            C_1+C_2+C_3=-2
          \end{cases}
          \\
          \\
          \\
          \Longrightarrow \begin{pmatrix}
            -\dfrac{8}{3} & 0 & 0
            \\
            \\
            \dfrac{3}{2} & -\dfrac{1}{2} & -2
            \\
            \\
            1 & 1 & 1
          \end{pmatrix} \begin{pmatrix}
            C_1
            \\
            \\
            C_2
            \\
            \\
            C_3
          \end{pmatrix}=\begin{pmatrix}
            0
            \\
            \\
            1
            \\
            \\
            -2
          \end{pmatrix}
        $
        \\
        \\
        \\
        Now can rewrite this in the reduced Row-echelon form:
        \\
        \\
        $
          \left(\begin{array}{ccc|c}
            -\dfrac{8}{3} & 0 & 0 & 0
            \\
            \dfrac{3}{2} & -\dfrac{1}{2} & -2 & 1
            \\
            1 & 1 & 1 & -2
          \end{array}\right)
          \\
          \\
          \rule{15cm}{2pt}
          \\
          \\
          -\dfrac{3}{8}R_1 \rightarrow R_1
          \\
          \\
          \left(\begin{array}{ccc|c}
            1 & 0 & 0 & 0
            \\
            \dfrac{3}{2} & -\dfrac{1}{2} & -2 & 1
            \\
            1 & 1 & 1 & -2
          \end{array}\right)
          \\
          \\
          \rule{15cm}{2pt}
          \\
          \\
          -\dfrac{3}{2}R_1+R_2 \rightarrow R_2
          \\
          \\
          \left(\begin{array}{ccc|c}
            1 & 0 & 0 & 0
            \\
            0 & -\dfrac{1}{2} & -2 & 1
            \\
            1 & 1 & 1 & -2
          \end{array}\right)
          \\
          \\
          \rule{15cm}{2pt}
          \\
          \\
          -R_1+R_3 \rightarrow R_3
          \\
          \\
          \left(\begin{array}{ccc|c}
            1 & 0 & 0 & 0
            \\
            0 & -\dfrac{1}{2} & -2 & 1
            \\
            0 & 1 & 1 & -2
          \end{array}\right)
          \\
          \\
          \rule{15cm}{2pt}
          \\
          \\
          -2R_2 \rightarrow R_2
          \\
          \\
          \left(\begin{array}{ccc|c}
            1 & 0 & 0 & 0
            \\
            0 & 1 & 4 & -2
            \\
            0 & 1 & 1 & -2
          \end{array}\right)
          \\
          \\
          \rule{15cm}{2pt}
          \\
          \\
          -R_2+R_3 \rightarrow R_3
          \\
          \\
          \left(\begin{array}{ccc|c}
            1 & 0 & 0 & 0
            \\
            0 & 1 & 4 & -2
            \\
            0 & 0 & -3 & 0
          \end{array}\right)
          \\
          \\
          \rule{15cm}{2pt}
          \\
          \\
          -\dfrac{1}{3}R_3 \rightarrow R_3
          \\
          \\
          \left(\begin{array}{ccc|c}
            1 & 0 & 0 & 0
            \\
            0 & 1 & 4 & -2
            \\
            0 & 0 & 1 & 0
          \end{array}\right)
          \\
          \\
          \rule{15cm}{2pt}
          \\
          \\
          -4R_3+R_2 \rightarrow R_2
          \\
          \\
          \left(\begin{array}{ccc|c}
            1 & 0 & 0 & 0
            \\
            0 & 1 & 0 & -2
            \\
            0 & 0 & 1 & 0
          \end{array}\right)
          \\
          \\
          \\
          \\
          \therefore ~~~~~ \begin{cases}
            C_1=0
            \\
            C_2=-2
            \\
            C_3=0
          \end{cases}
        $
        \\
        \\
        Therfore now we can find $\overrightarrow{X}$.
        \\
        \\
        $
          \overrightarrow{X}=0 e^{-2 t} \begin{pmatrix}
            -\dfrac{8}{3}
            \\
            \\
            \dfrac{3}{2}
            \\
            \\
            1
          \end{pmatrix}+(-2) e^{3 t} \begin{pmatrix}
            0
            \\
            \\
            -\dfrac{1}{2}
            \\
            \\
            1
          \end{pmatrix}+0 e^{6 t} \begin{pmatrix}
            0
            \\
            \\
            -2
            \\
            \\
            1
          \end{pmatrix}
          \\
          \\
          \\
          \\
          \therefore ~~~~~ \overrightarrow{X}=-2e^{3 t} \begin{pmatrix}
            0
            \\
            \\
            -\dfrac{1}{2}
            \\
            \\
            1
          \end{pmatrix} ~~~~ \checkmark
        $
      }


    \item (10 points) (The Triangle Inequality.) Let $V$ be an inner product space with an inner product
    $\left\langle u, v\right\rangle$. Prove that 
    $$
      ||u+v|| \leq ||u||+||v||, ~~~~~ ||w||^2=\left\langle w, w\right\rangle
    $$
    for all $u,v \in V$.  [Hint: You may use the Cauchy-Schwarz inequality: $|\left\langle u, v\right\rangle | \leq ||u||.||v||$]

      \textcolor{hwColor}{
        \\
        By the defintion of vector addition, $||\overrightarrow{u}||$ and $||\overrightarrow{v}||$ are two sides of a triangle, 
        where the tip of $||\overrightarrow{u}||$ is at the tail of $||\overrightarrow{v}||$, then 
        $||\overrightarrow{u}+\overrightarrow{v}||$  is the length of the third side of this triangle.
        So the triangle inequality states that the length of the third side is less than the sum of 
        the lengths of the other two sides. This is a classical theorem of Euclidean Geometry, written in terms of vectors.
        \\
        \\
        By using the \textbf{Cauchy–Schwarz} inequality, $||\overrightarrow{u}.\overrightarrow{v}|| \leq ||\overrightarrow{u}|| . ||\overrightarrow{v}||$
        to prove the Triangle inequality, we have:
        \\
        \\
        $
            ||\overrightarrow{u}+\overrightarrow{v}||^2
            =\left(\overrightarrow{u}+\overrightarrow{v}\right).\left(\overrightarrow{u}+\overrightarrow{v}\right)
            =\overrightarrow{u}.\overrightarrow{u}+\overrightarrow{u}.\overrightarrow{v}+\overrightarrow{v}.\overrightarrow{u}+\overrightarrow{v}.\overrightarrow{v}
            \\
            \\
            \\
            =||\overrightarrow{u}||^2+2 \overrightarrow{u}.\overrightarrow{v}+||\overrightarrow{v}||^2
        $
        \\
        \\
        Since $\overrightarrow{u}.\overrightarrow{v} \leq ||\overrightarrow{u}.\overrightarrow{v}||$ (any number is less than or equal to its absolute value)
        \\
        \\
        $
          ||\overrightarrow{u}+\overrightarrow{v}||^2 \leq ||\overrightarrow{u}||^2+2 ||\overrightarrow{u}.\overrightarrow{v}||+||\overrightarrow{v}||^2
        $
        \\
        \\
        Using \textbf{Cauchy–Schwarz}
        \\
        \\
        $
          ||\overrightarrow{u}+\overrightarrow{v}||^2 \leq ||\overrightarrow{u}||^2+2 ||\overrightarrow{u}||.||\overrightarrow{v}||+||\overrightarrow{v}||^2
          \\
          \\
          \\
          \Longrightarrow ||\overrightarrow{u}||^2+||\overrightarrow{v}||^2+2 ||\overrightarrow{u}|| ||\overrightarrow{v}||=\left(||\overrightarrow{u}||+||\overrightarrow{v}||\right)^2
          \\
          \\
          \\
          \Longrightarrow ||\overrightarrow{u}+\overrightarrow{v}||^2 \leq \left(||\overrightarrow{u}||+||\overrightarrow{v}||\right)^2
        $
        \\
        \\
        On both sides of this inequality, the quantity under the square is known to be positive, so we can remove the squares.
        \\
        \\
        $
          \therefore ~~~~~ ||\overrightarrow{u}+\overrightarrow{v}|| \leq ||\overrightarrow{u}||+||\overrightarrow{v}|| ~~~~ \checkmark
        $
      }


  \end{enumerate}

\end{document}
