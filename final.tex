\documentclass[fleqn]{article}
\oddsidemargin 0.0in
\textwidth 6.0in
\thispagestyle{empty}
\usepackage{import}
\usepackage{amsmath}
\usepackage{graphicx}
\usepackage{flexisym}
\usepackage{amssymb}
\usepackage{bigints} 
\usepackage[english]{babel}
\usepackage[utf8x]{inputenc}
\usepackage{float}
\usepackage[colorinlistoftodos]{todonotes}

\definecolor{hwColor}{HTML}{AD53BA}

\begin{document}

  \begin{titlepage}

    \newcommand{\HRule}{\rule{\linewidth}{0.5mm}}

    \center


    \textsc{\LARGE Arizona State University}\\[1.5cm]

    \textsc{\LARGE Linear Algebra }\\[1.5cm]


    \begin{figure}
      \includegraphics[width=\linewidth]{asu.png}
    \end{figure}


    \HRule \\[0.4cm]
    { \huge \bfseries Final}\\[0.4cm] 
    \HRule \\[1.5cm]

    \textbf{Behnam Amiri}

    \bigbreak

    \textbf{Prof: Sergei Suslov}

    \bigbreak


    \textbf{{\large \today}\\[2cm]}

    \vfill

  \end{titlepage}

  \begin{enumerate}
    \item Let 
    $$
      A=\begin{pmatrix}
        -7 & -3 & 1
        \\
        0 & 3 & -1 
        \\
        -3 & 4 & -2
      \end{pmatrix}
    $$
      \begin{itemize}
        \item (20 points) Find $A^{-1}$.

          \textcolor{hwColor}{
            $
              A^{-1}=\dfrac{1}{det(A)} adj(A) 
              \\
              \\
              \\
              adj(A)=\left(A_{ij}\right)^T=\begin{pmatrix}
                +\begin{vmatrix}
                  3 & -1 
                  \\
                  4 & -2
                \end{vmatrix} 
                & -\begin{vmatrix}
                  0 & -1 
                  \\
                  -3 & -2
                \end{vmatrix} 
                & +\begin{vmatrix}
                  0 & 3 
                  \\
                  -3 & 4
                \end{vmatrix}
                \\
                \\
                -\begin{vmatrix}
                  -3 & 1 
                  \\
                  4 & -2
                \end{vmatrix} 
                & +\begin{vmatrix}
                  -7 & 1 
                  \\
                  -3 & -2
                \end{vmatrix} 
                & -\begin{vmatrix}
                  -7 & -3
                  \\
                  -3 & 4
                \end{vmatrix}
                \\
                \\
                +\begin{vmatrix}
                  -3 & 1
                  \\
                  3 & -1
                \end{vmatrix} 
                & -\begin{vmatrix}
                  -7 & 1
                  \\
                  0 & -1
                \end{vmatrix} 
                & +\begin{vmatrix}
                  -7 & -3
                  \\
                  0 & 3
                \end{vmatrix}
              \end{pmatrix}^T=\begin{pmatrix}
                -2 & 3 & 9
                \\
                -2 & 17 & 37 
                \\
                0 & -7 & -21
              \end{pmatrix}^T
              \\
              \\
              \\
              \\
              \therefore ~~~~ adj(A)=\begin{pmatrix}
                -2 & -2 & 0
                \\
                3 & 17 & -7 
                \\
                9 & 37 & -21
              \end{pmatrix} ~~~~ \checkmark
              \\
              \\
              \rule{15cm}{2pt}
              \\
              \\
              det(A)=\begin{vmatrix}
                -7 & -3 & 1
                \\
                0 & 3 & -1 
                \\
                -3 & 4 & -2
              \end{vmatrix}
              =(-7)(-1)^{1+1} \begin{vmatrix}
                3 & -1 
                \\
                4 & -2
              \end{vmatrix}+0+(-3)(-1)^{3+1} \begin{vmatrix}
                -3 & 1
                \\
                3 & -1 
              \end{vmatrix}
              \\
              \\
              =(-7)(-2)+(-3)(0)
              \\
              \\
              \\
              \therefore ~~~~ det(A)=14 ~~~~ \checkmark
            $
            \\
            \\
            \\
            Now we can find the inverse matrix:
            \\
            \\
            $
              A^{-1}=\dfrac{1}{det(A)} adj(A)=\dfrac{1}{14} \begin{pmatrix}
                -2 & -2 & 0
                \\
                3 & 17 & -7 
                \\
                9 & 37 & -21
              \end{pmatrix}
              \\
              \\
              \\
              A^{-1}=\begin{pmatrix}
                -\dfrac{1}{7} & -\dfrac{1}{7} & 0
                \\
                \\
                \dfrac{3}{14} & \dfrac{17}{14} & -\dfrac{1}{2}
                \\
                \\
                \dfrac{9}{14} & \dfrac{37}{14} & -\dfrac{3}{2}
              \end{pmatrix} ~~~~ \checkmark
            $
            \\
            \\
            \\
            \\
            \textbf{Testing the solution:}
            \\
            \\
            $
              A.A^{-1}=\begin{pmatrix}
                -7 & -3 & 1
                \\
                0 & 3 & -1 
                \\
                -3 & 4 & -2
              \end{pmatrix}.\begin{pmatrix}
                -\dfrac{1}{7} & -\dfrac{1}{7} & 0
                \\
                \\
                \dfrac{3}{14} & \dfrac{17}{14} & -\dfrac{1}{2}
                \\
                \\
                \dfrac{9}{14} & \dfrac{37}{14} & -\dfrac{3}{2}
              \end{pmatrix}=\begin{pmatrix}
                1 & 0 & 0
                \\
                0 & 1 & 0
                \\
                0 & 0 & 1
              \end{pmatrix}=I_{2 \times 2} ~~~~ \checkmark
            $ 
            \\
            \\
            \\
            The above justifies the truthy of our result.
            \\
            \\
          }


        \item (10 points) Solve the system $Ax=(1, -1, 1)^T$.

          \textcolor{hwColor}{
            $
              \\
              Ax=(1, -1, 1)^T
              \Longrightarrow
              A^{-1}.Ax=A^{-1}\begin{pmatrix}
                1
                \\
                -1
                \\
                1
              \end{pmatrix}
              \Longrightarrow
              Ix=A^{-1}\begin{pmatrix}
                1
                \\
                -1
                \\
                1
              \end{pmatrix}
              \\
              \\
              \\
              x=\begin{pmatrix}
                -\dfrac{1}{7} & -\dfrac{1}{7} & 0
                \\
                \\
                \dfrac{3}{14} & \dfrac{17}{14} & -\dfrac{1}{2}
                \\
                \\
                \dfrac{9}{14} & \dfrac{37}{14} & -\dfrac{3}{2}
              \end{pmatrix}.\begin{pmatrix}
                1
                \\
                -1
                \\
                1
              \end{pmatrix}
              =\begin{pmatrix}
                (-\dfrac{1}{7})(1)+(-\dfrac{1}{7})(-1)+(0)(1)
                \\
                \\
                (\dfrac{3}{14})(1)+(\dfrac{17}{14})(-1)+(-\dfrac{1}{2})(1)
                \\
                \\
                (\dfrac{9}{14})(1)+(\dfrac{37}{14})(-1)+(-\dfrac{3}{2})(1)
              \end{pmatrix}
              \\
              \\
              \\
              \\
              \therefore ~~~~ x=\begin{pmatrix}
                0
                \\
                \\
                -\dfrac{3}{2}
                \\
                \\
                -\dfrac{7}{2}
              \end{pmatrix} ~~~~ \checkmark
            $
          }
      \end{itemize}

    \item (10 points) Evaluate the following determinant:
    $$
      det\begin{pmatrix}
        -1 & 2 & 3
        \\
        1 & -1 & 1
        \\
        3 & 2 & 1
      \end{pmatrix}
    $$

      \textcolor{hwColor}{
        $
          \begin{vmatrix}
            -1 & 2 & 3
            \\
            1 & -1 & 1
            \\
            3 & 2 & 1
          \end{vmatrix}
          =(-1)(-1)^{1+1} \begin{vmatrix}
            -1 & 1
            \\
            2 & 1
          \end{vmatrix}
          +2(-1)^{1+2} \begin{vmatrix}
            1 & 1
            \\
            3 & 1
          \end{vmatrix}
          +3(-1)^{1+3} \begin{vmatrix}
            1 & -1
            \\
            3 & 2
          \end{vmatrix}
          \\
          \\
          \\
          =(-1)(-3)+(-2)(-2)+(3)(5)=3+4+15
          \\
          \\
          \\
          \therefore ~~~~~   det\begin{pmatrix}
            -1 & 2 & 3
            \\
            1 & -1 & 1
            \\
            3 & 2 & 1
          \end{pmatrix}=22  ~~~~ \checkmark
        $
      }

    \item (25 points) Convert the basis $v_1=(1,-1, 0), ~ v_2=(0, 1, -1),$ and $v_3=(1, 1, 1)$ for
    $\mathbb{R}^3$ into an orthonormal basis, using the Gram-Schmidt process and the standard inner product in $\mathbb{R}^3$. 


      % \textcolor{hwColor}{
            
      % }

    \item Find the eigenvalues and associated eigenvectors of a given matrix $A$.
      \begin{itemize}
        \item (15 points) 
        $A=\begin{pmatrix}
          2 & 1
          \\
          6 & -3
        \end{pmatrix}$

          % \textcolor{hwColor}{
            
          % }


        \item (25 points) $A=\begin{pmatrix}
          -2 & 0 & 0
          \\
          3 & 7 & 2 
          \\
          1 & -2 & 2
        \end{pmatrix}$
      \end{itemize}

        % \textcolor{hwColor}{
            
        % }


    \item (25 points) Solve the following system of linear equations:
    $$
      \begin{cases}
        x+y-z-w=2
        \\
        x-y-z+w=6
        \\
        x+y-3z-2w=4
        \\
        -x+y+2z+w=-1
      \end{cases}
    $$

      % \textcolor{hwColor}{
            
      % }


    \item (20 points) Find both a basis for the row space and a basis for the column space of the matrix
    $$
      \begin{pmatrix}
        2 & -4 & 2 & -8
        \\
        -2 & 1 & 1 & 0
        \\
        0 & -3 & 3 & -8
      \end{pmatrix}
    $$
    What is the rank of this matrix? Find the nullspace basis of this matrix.

      % \textcolor{hwColor}{
            
      % }


    \item (20 points) The linear transformation, $L:\mathbb{R}^3 \rightarrow \mathbb{R}^2$ is given by 
    $$
      L(x_1, x_2, x_3)=\left(x_1+x_2-x_3, ~ x_1-x_2+x_3\right)
    $$ 
    Find the matrix representation of $L$ with respect to the canonical bases. What is the kurnel
    of this transformation?

      % \textcolor{hwColor}{
            
      % }

    \item (20 points) Solve the following initial value problem:
    $$
      \dfrac{d}{dt}x=\begin{pmatrix}
        -2 & 0 & 0
        \\
        3 & 7 & 2
        \\
        1 & -2 & 2
      \end{pmatrix}x, ~~~~~ x(0)=\begin{pmatrix}
        0
        \\
        1
        \\
        -2
      \end{pmatrix}
    $$

      % \textcolor{hwColor}{
            
      % }


    \item (10 points) (The Triangle Inequality.) Let $V$ be an inner product space with an inner product
    $\left\langle u, v\right\rangle$. Prove that 
    $$
      ||u+v|| \leq ||u||+||v||, ~~~~~ ||w||^2=\left\langle w, w\right\rangle
    $$
    for all $u,v \in V$.  [Hint: You may use the Cauchy-Schwarz inequality: $|\left\langle u, v\right\rangle | \leq ||u||.||v||$]

    % \textcolor{hwColor}{
          
    % }


  \end{enumerate}

\end{document}
