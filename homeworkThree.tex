\documentclass[fleqn]{article}
\oddsidemargin 0.0in
\textwidth 6.0in
\thispagestyle{empty}
\usepackage{import}
\usepackage{amsmath}
\usepackage{graphicx}
\usepackage{flexisym}
\usepackage{amssymb}
\usepackage{bigints} 
\usepackage[english]{babel}
\usepackage[utf8x]{inputenc}
\usepackage{float}
\usepackage[colorinlistoftodos]{todonotes}

\definecolor{hwColor}{HTML}{AD53BA}

\begin{document}

  \begin{titlepage}

    \newcommand{\HRule}{\rule{\linewidth}{0.5mm}}

    \center


    \textsc{\LARGE Arizona State University}\\[1.5cm]

    \textsc{\LARGE Linear Algebra }\\[1.5cm]


    \begin{figure}
      \includegraphics[width=\linewidth]{asu.png}
    \end{figure}


    \HRule \\[0.4cm]
    { \huge \bfseries Homework Three}\\[0.4cm] 
    \HRule \\[1.5cm]

    \textbf{Behnam Amiri}

    \bigbreak

    \textbf{Prof: Sergei Suslov}

    \bigbreak


    \textbf{{\large \today}\\[2cm]}

    \vfill

  \end{titlepage}

  \begin{enumerate}
    \item Let 
      $$
        A=\begin{pmatrix}
          -7 & -2 & 1
          \\
          0 & 3 & -1 
          \\
          -3 & 4 & -2
        \end{pmatrix}
      $$
      \begin{itemize}
        \item Find $A^{-1}$.

          \textcolor{hwColor}{
            $
              A^{-1}=\dfrac{1}{det(A)} adj(A) 
              \\
              \\
              \\
              adj(A)=\left(A_{ij}\right)^T=\begin{pmatrix}
                +\begin{vmatrix}
                  3 & -1
                  \\
                  4 & -2
                \end{vmatrix} & -\begin{vmatrix}
                  0 & -1 
                  \\
                  -3 & -2
                \end{vmatrix} & +\begin{vmatrix}
                  0 & 3
                  \\
                  -3 & 4
                \end{vmatrix}
                \\
                \\
                -\begin{vmatrix}
                  -2 & 1 
                  \\
                  4 & -2
                \end{vmatrix} & +\begin{vmatrix}
                  -7 & 1
                  \\
                  -3 & -2
                \end{vmatrix} & -\begin{vmatrix}
                  -7 & -2 
                  \\
                  -3 & 4
                \end{vmatrix} 
                \\
                \\
                +\begin{vmatrix}
                  -2 & 1
                  \\
                  3 & -1
                \end{vmatrix} & -\begin{vmatrix}
                  -7 & 1
                  \\
                  0 & -1
                \end{vmatrix} & +\begin{vmatrix}
                  -7 & -2 
                  \\
                  0 & 3
                \end{vmatrix}
              \end{pmatrix}^T=\begin{pmatrix}
                -2 & 3 & 9
                \\
                0 & 17 & 34
                \\
                -1 & -7 & -21
              \end{pmatrix}^T 
              \\
              \\
              \\
              \therefore ~~~~ adj(A)=\begin{pmatrix}
                -2 & 0 & -1
                \\
                3 & 17 & -7
                \\
                9 & 34 & -21
              \end{pmatrix}
              \\
              \\
              \rule{15cm}{1pt}
              \\
              \\
              det(A)=\begin{vmatrix}
                -7 & -2 & 1
                \\
                0 & 3 & -1 
                \\
                -3 & 4 & -2
              \end{vmatrix}
              =
              (-7)(-1)^{1+1} \begin{vmatrix}
                3 & -1
                \\
                4 & -2
              \end{vmatrix}
              +
              0 (-1)^{2+1} \begin{vmatrix}
                -2 & 1
                \\
                4 & -2
              \end{vmatrix}
              +
              (-3)(-1)^{3+1} \begin{vmatrix}
                -2 & 1 
                \\
                3 & -1
              \end{vmatrix}
              \\
              \\
              \\
              =(-7)(-2)+0+(-3)(-1)
              \\
              \\
              \\
              \therefore ~~~~ det(A)=17
              \\
              \\
            $
            Now we can find the inverse matrix: 
            \\
            \\
            $
              A^{-1}=\dfrac{1}{det(A)} adj(A)=\dfrac{1}{17} \begin{pmatrix}
                -2 & 0 & -1
                \\
                3 & 17 & -7
                \\
                9 & 34 & -21
              \end{pmatrix}=\begin{pmatrix}
                -\dfrac{2}{17} & 0 & -\dfrac{1}{17}
                \\
                \\
                \dfrac{3}{17} & 1 & -\dfrac{7}{17}
                \\
                \\
                \dfrac{9}{17} & 2 & -\dfrac{21}{17}
              \end{pmatrix}
              \\
              \\
              \\
              \therefore ~~~~ A^{-1}=\begin{pmatrix}
                -\dfrac{2}{17} & 0 & -\dfrac{1}{17}
                \\
                \\
                \dfrac{3}{17} & 1 & -\dfrac{7}{17}
                \\
                \\
                \dfrac{9}{17} & 2 & -\dfrac{21}{17}
              \end{pmatrix}
            $
            \\
            \\
            Check point:
            \\
            \\
            $
              A.A^{-1}=\begin{pmatrix}
                -7 & -2 & 1
                \\
                0 & 3 & -1 
                \\
                -3 & 4 & -2
              \end{pmatrix}.\begin{pmatrix}
                -\dfrac{2}{17} & 0 & -\dfrac{1}{17}
                \\
                \\
                \dfrac{3}{17} & 1 & -\dfrac{7}{17}
                \\
                \\
                \dfrac{9}{17} & 2 & -\dfrac{21}{17}
              \end{pmatrix}=\begin{pmatrix}
                1 & 0 & 0
                \\
                0 & 1 & 0
                \\
                0 & 0 & 1
              \end{pmatrix}
              \\
              \\
              \\
              \therefore ~~~~ A.A^{-1}=I_{3 \times 3}
            $
          }

        \item Solve the system $Ax=\left(1, -1, 1\right)^T$.

          \textcolor{hwColor}{
            $
              Ax=\left(1, -1, 1\right)^T=\begin{pmatrix}
                1
                \\
                -1
                \\
                1
              \end{pmatrix}
              \\
              \\
              \\
              \therefore ~~~~ A^{-1}.A.x=A^{-1}\begin{pmatrix}
                1
                \\
                -1
                \\
                1
              \end{pmatrix}
              \\
              \\
              \\
              \therefore ~~~~ I.x=x=A^{-1}\begin{pmatrix}
                1
                \\
                -1
                \\
                1
              \end{pmatrix}=\begin{pmatrix}
                -\dfrac{2}{17} & 0 & -\dfrac{1}{17}
                \\
                \\
                \dfrac{3}{17} & 1 & -\dfrac{7}{17}
                \\
                \\
                \dfrac{9}{17} & 2 & -\dfrac{21}{17}
              \end{pmatrix}.\begin{pmatrix}
                1
                \\
                -1
                \\
                1
              \end{pmatrix}
              \\
              \\
              \\
              =\begin{pmatrix}
                \left(-\dfrac{2}{17}\right)(1)+0(-1)+\left(-\dfrac{1}{17}\right)(1)
                \\
                \\
                \dfrac{3}{17}(1)+(1)(-1)+\left(-\dfrac{7}{17}\right)(1)
                \\
                \\
                \dfrac{9}{17} (1)+2(-1)+\left(-\dfrac{21}{17}\right)(1)
              \end{pmatrix}
              \\
              \\
              \\
              \\
              \therefore ~~~~ x=\begin{pmatrix}
                -\dfrac{3}{17}
                \\
                \\
                -\dfrac{21}{17}
                \\
                \\
                -\dfrac{46}{17}
              \end{pmatrix}
            $
          }
        
      \end{itemize}

    \item Evaluate the following determinant: 
      $$
        det\begin{pmatrix}
          1 & 2 & 3
          \\
          1 & -1 & 1
          \\
          3 & 2 & 1
        \end{pmatrix}
      $$

        \textcolor{hwColor}{
          $
            det\begin{pmatrix}
              1 & 2 & 3
              \\
              1 & -1 & 1
              \\
              3 & 2 & 1
            \end{pmatrix}=1(-1)^{1+1} \begin{vmatrix}
              -1 & 1 
              \\
              2 & 1
            \end{vmatrix}
            +2(-1)^{1+2} \begin{vmatrix}
              1 & 1 
              \\
              3 & 1
            \end{vmatrix}
            +3 (-1)^{1+3} \begin{vmatrix}
              1 & -1 
              \\
              3 & 2
            \end{vmatrix}
            \\
            \\
            \\
            =\left(-1-2\right)+(-2)\left(1-3\right)+3\left(2+3\right)=-3+4+15
            \\
            \\
            \\
            \therefore ~~~~ det\begin{pmatrix}
              1 & 2 & 3
              \\
              1 & -1 & 1
              \\
              3 & 2 & 1
            \end{pmatrix}=16
          $
        }

    \item Find the eigenvalues and associated eigenvectors of a given matrix $A$.
    
      \begin{itemize}
        \item $A=\begin{pmatrix}
          2 & 1 
          \\
          6 & -3
        \end{pmatrix}$
        \item $A=\begin{pmatrix}
          -2 & 0 & 0 
          \\
          3 & 7 & 2 
          \\
          1 & -2 & 2
        \end{pmatrix}$
      \end{itemize}

        \textcolor{hwColor}{
          To solve for the eigenvalues, $\lambda_i$, and the corresponding eigenvectors, $x$ of an $n \times n$ matrix 
          $A$, we should do the following: 
          \\
          \begin{itemize}
            \item Multiply an $n \times n$ identity matrix by the scalar $\lambda$.
            \item Subtract the identity matrix multiple from the matrix $A$.
            \item Find the determinant of the matrix and the difference.
            \item Solve for the values of $\lambda$ that satisfy the equation $det\left(A-\lambda I\right)=0$.
            \item Solve for the corresponding vector to each $\lambda$.
          \end{itemize}
        }

        \pagebreak

        \textcolor{hwColor}{
          $
            A=\begin{pmatrix}
              2 & 1 
              \\
              6 & -3
            \end{pmatrix}
            \\
            \\
            \\
            \lambda I=\lambda \begin{pmatrix}
              1 & 0 
              \\
              0 & 1
            \end{pmatrix}=\begin{pmatrix}
              \lambda & 0
              \\
              0 & \lambda
            \end{pmatrix}
            \\
            \\
            \\
            A-\lambda I=\begin{pmatrix}
              2 & 1 
              \\
              6 & -3
            \end{pmatrix}-\begin{pmatrix}
              \lambda & 0
              \\
              0 & \lambda
            \end{pmatrix}=\begin{pmatrix}
              2-\lambda & 1
              \\
              6 & -3-\lambda
            \end{pmatrix}
            \\
            \\
            \\
            det\left(A-\lambda I\right)=\begin{vmatrix}
              2-\lambda & 1
              \\
              6 & -3-\lambda
            \end{vmatrix}=\left(2-\lambda\right) \left(-3-\lambda\right)-6=\lambda^2+\lambda-12
            \\
            \\
            \\
            det\left(A-\lambda I\right)=\lambda^2+\lambda-12=0 \rightarrow \begin{cases}
              \lambda=-4 
              \\
              \lambda=3
            \end{cases}
          $
          \\
          \\
          Now let's find the corresponding eigenvectors: 
          \\
          \\
          $
            \lambda=-4 \Rightarrow A-\lambda I=\begin{pmatrix}
              6 & 1
              \\
              6 & 1
            \end{pmatrix}
            \\
            \\
            \\
            \left(A-\lambda I\right) X=0 \Rightarrow \begin{pmatrix}
              6 & 1
              \\
              6 & 1
            \end{pmatrix} \begin{pmatrix}
              x_1
              \\
              x_2
            \end{pmatrix}=\begin{pmatrix}
              0
              \\
              0
            \end{pmatrix} \Rightarrow \left(\begin{array}{cc|c}  
              6 & 1 & 0
              \\  
              6 & 1 & 0
            \end{array}\right)
            \\
            \\
            \rule{15cm}{1pt}
            \\
            \\
            \dfrac{1}{6}R_1 \rightarrow R_1
            \\
            \\
            \left(\begin{array}{cc|c}  
              1 & \dfrac{1}{6} & 0
              \\  
              6 & 1 & 0
            \end{array}\right)
            \\
            \\
            \rule{15cm}{1pt}
            \\
            \\
            R_2-6R_1 \rightarrow R_2
            \\
            \\
            \left(\begin{array}{cc|c}  
              1 & \dfrac{1}{6} & 0
              \\  
              0 & 0 & 0
            \end{array}\right)
            \\
            \\
            \rule{15cm}{1pt}
            \\
            \\
            \Longrightarrow x_1+\dfrac{1}{6}x_2=0 \Longrightarrow X=\begin{pmatrix}
              -\dfrac{1}{6} x_2
              \\
              x_2
            \end{pmatrix}
          $
          \\
          \\
          Now let's set $x_2=1$ then we have the following eigenvector:
          \\
          \\
          $
            X=\begin{pmatrix}
              -\dfrac{1}{6}
              \\
              1
            \end{pmatrix}
          $
          \\
          \\
          \rule{15cm}{3pt}
          \\
          \\
          $
            \lambda=3 \Rightarrow A-\lambda I=\begin{pmatrix}
              -1 & 1
              \\
              6 & -6
            \end{pmatrix} 
            \\
            \\
            \left(A-\lambda I\right) X=0 \Rightarrow \begin{pmatrix}
              -1 & 1
              \\
              6 & -6
            \end{pmatrix} \begin{pmatrix}
              x_1
              \\
              x_2
            \end{pmatrix}=\begin{pmatrix}
              0
              \\
              0
            \end{pmatrix} \Rightarrow \left(\begin{array}{cc|c}  
              -1 & 1 & 0
              \\  
              6 & -6 & 0
            \end{array}\right)
            \\
            \\
            \rule{15cm}{1pt}
            \\
            \\
            -R_1 \rightarrow R_1
            \\
            \\
            \left(\begin{array}{cc|c}  
              1 & -1 & 0
              \\  
              6 & -6 & 0
            \end{array}\right)
            \\
            \\
            \rule{15cm}{1pt}
            \\
            \\
            R_2-6R_1 \rightarrow R_2
            \\
            \\
            \left(\begin{array}{cc|c}  
              1 & -1 & 0
              \\  
              0 & 0 & 0
            \end{array}\right)
            \\
            \\
            \rule{15cm}{1pt}
            \\
            \\
            \Longrightarrow x_1-x_2=0 \Longrightarrow X=\begin{pmatrix}
              x_2
              \\
              x_2
            \end{pmatrix}
            \\
            \\
          $
          Let's set $x_2=1$ then $X=\begin{pmatrix}
            1
            \\
            1
          \end{pmatrix}$
          \\
          \\
          \\
          $
            \therefore ~~~~ \begin{cases}
              \lambda=-4 \Longrightarrow X=\begin{pmatrix}
                -\dfrac{1}{6} 
                \\
                \\
                1
              \end{pmatrix}
              \\
              \\
              \lambda=3 \longrightarrow X=\begin{pmatrix}
                1
                \\
                1
              \end{pmatrix}
            \end{cases}
          $
        }

        \pagebreak

        \textcolor{hwColor}{
          $
            A=\begin{pmatrix}
              -2 & 0 & 0 
              \\
              3 & 7 & 2 
              \\
              1 & -2 & 2
            \end{pmatrix}
            \\
            \\
            \\
            \lambda I=\lambda \begin{pmatrix}
              1 & 0 & 0
              \\
              0 & 1 & 0
              \\
              0 & 0 & 1
            \end{pmatrix}=\begin{pmatrix}
              \lambda & 0 & 0
              \\
              0 & \lambda & 0
              \\
              0 & 0 & \lambda
            \end{pmatrix}
            \\
            \\
            \\
            A-\lambda I=\begin{pmatrix}
              -2 & 0 & 0 
              \\
              3 & 7 & 2 
              \\
              1 & -2 & 2
            \end{pmatrix}-\begin{pmatrix}
              \lambda & 0 & 0
              \\
              0 & \lambda & 0
              \\
              0 & 0 & \lambda
            \end{pmatrix}=\begin{pmatrix}
              -2-\lambda & 0 & 0
              \\
              3 & 7-\lambda & 2
              \\
              1 & -2 & 2-\lambda
            \end{pmatrix}
            \\
            \\
            \\
            \\
            det\left(A-\lambda I\right)=\begin{vmatrix}
              -2-\lambda & 0 & 0
              \\
              3 & 7-\lambda & 2
              \\
              1 & -2 & 2-\lambda
            \end{vmatrix}=\left(-2-\lambda\right) (-1)^{1+1} \begin{vmatrix}
              7-\lambda & 2
              \\
              -2 & 2-\lambda
            \end{vmatrix}
            \\
            \\
            \\
            =\left(-2-\lambda\right) \left(7-\lambda\right) \left(2-\lambda\right)+4
            =-\lambda^3+7\lambda^2-36=\left(-\lambda-2\right) \left(\lambda-3\right) \left(\lambda-6\right)
            \\
            \\
            \\
            det\left(A-\lambda I\right)=\left(-\lambda-2\right) \left(\lambda-3\right) \left(\lambda-6\right)=0
            \\
            \\
            \begin{cases}
              \lambda=-2
              \\
              \lambda=3
              \\
              \lambda=6
            \end{cases}
          $
          \\
          \\
          Now we need to find the corresponding eigenvectors for each of the above eigenvalues.
          \\
          \\
          \rule{15cm}{1pt}
          \\
          \\
          $
            \lambda=-2 \Longrightarrow A-\lambda I=\begin{vmatrix}
              0 & 0 & 0
              \\
              3 & 9 & 2
              \\
              1 & -2 & 4
            \end{vmatrix}
          $
          \\
          \\
          \rule{15cm}{1pt}
          \\
          \\
          $
            \lambda=3 \Longrightarrow A-\lambda I=\begin{vmatrix}
              -5 & 0 & 0
              \\
              3 & 4 & 2
              \\
              1 & -2 & -1
            \end{vmatrix}
          $
          \\
          \\
          \rule{15cm}{1pt}
          \\
          \\
          $
            \lambda=6 \Longrightarrow A-\lambda I=\begin{vmatrix}
              -8 & 0 & 0
              \\
              3 & 1 & 2
              \\
              1 & -2 & -4
            \end{vmatrix}
          $
        }


  \end{enumerate}

\end{document}
