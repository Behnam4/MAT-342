\documentclass[fleqn]{article}
\oddsidemargin 0.0in
\textwidth 6.0in
\thispagestyle{empty}
\usepackage{import}
\usepackage{amsmath}
\usepackage{graphicx}
\usepackage{flexisym}
\usepackage{amssymb}
\usepackage{bigints} 
\usepackage[english]{babel}
\usepackage[utf8x]{inputenc}
\usepackage{float}
\usepackage[colorinlistoftodos]{todonotes}

\definecolor{hwColor}{HTML}{AD53BA}

\begin{document}

  \begin{titlepage}

    \newcommand{\HRule}{\rule{\linewidth}{0.5mm}}

    \center


    \textsc{\LARGE Arizona State University}\\[1.5cm]

    \textsc{\LARGE Linear Algebra }\\[1.5cm]


    \begin{figure}
      \includegraphics[width=\linewidth]{asu.png}
    \end{figure}


    \HRule \\[0.4cm]
    { \huge \bfseries Quiz Three}\\[0.4cm] 
    \HRule \\[1.5cm]

    \textbf{Behnam Amiri}

    \bigbreak

    \textbf{Prof: Sergei Suslov}

    \bigbreak


    \textbf{{\large \today}\\[2cm]}

    \vfill

  \end{titlepage}

  \begin{enumerate}
    \item (10 points) Are the vectors $v_1=(7, -1, 3), ~ v_2=(-3, 2, -5),$ and $v_3=(-1, 2, -4)$ linearly independent? Explain.

      \textcolor{hwColor}{
        A collection of vectors is either linearly independent or linearly dependent. 
        The vectors $v_1, v_2, v_3$ are linearly independent if the equation involving 
        linear combinations $$c_1 v_1+c_2 v_2+c_3 v_3=0$$ is true only when the scalars $c_i$
        are all equal to zero. The vectors are linearly dependent if the equation has a solution
        when at least one of the scalars is not zero.
        \\
        \\
        $
          c_1(7, -1, 3)+c_2 (-3, 2, -5)+c_3 (-1, 2, -4)=0
          \\
          \\
          \\
          \Longrightarrow \left(\begin{array}{ccc|c}  
            7 & -1 & 3 & 0
            \\  
            -3 & 2 & -5 & 0
            \\
            -1 & 2 & -4 & 0
          \end{array}\right)
          \\
          \\
        $
        Let's reduce this augmented matrix to the reduced row echelon form.
        \\
        \\
        $
          A=\left(\begin{array}{ccc|c}  
            7 & -1 & 3 & 0
            \\  
            -3 & 2 & -5 & 0
            \\
            -1 & 2 & -4 & 0
          \end{array}\right)
          \\
          \\
          \rule{15cm}{1pt}
          \\
          \\
          \dfrac{1}{7}R_1 \rightarrow R_1
          \\
          \\
          =\left(\begin{array}{ccc|c}  
            1 & -\dfrac{1}{7} & \dfrac{3}{7} & 0
            \\  
            -3 & 2 & -5 & 0
            \\
            -1 & 2 & -4 & 0
          \end{array}\right)
          \\
          \\
          \rule{15cm}{1pt}
          \\
          \\
          3\dfrac{1}{7}R_1+R_2 \rightarrow R_2
          \\
          \\
          =\left(\begin{array}{ccc|c}  
            1 & -\dfrac{1}{7} & \dfrac{3}{7} & 0
            \\  
            0 & \dfrac{11}{7} & -\dfrac{26}{7} & 0
            \\
            -1 & 2 & -4 & 0
          \end{array}\right)
          \\
          \\
          \rule{15cm}{1pt}
          \\
          \\
          R_1+R_3 \rightarrow R_3
          \\
          \\
          =\left(\begin{array}{ccc|c}  
            1 & -\dfrac{1}{7} & \dfrac{3}{7} & 0
            \\
            \\
            0 & \dfrac{11}{7} & -\dfrac{26}{7} & 0
            \\
            \\
            0 & \dfrac{13}{7} & -\dfrac{25}{7} & 0
          \end{array}\right)
          \\
          \\
          \rule{15cm}{1pt}
          \\
          \\
          \dfrac{7}{11}R_2 \rightarrow R_2
          \\
          \\
          =\left(\begin{array}{ccc|c}  
            1 & -\dfrac{1}{7} & \dfrac{3}{7} & 0
            \\
            \\
            0 & 1 & -\dfrac{26}{11} & 0
            \\
            \\
            0 & \dfrac{13}{7} & -\dfrac{25}{7} & 0
          \end{array}\right)
          \\
          \\
          \rule{15cm}{1pt}
          \\
          \\
          -\dfrac{13}{7}R_2 \rightarrow R_3
          \\
          \\
          =\left(\begin{array}{ccc|c}  
            1 & -\dfrac{1}{7} & \dfrac{3}{7} & 0
            \\
            \\
            0 & 1 & -\dfrac{26}{11} & 0
            \\
            \\
            0 & 0 & \dfrac{9}{11} & 0
          \end{array}\right)
          \\
          \\
          \rule{15cm}{1pt}
          \\
          \\
          \dfrac{11}{9}R_3 \rightarrow R_3
          \\
          \\
          =\left(\begin{array}{ccc|c}  
            1 & -\dfrac{1}{7} & \dfrac{3}{7} & 0
            \\
            \\
            0 & 1 & -\dfrac{26}{11} & 0
            \\
            0 & 0 & 1 & 0
          \end{array}\right)
          \\
          \\
          \rule{15cm}{1pt}
          \\
          \\
          \dfrac{26}{11}R_3+R_2 \rightarrow R_2
          \\
          \\
          =\left(\begin{array}{ccc|c}  
            1 & -\dfrac{1}{7} & \dfrac{3}{7} & 0
            \\
            0 & 1 & 0 & 0
            \\
            0 & 0 & 1 & 0
          \end{array}\right)
          \\
          \\
          \rule{15cm}{1pt}
          \\
          \\
          -\dfrac{3}{7}R_3+R_1 \rightarrow R_1
          \\
          \\
          =\left(\begin{array}{ccc|c}  
            1 & -\dfrac{1}{7} & 0 & 0
            \\
            0 & 1 & 0 & 0
            \\
            0 & 0 & 1 & 0
          \end{array}\right)
          \\
          \\
          \rule{15cm}{1pt}
          \\
          \\
          \dfrac{1}{7}R_2+R_1 \rightarrow R_1
          \\
          \\
          =\left(\begin{array}{ccc|c}  
            1 & 0 & 0 & 0
            \\
            0 & 1 & 0 & 0
            \\
            0 & 0 & 1 & 0
          \end{array}\right)
        $
        \\
        \\
        The reduced row echelon form of the coefficient matrix of the homogeneous system meaning:
        $
          \begin{cases}
            c_1=0
            \\
            c_2=0
            \\
            c_3=0
          \end{cases}
        $
        \\
        \\
        This system has only the trivial solution, so that the only solution is $c_1=c_2=c_3=0$. \\ \\ \\
        $\therefore ~~~~$ The vectors $v_1, v_2$ and $v_3$ are linearly independent. $\blacksquare$ 
      }

    \item (10 points) Find the matrix representation of the following linear transformation $L:\mathbb{R}^3 \rightarrow \mathbb{R}^2$.
    $$L(x)=\left(x_1-x_2, x_2+x_3\right)^T$$
    Determine the kernel, $L(x)=0$ of this transformation.


    \item (15 points) Show that the set of powers $1, x, x^2, x^3$ form a basis in the subspace of all polynomials of degree $\leq 3$. 
    [Hint: You may use the Vandermonde determinant.]


    \item (15 points) Find both a basis for the row space and a basis for the column space of the matrix
    $$
      \begin{pmatrix}
        2 & -4 & 2 & -7
        \\
        -2 & 1 & 1 & 0
        \\
        -1 & 2 & -1 & 4
      \end{pmatrix}
    $$
    What is the rank of this matrix? Find the nullspace basis of this m
  \end{enumerate}

\end{document}
