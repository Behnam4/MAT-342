\documentclass[fleqn]{article}
\oddsidemargin 0.0in
\textwidth 6.0in
\thispagestyle{empty}
\usepackage{import}
\usepackage{amsmath}
\usepackage{graphicx}
\usepackage{flexisym}
\usepackage{amssymb}
\usepackage{bigints} 
\usepackage[english]{babel}
\usepackage[utf8x]{inputenc}
\usepackage{float}
\usepackage[colorinlistoftodos]{todonotes}

\definecolor{hwColor}{HTML}{AD53BA}

\begin{document}

  \begin{titlepage}

    \newcommand{\HRule}{\rule{\linewidth}{0.5mm}}

    \center


    \textsc{\LARGE Arizona State University}\\[1.5cm]

    \textsc{\LARGE Linear Algebra }\\[1.5cm]


    \begin{figure}
      \includegraphics[width=\linewidth]{asu.png}
    \end{figure}


    \HRule \\[0.4cm]
    { \huge \bfseries Quiz Three}\\[0.4cm] 
    \HRule \\[1.5cm]

    \textbf{Behnam Amiri}

    \bigbreak

    \textbf{Prof: Sergei Suslov}

    \bigbreak


    \textbf{{\large \today}\\[2cm]}

    \vfill

  \end{titlepage}

  \begin{enumerate}
    \item (10 points) Are the vectors $v_1=(7, -1, 3), ~ v_2=(-3, 2, -5),$ and $v_3=(-1, 2, -4)$ linearly independent? Explain.

      \textcolor{hwColor}{
        A collection of vectors is either linearly independent or linearly dependent. 
        The vectors $v_1, v_2, v_3$ are linearly independent if the equation involving 
        linear combinations $$c_1 v_1+c_2 v_2+c_3 v_3=0$$ is true only when the scalars $c_i$
        are all equal to zero. The vectors are linearly dependent if the equation has a solution
        when at least one of the scalars is not zero.
        \\
        \\
        $
          c_1(7, -1, 3)+c_2 (-3, 2, -5)+c_3 (-1, 2, -4)=0
          \\
          \\
          \\
          \Longrightarrow \left(\begin{array}{ccc|c}  
            7 & -1 & 3 & 0
            \\  
            -3 & 2 & -5 & 0
            \\
            -1 & 2 & -4 & 0
          \end{array}\right)
          \\
          \\
        $
        Let's reduce this augmented matrix to the reduced row echelon form.
        \\
        \\
        $
          A=\left(\begin{array}{ccc|c}  
            7 & -1 & 3 & 0
            \\  
            -3 & 2 & -5 & 0
            \\
            -1 & 2 & -4 & 0
          \end{array}\right)
          \\
          \\
          \rule{15cm}{1pt}
          \\
          \\
          \dfrac{1}{7}R_1 \rightarrow R_1
          \\
          \\
          =\left(\begin{array}{ccc|c}  
            1 & -\dfrac{1}{7} & \dfrac{3}{7} & 0
            \\  
            -3 & 2 & -5 & 0
            \\
            -1 & 2 & -4 & 0
          \end{array}\right)
          \\
          \\
          \rule{15cm}{1pt}
          \\
          \\
          3\dfrac{1}{7}R_1+R_2 \rightarrow R_2
          \\
          \\
          =\left(\begin{array}{ccc|c}  
            1 & -\dfrac{1}{7} & \dfrac{3}{7} & 0
            \\  
            0 & \dfrac{11}{7} & -\dfrac{26}{7} & 0
            \\
            -1 & 2 & -4 & 0
          \end{array}\right)
          \\
          \\
          \rule{15cm}{1pt}
          \\
          \\
          R_1+R_3 \rightarrow R_3
          \\
          \\
          =\left(\begin{array}{ccc|c}  
            1 & -\dfrac{1}{7} & \dfrac{3}{7} & 0
            \\
            \\
            0 & \dfrac{11}{7} & -\dfrac{26}{7} & 0
            \\
            \\
            0 & \dfrac{13}{7} & -\dfrac{25}{7} & 0
          \end{array}\right)
          \\
          \\
          \rule{15cm}{1pt}
          \\
          \\
          \dfrac{7}{11}R_2 \rightarrow R_2
          \\
          \\
          =\left(\begin{array}{ccc|c}  
            1 & -\dfrac{1}{7} & \dfrac{3}{7} & 0
            \\
            \\
            0 & 1 & -\dfrac{26}{11} & 0
            \\
            \\
            0 & \dfrac{13}{7} & -\dfrac{25}{7} & 0
          \end{array}\right)
          \\
          \\
          \rule{15cm}{1pt}
          \\
          \\
          -\dfrac{13}{7}R_2 \rightarrow R_3
          \\
          \\
          =\left(\begin{array}{ccc|c}  
            1 & -\dfrac{1}{7} & \dfrac{3}{7} & 0
            \\
            \\
            0 & 1 & -\dfrac{26}{11} & 0
            \\
            \\
            0 & 0 & \dfrac{9}{11} & 0
          \end{array}\right)
          \\
          \\
          \rule{15cm}{1pt}
          \\
          \\
          \dfrac{11}{9}R_3 \rightarrow R_3
          \\
          \\
          =\left(\begin{array}{ccc|c}  
            1 & -\dfrac{1}{7} & \dfrac{3}{7} & 0
            \\
            \\
            0 & 1 & -\dfrac{26}{11} & 0
            \\
            0 & 0 & 1 & 0
          \end{array}\right)
          \\
          \\
          \rule{15cm}{1pt}
          \\
          \\
          \dfrac{26}{11}R_3+R_2 \rightarrow R_2
          \\
          \\
          =\left(\begin{array}{ccc|c}  
            1 & -\dfrac{1}{7} & \dfrac{3}{7} & 0
            \\
            0 & 1 & 0 & 0
            \\
            0 & 0 & 1 & 0
          \end{array}\right)
          \\
          \\
          \rule{15cm}{1pt}
          \\
          \\
          -\dfrac{3}{7}R_3+R_1 \rightarrow R_1
          \\
          \\
          =\left(\begin{array}{ccc|c}  
            1 & -\dfrac{1}{7} & 0 & 0
            \\
            0 & 1 & 0 & 0
            \\
            0 & 0 & 1 & 0
          \end{array}\right)
          \\
          \\
          \rule{15cm}{1pt}
          \\
          \\
          \dfrac{1}{7}R_2+R_1 \rightarrow R_1
          \\
          \\
          =\left(\begin{array}{ccc|c}  
            1 & 0 & 0 & 0
            \\
            0 & 1 & 0 & 0
            \\
            0 & 0 & 1 & 0
          \end{array}\right)
        $
        \\
        \\
        The reduced row echelon form of the coefficient matrix of the homogeneous system meaning:
        $
          \begin{cases}
            c_1=0
            \\
            c_2=0
            \\
            c_3=0
          \end{cases}
        $
        \\
        \\
        This system has only the trivial solution, so that the only solution is $c_1=c_2=c_3=0$. \\ \\ \\
        $\therefore ~~~~$ The vectors $v_1, v_2$ and $v_3$ are linearly independent. $\blacksquare$ 
      }

    \item (10 points) Find the matrix representation of the following linear transformation $L:\mathbb{R}^3 \rightarrow \mathbb{R}^2$.
    $$L(x)=\left(x_1-x_2, x_2+x_3\right)^T$$
    Determine the kernel, $L(x)=0$ of this transformation.

      \textcolor{hwColor}{
        \\
        \\
        $L:\mathbb{R}^3 \rightarrow \mathbb{R}^2$ is a linear transformation defined by $L(x)=\left(x_1-x_2, x_2+x_3\right)^T$.
        \\
        \\
        We know the basis vectors for $\mathbb{R}^3$ are $e_1=(1, 0, 0), e_2=(0, 1, 0)$ and $e_3=(0, 0, 1)$.
        Based on the given transformation we have:
        \\
        \\
        $
        L\left(\begin{bmatrix}
          x_1 
          \\
          x_2
          \\
          x_3
        \end{bmatrix}\right)=L\left(\begin{bmatrix}
            x_1-x_2 
            \\
            x_2+x_3
          \end{bmatrix}\right)
          \\
          \\
          \Longrightarrow 
          \begin{cases}
            L\left[(e_1)^T\right]=T\left(\begin{bmatrix}
              1 
              \\
              0
              \\
              0
            \end{bmatrix}\right)=\begin{bmatrix}
              1-0
              \\
              0+0
            \end{bmatrix}=\begin{bmatrix}
              1
              \\
              0
            \end{bmatrix}=1\begin{bmatrix}
              1
              \\
              0
            \end{bmatrix}+0\begin{bmatrix}
              0
              \\
              1
            \end{bmatrix}
            \\
            \\
            L\left[(e_2)^T\right]=T\left(\begin{bmatrix}
              0 
              \\
              1
              \\
              0
            \end{bmatrix}\right)=\begin{bmatrix}
              0-1
              \\
              1+0
            \end{bmatrix}=\begin{bmatrix}
              -1
              \\
              1
            \end{bmatrix}=-1\begin{bmatrix}
              1
              \\
              0
            \end{bmatrix}+1\begin{bmatrix}
              0
              \\
              1
            \end{bmatrix}
            \\
            \\
            L\left[(e_3)^T\right]=T\left(\begin{bmatrix}
              0 
              \\
              0
              \\
              1
            \end{bmatrix}\right)=\begin{bmatrix}
              0-0
              \\
              0+1
            \end{bmatrix}=\begin{bmatrix}
              0
              \\
              1
            \end{bmatrix}=0\begin{bmatrix}
              1
              \\
              0
            \end{bmatrix}+1\begin{bmatrix}
              0
              \\
              1
            \end{bmatrix}
          \end{cases}
        $
        \\
        \\
        \\
        Therefore, the matrix for L relative to the standard bases is
        $
          \begin{pmatrix}
            1 & -1 & 0
            \\
            0 & 1 & 1
          \end{pmatrix}
        $
        \\
        \\
        \\
        Recall that the kernel of a transformation is the subset of the domain which maps into the zero vector.
        Depending on what we pick for $x_3$ we have the following:
        \\
        \\
        $
          L(x)=\begin{bmatrix}
            x_1-x_2
            \\
            x_2+x_3
          \end{bmatrix}=\begin{bmatrix}
            0
            \\
            0
          \end{bmatrix}
          \Longrightarrow \begin{cases}
            x_1=x_2
            \\
            x_2=-x_3
          \end{cases}
          \\
          \\
        $
        If $x_3=1$ then we have: 
        \\
        \\
        $
          ker(L)=\left(1, 1, -1\right) ~~~~~ \blacksquare
        $
      }

    \item (15 points) Show that the set of powers $1, x, x^2, x^3$ form a basis in the subspace of all polynomials of degree $\leq 3$. 
    [Hint: You may use the Vandermonde determinant.]


    \item (15 points) Find both a basis for the row space and a basis for the column space of the matrix
    $$
      \begin{pmatrix}
        2 & -4 & 2 & -7
        \\
        -2 & 1 & 1 & 0
        \\
        -1 & 2 & -1 & 4
      \end{pmatrix}
    $$
    What is the rank of this matrix? Find the nullspace basis of this matrix.

      \textcolor{hwColor}{
        We start by putting the matrix in row echelon form.
        \\
        \\
        $
          A=\begin{pmatrix}
            2 & -4 & 2 & -7
            \\
            -2 & 1 & 1 & 0
            \\
            -1 & 2 & -1 & 4
          \end{pmatrix}
          \\
          \\
          \rule{15cm}{1pt}
          \\
          \\
          \dfrac{1}{2}R_1 \rightarrow R_1
          \\
          \\
          \begin{pmatrix}
            1 & -2 & 1 & -\dfrac{7}{2}
            \\
            -2 & 1 & 1 & 0
            \\
            -1 & 2 & -1 & 4
          \end{pmatrix}
          \\
          \\
          \rule{15cm}{1pt}
          \\
          \\
          2R_1+R_2 \rightarrow R_2
          \\
          \\
          \begin{pmatrix}
            1 & -2 & 1 & -\dfrac{7}{2}
            \\
            0 & -3 & 3 & -7
            \\
            -1 & 2 & -1 & 4
          \end{pmatrix}
          \\
          \\
          \rule{15cm}{1pt}
          \\
          \\
          R_1+R_3 \rightarrow R_3
          \\
          \\
          \begin{pmatrix}
            1 & -2 & 1 & -\dfrac{7}{2}
            \\
            0 & -3 & 3 & -7
            \\
            0 & 0 & 0 & \dfrac{1}{2}
          \end{pmatrix}
          \\
          \\
          \rule{15cm}{1pt}
          \\
          \\
          -\dfrac{1}{3}R_2 \rightarrow R_2
          \\
          \\
          \begin{pmatrix}
            1 & -2 & 1 & -\dfrac{7}{2}
            \\
            0 & 1 & -1 & \dfrac{7}{3}
            \\
            0 & 0 & 0 & \dfrac{1}{2}
          \end{pmatrix}
          \\
          \\
          \rule{15cm}{1pt}
          \\
          \\
          2R_3 \rightarrow R_3
          \\
          \\
          \begin{pmatrix}
            1 & -2 & 1 & -\dfrac{7}{2}
            \\
            0 & 1 & -1 & \dfrac{7}{3}
            \\
            0 & 0 & 0 & 1
          \end{pmatrix}
          \\
          \\
          \rule{15cm}{1pt}
          \\
          \\
          -\dfrac{7}{3}R_3+R_2 \rightarrow R_2
          \\
          \\
          \begin{pmatrix}
            1 & -2 & 1 & -\dfrac{7}{2}
            \\
            0 & 1 & -1 & 0
            \\
            0 & 0 & 0 & 1
          \end{pmatrix}
          \\
          \rule{15cm}{1pt}
          \\
          \dfrac{7}{2}R_3+R_1 \rightarrow R_1
          \\
          \begin{pmatrix}
            1 & -2 & 1 & 0
            \\
            0 & 1 & -1 & 0
            \\
            0 & 0 & 0 & 1
          \end{pmatrix}
          \\
          \rule{15cm}{1pt}
          \\
          2R_2+R_1 \rightarrow R_1
          \\
          \\
          \begin{pmatrix}
            1 & 0 & -1 & 0
            \\
            0 & 1 & -1 & 0
            \\
            0 & 0 & 0 & 1
          \end{pmatrix}
        $
        \\
        \\
        \textbf{Basis of Row Space:} \\
        Row operations do not change the row space, so the rows of the matrix at the end have
        the same span as our given matrix. Furthermore, the nonzero rows of a matrix in row echelon
        form are linearly independent. Rows that contain the leading are first, second and third rows:
        \\
        \\
        $
          \{ 
            \begin{bmatrix}
              2
              \\
              -4
              \\
              2
              \\
              -7
            \end{bmatrix},
            \begin{bmatrix}
              -2
              \\
              1
              \\
              1 
              \\
              0
            \end{bmatrix},
            \begin{bmatrix}
              -1
              \\
              2
              \\
              -1
              \\
              4
            \end{bmatrix}
          \} 
        $
        \\
        \\
        \\
        \\
        \textbf{Basis of Column Space:} \\
        Recall that a leading one is the first nonzero entry in a row. The columns containing leading ones 
        are the pivot columns. To obtain a basis for the column space, we just use the pivot columns from 
        the original matrix:
        \\
        \\
        $
          \{ 
            \begin{bmatrix}
              2
              \\
              -2
              \\
              -1
            \end{bmatrix},
            \begin{bmatrix}
              -4
              \\
              1
              \\
              2
            \end{bmatrix},
            \begin{bmatrix}
              -7
              \\
              0
              \\
              4
            \end{bmatrix}
          \} 
        $
        \\
        \\
        \\
        Finally, the row space and column space each have bases with three vectors, so they have
        dimension three. Therefore, the rank of A is 3.
        \\
        \\
        \\
        We already found the reduced row echelon form of the of given matrix. Now, we can write the
        following system of euqations:
        \\
        \\
        $
            \begin{cases}
              x_1-x_3=0
              \\
              x_2-x_3=0
              \\
              x_4=0
            \end{cases}
        $
        \\
        \\
        This system has infinity solutions. One solution can be written as the following:
        \\
        \\
        $
          \begin{cases}
            x_1=x_3
            \\
            x_2=x_3
            \\
            x_3=arbitrary
            \\
            x_4=0
          \end{cases}
        $
        \\
        \\
        Therefore the null space has a basis formed by the set 
        $
          \{
            \begin{bmatrix}
              1
              \\
              1
              \\
              1
              \\
              0
            \end{bmatrix}
          \} 
        $.
      }

  \end{enumerate}

\end{document}
