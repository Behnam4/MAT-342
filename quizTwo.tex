\documentclass[fleqn]{article}
\oddsidemargin 0.0in
\textwidth 6.0in
\thispagestyle{empty}
\usepackage{import}
\usepackage{amsmath}
\usepackage{graphicx}
\usepackage{flexisym}
\usepackage{amssymb}
\usepackage{bigints} 
\usepackage[english]{babel}
\usepackage[utf8x]{inputenc}
\usepackage{float}
\usepackage[colorinlistoftodos]{todonotes}

\definecolor{hwColor}{HTML}{AD53BA}

\begin{document}

  \begin{titlepage}

    \newcommand{\HRule}{\rule{\linewidth}{0.5mm}}

    \center


    \textsc{\LARGE Arizona State University}\\[1.5cm]

    \textsc{\LARGE Linear Algebra }\\[1.5cm]


    \begin{figure}
      \includegraphics[width=\linewidth]{asu.png}
    \end{figure}


    \HRule \\[0.4cm]
    { \huge \bfseries Quiz 2}\\[0.4cm] 
    \HRule \\[1.5cm]

    \textbf{Behnam Amiri}

    \bigbreak

    \textbf{Prof: Sergei Suslov}

    \bigbreak


    \textbf{{\large \today}\\[2cm]}

    \vfill

  \end{titlepage}

  \begin{enumerate}
    \item (10 points) Let $A$ and $B$ be $n \times n$ similar matrices, namely, $B=S^{-1} AS$. Show that the matrices $A$ and $B$ have 
    the same characteristic polynomial and,  consequently, the same eigenvalues.

      \textcolor{hwColor}{
        We are told that matrices $A$ and $B$ are similiar, therefore, there exists an invertible matrix $S$
        such that $S^{-1} AS=B$. Allow $p$ to show the characteristic polynomials of the two matrices. Hence we have:
        \\
        \\
        $
          det\left(B-t I\right)=det \left(S^{-1} AS-t I\right)=p_B(t)
          \\
          \\
          =\left[S^{-1} \left(A-t I\right)S\right]=det\left(S^{-1}\right) det\left(A-t I\right) det(S)=W
          \\
          \\
        $
        We know that $det(S^{-1})=det(S)^{-1}$, therefore:
        \\
        \\
        $
          W=det\left(A-t I\right)=p_A(t)
          \\
          \\
          \\
          \therefore ~~~~ p_B{t}=p_B(t)  ~~~~ \checkmark
        $
      }

    \item Find the eigenvalues and associated eigenvectors of a given matrix.
    \begin{itemize}
      \item (10 points) $A=\begin{pmatrix}
        2 & -1 
        \\
        -6 & -3
      \end{pmatrix}$

        \textcolor{hwColor}{
          $
            A=\begin{pmatrix}
              2 & -1 
              \\
              -6 & -3
            \end{pmatrix}
            \\
            \\
            \lambda I= \lambda \begin{pmatrix}
              1 & 0
              \\
              0 & 1
            \end{pmatrix}=\begin{pmatrix}
              \lambda & 0
              \\
              0 & \lambda
            \end{pmatrix}
            \\
            \\
            \\
            A-\lambda I=\begin{pmatrix}
              2 & -1 
              \\
              -6 & -3
            \end{pmatrix}-\begin{pmatrix}
              \lambda & 0
              \\
              0 & \lambda
            \end{pmatrix}=\begin{pmatrix}
              2-\lambda & -1
              \\
              -6 & -3-\lambda
            \end{pmatrix}
            \\
            \\
            \\
            det\left(A-\lambda I\right)=\begin{vmatrix}
              2-\lambda & -1
              \\
              -6 & -3-\lambda
            \end{vmatrix}=\left(2-\lambda\right) \left(-3-\lambda\right)- (-6)(-1)
            \\
            \\
            =\lambda^2+\lambda-12=\left(\lambda-3\right) \left(\lambda+4\right)
            \\
            \\
            \\
            det\left(A-\lambda I\right)=0 \Longrightarrow \left(\lambda-3\right) \left(\lambda+4\right)=0 
            \Longrightarrow \begin{cases}
              \lambda=-4
              \\
              \lambda=3
            \end{cases}
          $
          \\
          \\
          \\
          \rule{15cm}{3pt}
          \\
          \\
          When $\lambda=-4$ we have:
          \\
          \\
          $
            A-\lambda I=\begin{pmatrix}
              2-\lambda & -1
              \\
              -6 & -3-\lambda
            \end{pmatrix}=\begin{pmatrix}
              6 & -1
              \\
              -6 & 1
            \end{pmatrix}
            \\
            \\
            \left(A-\lambda I\right)X=0 \Longrightarrow \begin{pmatrix}
              6 & -1
              \\
              -6 & 1
            \end{pmatrix} \begin{pmatrix}
              x_1
              \\
              x_2
            \end{pmatrix}=\begin{pmatrix}
              0
              \\
              0
            \end{pmatrix}
            \\
            \\
            \\
            \left(\begin{array}{cc|c}  
              6 & -1 & 0
              \\  
              -6 & 1 & 0
            \end{array}\right)
            \\
            \\
            \rule{15cm}{1pt}
            \\
            \\
            \dfrac{R_1}{6} \rightarrow R_1
            \\
            \\
            \left(\begin{array}{cc|c}  
              1 & -\dfrac{1}{6} & 0
              \\  
              -6 & 1 & 0
            \end{array}\right)
            \\
            \\
            \rule{15cm}{1pt}
            \\
            \\
            R_2+6R_1 \rightarrow R_2
            \\
            \\
            \left(\begin{array}{cc|c}  
              1 & -\dfrac{1}{6} & 0
              \\  
              0 & 0 & 0
            \end{array}\right)
            \\
            \\
            \\
            \therefore ~~~~ x_1-\dfrac{1}{6}x_2=0 \longrightarrow \begin{cases}
              x_1=\dfrac{1}{6}x_2
              \\
              \\
              x_2=x_2
            \end{cases}
          $
          \\
          \\
          Now let's set $x_2=1$ then we have:
          \\
          \\
          $
            X=\begin{pmatrix}
              \dfrac{1}{6} (1)
              \\
              \\
              1
            \end{pmatrix}=\begin{pmatrix}
              \dfrac{1}{6}
              \\
              \\
              1
            \end{pmatrix}
          $
          \\
          \\
          \rule{15cm}{3pt}
          \\
          \\
          When $\lambda=3$ we have:
          \\
          \\
          $
            A-\lambda I=\begin{pmatrix}
              2-\lambda & -1
              \\
              -6 & -3-\lambda
            \end{pmatrix}=\begin{pmatrix}
              -1 & -1
              \\
              -6 & -6
            \end{pmatrix}
            \\
            \\
            \\
            \left(A-\lambda I\right)X=0 \longrightarrow \begin{pmatrix}
              -1 & -1
              \\
              -6 & -6
            \end{pmatrix} \begin{pmatrix}
              x_1
              \\
              x_2
            \end{pmatrix}=\begin{pmatrix}
              0
              \\
              0
            \end{pmatrix}
            \\
            \\
            \left(\begin{array}{cc|c}  
              -1 & -1 & 0
              \\  
              -6 & -6 & 0
            \end{array}\right)
            \\
            \\
            \rule{15cm}{1pt}
            \\
            \\
            -R_1 \rightarrow R_1
            \\
            \\
            \left(\begin{array}{cc|c}  
              1 & 1 & 0
              \\  
              -6 & -6 & 0
            \end{array}\right)
            \\
            \\
            \rule{15cm}{1pt}
            \\
            \\
            R_2+6R_1 \rightarrow R_2
            \\
            \\
            \left(\begin{array}{cc|c}  
              1 & 1 & 0
              \\  
              0 & 0 & 0
            \end{array}\right)
            \\
            \\
            \\
            \therefore ~~~~ x_1+x_2=0 \Longrightarrow X=\begin{pmatrix}
              -x_2
              \\
              x_2
            \end{pmatrix}
          $
          \\
          \\
          By setting $x_2=1$ we get:
          \\
          \\
          $
            X=\begin{pmatrix}
              -1
              \\
              1
            \end{pmatrix}
          $
          \\
          \\
          \textbf{Summary:}
          \\
          \\
          $
            \therefore ~~~~~ \begin{cases}
              \lambda=-4 \Longrightarrow X=\begin{pmatrix}
                \dfrac{1}{6}
                \\
                \\
                1
              \end{pmatrix}
              \\
              \\
              \lambda=3 \Longrightarrow X=\begin{pmatrix}
                -1
                \\
                1
              \end{pmatrix}
            \end{cases}
            \\
            \\
          $
        }
  
      \item (15 points) $A=\begin{pmatrix}
        -3 & 0 & 0
        \\
        3 & 7 & 2
        \\
        1 & -2 & 2
      \end{pmatrix}$

        \textcolor{hwColor}{
          $
            A=\begin{pmatrix}
              -3 & 0 & 0
              \\
              3 & 7 & 2
              \\
              1 & -2 & 2
            \end{pmatrix}
            \\
            \\
            \\
            \lambda I= \lambda \begin{pmatrix}
              1 & 0 & 0
              \\
              0 & 1 & 0
              \\
              0 & 0 & 1
            \end{pmatrix}=\begin{pmatrix}
              \lambda & 0 & 0
              \\
              0 & \lambda & 0
              \\
              0 & 0 & \lambda
            \end{pmatrix}
            \\
            \\
            \\
            A-\lambda I=\begin{pmatrix}
              -3 & 0 & 0
              \\
              3 & 7 & 2
              \\
              1 & -2 & 2
            \end{pmatrix}-\begin{pmatrix}
              \lambda & 0 & 0
              \\
              0 & \lambda & 0
              \\
              0 & 0 & \lambda
            \end{pmatrix}=\begin{pmatrix}
              -3-\lambda & 0 & 0
              \\
              3 & 7-\lambda  & 2
              \\
              1 & -2 & 2-\lambda
            \end{pmatrix}
            \\
            \\
            \\
            det\left(A-\lambda I\right)=\begin{vmatrix}
              -3-\lambda & 0 & 0
              \\
              3 & 7-\lambda  & 2
              \\
              1 & -2 & 2-\lambda
            \end{vmatrix}=\left(-3-\lambda\right) (-1)^{1+1}\begin{vmatrix}
              7-\lambda  & 2
              \\
              -2 & 2-\lambda
            \end{vmatrix}+0+0
            \\
            \\
            \\
            =\left(-3-\lambda\right) \left[\left(7-\lambda\right) \left(2-\lambda\right)-(-2)(2)\right]
            \\
            \\
            \\
            =\left(-3-\lambda\right) \left[14-7\lambda-2\lambda+\lambda^2+4\right]
            \\
            \\
            \\
            =\left(-3-\lambda\right) \left(18-9\lambda+\lambda^2\right)
            \\
            \\
            \\
            =\left(-3-\lambda\right) \left(\lambda-6\right) \left(\lambda-3\right)
            \\
            \\
            \\
            \therefore ~~~~ det\left(A-\lambda I\right)=\left(-3-\lambda\right) \left(\lambda-6\right) \left(\lambda-3\right)
            \\
            \\
            \\
            det\left(A-\lambda I\right)=0 \Longrightarrow \left(-3-\lambda\right) \left(\lambda-6\right) \left(\lambda-3\right)=0
            \\
            \\
            \\
            \begin{cases}
              \lambda=-3
              \\
              \lambda=3
              \\
              \lambda=6
            \end{cases}
          $
          \\
          \\
          Now we need to find the corresonding eigenvectors for the aboce eigenvalues. 
          \\
          \\
          \rule{15cm}{3pt}
          \\
          \\
          When $\lambda=-3$.
          \\
          \\
          $
            A-\lambda I=\begin{pmatrix}
              -3-\lambda & 0 & 0
              \\
              3 & 7-\lambda  & 2
              \\
              1 & -2 & 2-\lambda
            \end{pmatrix}=\begin{pmatrix}
              0 & 0 & 0
              \\
              3 & 10 & 2
              \\
              1 & -2 & 5
            \end{pmatrix}
            \\
            \\
            \\
            \left(A-\lambda I\right) X=0 \Longrightarrow \begin{pmatrix}
              0 & 0 & 0
              \\
              3 & 10 & 2
              \\
              1 & -2 & 5
            \end{pmatrix} \begin{pmatrix}
              x_1
              \\
              x_2
              \\
              x_3
            \end{pmatrix}=\begin{pmatrix}
              0
              \\
              0
              \\
              0
            \end{pmatrix}
            \\
            \\
            \\
            \left(\begin{array}{ccc|c}  
              0 & 0 & 0 & 0
              \\
              3 & 10 & 2 & 0
              \\
              1 & -2 & 5 & 0
            \end{array}\right)
            \\
            \\
            \rule{15cm}{1pt}
            \\
            \\
            R_2 \leftrightarrow R_1
            \\
            \\
            \left(\begin{array}{ccc|c}
              3 & 10 & 2 & 0
              \\ 
              0 & 0 & 0 & 0
              \\
              1 & -2 & 5 & 0
            \end{array}\right)
            \\
            \\
            \rule{15cm}{1pt}
            \\
            \\
            \dfrac{1}{3} R_1 \rightarrow R_1
            \\
            \\
            \left(\begin{array}{ccc|c}
              1 & \dfrac{10}{3} & \dfrac{2}{3} & 0
              \\ 
              0 & 0 & 0 & 0
              \\
              1 & -2 & 5 & 0
            \end{array}\right)
            \\
            \\
            \rule{15cm}{1pt}
            \\
            \\
            R_3-R_1 \rightarrow R_3
            \\
            \\
            \left(\begin{array}{ccc|c}
              1 & \dfrac{10}{3} & \dfrac{2}{3} & 0
              \\ 
              0 & 0 & 0 & 0
              \\
              0 & -\dfrac{16}{3} & \dfrac{13}{3} & 0
            \end{array}\right)
            \\
            \\
            \rule{15cm}{1pt}
            \\
            \\
            R_3 \leftrightarrow R_2
            \\
            \\
            \left(\begin{array}{ccc|c}
              1 & \dfrac{10}{3} & \dfrac{2}{3} & 0
              \\
              \\
              0 & -\dfrac{16}{3} & \dfrac{13}{3} & 0
              \\
              \\
              0 & 0 & 0 & 0
            \end{array}\right)
            \\
            \\
            \rule{15cm}{1pt}
            \\
            \\
            -\dfrac{3}{16} R_2 \rightarrow R_2
            \\
            \\
            \left(\begin{array}{ccc|c}
              1 & \dfrac{10}{3} & \dfrac{2}{3} & 0
              \\
              \\
              0 & 1 & -\dfrac{13}{16} & 0
              \\
              \\
              0 & 0 & 0 & 0
            \end{array}\right)
            \\
            \\
            \rule{15cm}{1pt}
            \\
            \\
            R_1 \dfrac{10}{3} R_2 \rightarrow R_1
            \\
            \\
            \left(\begin{array}{ccc|c}
              1 & 0 & \dfrac{27}{8} & 0
              \\
              \\
              0 & 1 & -\dfrac{13}{16} & 0
              \\
              \\
              0 & 0 & 0 & 0
            \end{array}\right)
            \\
            \\
            \\
            \therefore ~~~~ \begin{cases}
              x_1+\dfrac{27}{8}x_3=0
              \\
              \\
              x_2-\dfrac{13}{16}x_3=0
            \end{cases}
          $
          \\
          \\
          By doing a bit of algebra we have:
          \\
          \\
          $
            \begin{cases}
              x_1=-\dfrac{27}{8} x_3
              \\
              \\
              x_2=\dfrac{13}{16} x_3
              \\
              \\
              x_3=x_3
            \end{cases}
            \Longrightarrow X=\begin{pmatrix}
              -\dfrac{27}{8} x_3
              \\
              \\
              \dfrac{13}{16} x_3
              \\
              \\
              x_3
            \end{pmatrix}
          $
          \\
          \\
          Let's set $x_3=1$ hence we get
          \\
          \\
          $
          X=\begin{pmatrix}
            -\dfrac{27}{8}
            \\
            \\
            \dfrac{13}{16}
            \\
            \\
            1
          \end{pmatrix}
          $
          \\
          \\
          \rule{15cm}{3pt}
          \\
          \\
          When $\lambda=3$.
          \\
          \\
          $
            A-\lambda I=\begin{pmatrix}
              -3-\lambda & 0 & 0
              \\
              3 & 7-\lambda  & 2
              \\
              1 & -2 & 2-\lambda
            \end{pmatrix}=\begin{pmatrix}
              -6 & 0 & 0
              \\
              3 & 4  & 2
              \\
              1 & -2 & -1
            \end{pmatrix}
            \\
            \\
            \\
            \left(A-\lambda I\right)X=0 \Longrightarrow \begin{pmatrix}
              -6 & 0 & 0
              \\
              3 & 4  & 2
              \\
              1 & -2 & -1
            \end{pmatrix} \begin{pmatrix}
              x_1
              \\
              x_2
              \\
              x_3
            \end{pmatrix}=\begin{pmatrix}
              0
              \\ 
              0
              \\
              0
            \end{pmatrix}
            \\
            \\
            \\
            \left(\begin{array}{ccc|c}
              -6 & 0 & 0 & 0
              \\
              3 & 4 & 2 & 0
              \\
              1 & -2 & -1 & 0
            \end{array}\right)
            \\
            \\
            \rule{15cm}{1pt}
            \\
            \\
            -\dfrac{1}{6}R_1 \rightarrow R_1
            \\
            \\
            \left(\begin{array}{ccc|c}
              1 & 0 & 0 & 0
              \\
              3 & 4 & 2 & 0
              \\
              1 & -2 & -1 & 0
            \end{array}\right)
            \\
            \\
            \rule{15cm}{1pt}
            \\
            \\
            R_2-3R_1 \rightarrow R_2
            \\
            \\
            \left(\begin{array}{ccc|c}
              1 & 0 & 0 & 0
              \\
              0 & 4 & 2 & 0
              \\
              1 & -2 & -1 & 0
            \end{array}\right)
            \\
            \\
            \rule{15cm}{1pt}
            \\
            \\
            R_3-R_1 \rightarrow R_3
            \\
            \\
            \left(\begin{array}{ccc|c}
              1 & 0 & 0 & 0
              \\
              0 & 4 & 2 & 0
              \\
              0 & -2 & -1 & 0
            \end{array}\right)
            \\
            \\
            \rule{15cm}{1pt}
            \\
            \\
            \dfrac{1}{4}R_2 \rightarrow R_2
            \\
            \\
            \left(\begin{array}{ccc|c}
              1 & 0 & 0 & 0
              \\
              0 & 1 & 0.5 & 0
              \\
              0 & -2 & -1 & 0
            \end{array}\right)
            \\
            \\
            \rule{15cm}{1pt}
            \\
            \\
            R_3+2R_2 \rightarrow R_3
            \\
            \\
            \left(\begin{array}{ccc|c}
              1 & 0 & 0 & 0
              \\
              0 & 1 & 0.5 & 0
              \\
              0 & 0 & 0 & 0
            \end{array}\right)
            \\
            \\
            \\
            \therefore ~~~~ \begin{cases}
              x_1=0
              \\
              x_2+\dfrac{1}{2}x_3=0
            \end{cases} \Longrightarrow X=\begin{pmatrix}
              0
              \\
              \\
              -\dfrac{1}{2}x_3
              \\
              \\
              x_3
            \end{pmatrix}
          $
          \\
          \\
          Setting $x_3=1$ gives us:
          \\
          \\
          $
            X=\begin{pmatrix}
              0
              \\
              \\
              -\dfrac{1}{2}
              \\
              \\
              1
            \end{pmatrix}
          $
          \\
          \\
          \rule{15cm}{3pt}
          \\
          \\
          When $\lambda=6$.
          \\
          \\
          $
            A-\lambda I=\begin{pmatrix}
              -3-\lambda & 0 & 0
              \\
              3 & 7-\lambda  & 2
              \\
              1 & -2 & 2-\lambda
            \end{pmatrix}=\begin{pmatrix}
              -9 & 0 & 0
              \\
              3 & 1  & 2
              \\
              1 & -2 & -4
            \end{pmatrix}
            \\
            \\
            \\
            \left(A-\lambda I\right)X=0 \Longrightarrow \begin{pmatrix}
              -9 & 0 & 0
              \\
              3 & 1  & 2
              \\
              1 & -2 & -4
            \end{pmatrix} \begin{pmatrix}
              x_1
              \\
              x_2
              \\
              x_3
            \end{pmatrix}=\begin{pmatrix}
              0
              \\
              0
              \\
              0
            \end{pmatrix}
            \\
            \\
            \\
            \left(\begin{array}{ccc|c}
              -9 & 0 & 0 & 0
              \\
              3 & 1  & 2 & 0
              \\
              1 & -2 & -4 & 0
            \end{array}\right)
            \\
            \\
            \rule{15cm}{1pt}
            \\
            \\
            -\dfrac{1}{9}R_1 \rightarrow R_1
            \\
            \\
            \left(\begin{array}{ccc|c}
              1 & 0 & 0 & 0
              \\
              3 & 1  & 2 & 0
              \\
              1 & -2 & -4 & 0
            \end{array}\right)
            \\
            \\
            \rule{15cm}{1pt}
            \\
            \\
            R_2-3R_1 \rightarrow R_2
            \\
            \\
            \left(\begin{array}{ccc|c}
              1 & 0 & 0 & 0
              \\
              0 & 1  & 2 & 0
              \\
              1 & -2 & -4 & 0
            \end{array}\right)
            \\
            \\
            \rule{15cm}{1pt}
            \\
            \\
            R_3-R_1 \rightarrow R_3
            \\
            \\
            \left(\begin{array}{ccc|c}
              1 & 0 & 0 & 0
              \\
              0 & 1  & 2 & 0
              \\
              0 & -2 & -4 & 0
            \end{array}\right)
            \\
            \\
            \rule{15cm}{1pt}
            \\
            \\
            R_3+2R_2 \rightarrow R_3
            \\
            \\
            \left(\begin{array}{ccc|c}
              1 & 0 & 0 & 0
              \\
              0 & 1  & 2 & 0
              \\
              0 & 0 & 0 & 0
            \end{array}\right)
            \\
            \\
            \\
            \therefore ~~~~ \begin{cases}
              x_1=0
              \\
              x_2+2x_3=0 
            \end{cases} \Longrightarrow X=\begin{pmatrix}
              0
              \\
              \\
              -2x_3
              \\
              \\
              x_3
            \end{pmatrix}
          $
          \\
          \\
          Let's set $x_3=1$ to get the following:
          \\
          \\
          $
            X=\begin{pmatrix}
              0
              \\
              \\
              -2
              \\
              \\
              1
            \end{pmatrix}
          $
          \\
          \\
          \\
          \textbf{Summary:}
          \\
          \\
          $
            \therefore ~~~~~ \begin{cases}
              \lambda=-3 \Longrightarrow X=\begin{pmatrix}
                -\dfrac{27}{8}
                \\
                \\
                \dfrac{13}{16}
                \\
                \\
                1
              \end{pmatrix}
              \\
              \\
              \lambda=3 \Longrightarrow X=\begin{pmatrix}
                0
                \\
                \\
                -\dfrac{1}{2}
                \\
                \\
                1
              \end{pmatrix}
              \\
              \\
              \lambda=6 \Longrightarrow X=\begin{pmatrix}
                0
                \\
                \\
                -2
                \\
                \\
                1
              \end{pmatrix}
            \end{cases}
            \\
            \\
          $
        }

    \end{itemize}

    \item (15 points) Solve the following initial value problem:
    $$
      \dfrac{d}{dx} X=\begin{pmatrix}
        -3 & 0 & 0
        \\
        3 & 7 & 2 
        \\
        1 & -2 & 2
      \end{pmatrix} X, ~~~~~~~ X(0)=\begin{pmatrix}
        0 
        \\
        1
        \\
        -1
      \end{pmatrix}
    $$

      \textcolor{hwColor}{
        $
        \begin{pmatrix}
          -3 & 0 & 0
          \\
          3 & 7 & 2 
          \\
          1 & -2 & 2
        \end{pmatrix}
        $ matrix is the exact same matrix given in the previous question. The way we are going to 
        solve this problem is to calculate the eigenvalues of the matrix and the corresonding
        eigenvectors. We already did that in question 2. so we have:
        \\
        \\
        $
          \begin{cases}
            \lambda=-3 \Longrightarrow v=\begin{pmatrix}
              -\dfrac{27}{8}
              \\
              \\
              \dfrac{13}{16}
              \\
              1
            \end{pmatrix}
            \\
            \lambda=3 \Longrightarrow v=\begin{pmatrix}
              0
              \\
              -\dfrac{1}{2}
              \\
              1
            \end{pmatrix}
            \\
            \\
            \lambda=6 \Longrightarrow v=\begin{pmatrix}
              0
              \\
              -2
              \\
              1
            \end{pmatrix}
          \end{cases}
          \\
          \\
          \\
          \overrightarrow{X}=e^{\lambda t} \overrightarrow{v}, ~~~ \overrightarrow{v} ~~ constant \neq 0
          \\
          \\
          \\
        $
        General solution:
        \\
        \\
        $
          \overrightarrow{X}=C_1 e^{-3t} \begin{pmatrix}
            -\dfrac{27}{8}
            \\
            \\
            \dfrac{13}{16}
            \\
            \\
            1
          \end{pmatrix}+C_2 e^{3t} \begin{pmatrix}
            0
            \\
            \\
            -\dfrac{1}{2}
            \\
            \\
            1
          \end{pmatrix}+C_3 e^{6t}\begin{pmatrix}
            0
            \\
            -2
            \\
            1
          \end{pmatrix}
        $
        \\
        \\
        \\
        Initial-value problem:
        \\
        \\
        $
          X(0)=\begin{pmatrix}
            0 
            \\
            1 
            \\
            -1
          \end{pmatrix}=C_1 \begin{pmatrix}
            -\dfrac{27}{8}
            \\
            \\
            \dfrac{13}{16}
            \\
            \\
            1
          \end{pmatrix}+C_2 \begin{pmatrix}
            0
            \\
            \\
            -\dfrac{1}{2}
            \\
            \\
            1
          \end{pmatrix}+C_3 \begin{pmatrix}
            0
            \\
            \\
            -2
            \\
            \\
            1
          \end{pmatrix}
          \\
          \\
          \\
          =\begin{pmatrix}
            -\dfrac{27}{8}C_1
            \\
            \\
            \dfrac{13}{16}C_1
            \\
            \\
            C_1
          \end{pmatrix}+\begin{pmatrix}
            0
            \\
            \\
            -\dfrac{1}{2}C_2
            \\
            \\
            C_2
          \end{pmatrix}+\begin{pmatrix}
            0
            \\
            \\
            -2C_3
            \\
            \\
            C_3
          \end{pmatrix}=\begin{pmatrix}
            0 
            \\
            \\
            1 
            \\
            \\
            -1
          \end{pmatrix}
          \\
          \\
          \\
          \therefore ~~~~ \begin{cases}
            -\dfrac{27}{8}C_1+0 C_2+0 C_3=0
            \\
            \\
            \dfrac{13}{16}C_1-\dfrac{1}{2}C_2-2C_3=1
            \\
            \\
            C_1+C_2+C_3=-1
          \end{cases}
          \\
          \\
          \\
          \Longrightarrow \begin{pmatrix}
            -\dfrac{27}{8} & 0 & 0
            \\
            \\
            \dfrac{13}{16} & -\dfrac{1}{2} & -2
            \\
            \\
            1 & 1 & 1
          \end{pmatrix} \begin{pmatrix}
            C_1
            \\
            \\
            C_2
            \\
            \\
            C_3
          \end{pmatrix}=\begin{pmatrix}
            0
            \\
            \\
            1
            \\
            \\
            -1
          \end{pmatrix}
          \\
          \\
          \\
          \Longrightarrow \left(\begin{array}{ccc|c}
            -\dfrac{27}{8} & 0 & 0 & 0
            \\
            \dfrac{13}{16} & -\dfrac{1}{2} & -2 & 1
            \\
            1 & 1 & 1 & -1
          \end{array}\right)
          \\
          \\
          \\
        $
        Now can rewrite this in the reduced Row-echelon form:
        \\
        \\
        $
          -\dfrac{8}{27}R_1 \rightarrow R_1
          \\
          \\
          \left(\begin{array}{ccc|c}
            1 & 0 & 0 & 0
            \\
            \dfrac{13}{16} & -\dfrac{1}{2} & -2 & 1
            \\
            1 & 1 & 1 & -1
          \end{array}\right)
          \\
          \\
          \rule{15cm}{1pt}
          \\
          \\
          R_2-\dfrac{13}{16}R_1 \rightarrow R_2 
          \\
          \\
          \left(\begin{array}{ccc|c}
            1 & 0 & 0 & 0
            \\
            0 & -\dfrac{1}{2} & -2 & 1
            \\
            1 & 1 & 1 & -1
          \end{array}\right)
          \\
          \\
          \rule{15cm}{1pt}
          \\
          \\
          R_3-R_1 \rightarrow R_3 
          \\
          \\
          \left(\begin{array}{ccc|c}
            1 & 0 & 0 & 0
            \\
            0 & -\dfrac{1}{2} & -2 & 1
            \\
            0 & 1 & 1 & -1
          \end{array}\right)
          \\
          \\
          \rule{15cm}{1pt}
          \\
          \\
          -2R_2 \rightarrow R_2
          \\
          \\
          \left(\begin{array}{ccc|c}
            1 & 0 & 0 & 0
            \\
            0 & 1 & 4 & -2
            \\
            0 & 1 & 1 & -1
          \end{array}\right)
          \\
          \\
          \rule{15cm}{1pt}
          \\
          \\
          R_3-R_2 \rightarrow R_3
          \\
          \\
          \left(\begin{array}{ccc|c}
            1 & 0 & 0 & 0
            \\
            0 & 1 & 4 & -2
            \\
            0 & 0 & -3 & 1
          \end{array}\right)
          \\
          \\
          \rule{15cm}{1pt}
          \\
          \\
          -\dfrac{1}{3}R_3 \rightarrow R_3
          \\
          \\
          \left(\begin{array}{ccc|c}
            1 & 0 & 0 & 0
            \\
            0 & 1 & 4 & -2
            \\
            0 & 0 & 1 & -\dfrac{1}{3}
          \end{array}\right)
          \\
          \\
          \rule{15cm}{1pt}
          \\
          \\
          R_2-4R_3 \rightarrow R_2
          \\
          \\
          \left(\begin{array}{ccc|c}
            1 & 0 & 0 & 0
            \\
            0 & 1 & 0 & -\dfrac{2}{3}
            \\
            0 & 0 & 1 & -\dfrac{1}{3}
          \end{array}\right)
          \\
          \\
          \\
          \therefore ~~~~ \begin{cases}
            C_1=0
            \\
            \\
            C_2=-\dfrac{2}{3}
            \\
            \\
            C_3=-\dfrac{1}{3}
          \end{cases}
        $
        \\
        \\
        Hence we have:
        \\
        \\
        $
          \overrightarrow{X}=0 e^{-3t} \begin{pmatrix}
            -\dfrac{27}{8}
            \\
            \\
            \dfrac{13}{16}
            \\
            \\
            1
          \end{pmatrix}-\dfrac{2}{3} e^{3t} \begin{pmatrix}
            0
            \\
            \\
            -\dfrac{1}{2}
            \\
            \\
            1
          \end{pmatrix}-\dfrac{1}{3} e^{6t}\begin{pmatrix}
            0
            \\
            -2
            \\
            1
          \end{pmatrix}
          \\
          \\
          \\
          \therefore ~~~~ \overrightarrow{X}=-\dfrac{2}{3} e^{3t} \begin{pmatrix}
            0
            \\
            \\
            -\dfrac{1}{2}
            \\
            \\
            1
          \end{pmatrix}-\dfrac{1}{3} e^{6t}\begin{pmatrix}
            0
            \\
            -2
            \\
            1
          \end{pmatrix}
        $
      }

    \item  (Extra Credit, 15 points) Let
    $$
      A=\begin{pmatrix}
        0 & a_{12} & a_{13} & a_{14}
        \\
        a_{21} & 0 & a_{23} & a_{24}
        \\
        a_{31} & a_{32} & 0 & a_{34}
        \\
        a_{41} & a_{42} & a_{43} & 0
      \end{pmatrix}
      , B=\begin{pmatrix}
        0 & a_{34} & a_{42} & a_{23}
        \\
        a_{43} & 0 & a_{14} & a_{31}
        \\
        a_{24} & a_{41} & 0 & a_{12}
        \\
        a_{32} & a_{13} & a_{21} & 0
      \end{pmatrix}
    $$
    be two antisymmetric matrices, where $a_{ik}=-a_{ki}$ or $A^T=-A$ and $B^T=-B$. Show that 
    $AB=BA$ and present this diagonal matrix as follows
    $$
      BA=AB=\left(a_{32} a_{14}+a_{13} a_{24}+a_{21} a_{34}\right) I
    $$ 
    Where $I$ is the $4 \times 4$ identity matrix. Find $A^{-1}$ and $B^{-1}$. (H. Minkowski, 1908)

      \textcolor{hwColor}{
        \\
        \\
        $
          AB=\begin{pmatrix}
            0 & a_{12} & a_{13} & a_{14}
            \\
            a_{21} & 0 & a_{23} & a_{24}
            \\
            a_{31} & a_{32} & 0 & a_{34}
            \\
            a_{41} & a_{42} & a_{43} & 0
          \end{pmatrix} \begin{pmatrix}
            0 & a_{34} & a_{42} & a_{23}
            \\
            a_{43} & 0 & a_{14} & a_{31}
            \\
            a_{24} & a_{41} & 0 & a_{12}
            \\
            a_{32} & a_{13} & a_{21} & 0
          \end{pmatrix}
          \\
          \\
          \\
          =\begin{pmatrix}
            a_{12}a_{43}+a_{13}a_{24}+a_{14}a_{32} & a_{12}a_{43}+a_{13}a_{24}+a_{14}a_{32} & a_{12}a_{14}+a_{14}a_{21} & a_{12}a_{14}+a_{14}a_{21}
            \\
            \\
            a_{23}a_{24}+a_{24}a_{32} & a_{13}a_{24}+a_{21}a_{34}+a_{23}a_{41} & a_{21}a_{42}+a_{21}a_{24} & a_{21}a_{23}+a_{12}a_{23}
            \\
            \\
            a_{32}a_{43}+a_{32}a_{34} & a_{34}a_{41}+a_{41}a_{43} & a_{41}a_{42}+a_{14}a_{42} & a_{12}a_{43}+a_{23}a_{41}+a_{31}a_{42}
            \\
            \\
            a_{42}a_{43}+a_{24}a_{43} & a_{34}a_{41}+a_{41}a_{43} & a_{41}a_{42}+a_{14}a_{42} & a_{12}a_{43}+a_{23}a_{41}+a_{31}a_{42}
          \end{pmatrix}
          \\
          \\
          \rule{15cm}{2pt}
          \\
          \\
          BA=\begin{pmatrix}
            0 & a_{34} & a_{42} & a_{23}
            \\
            a_{43} & 0 & a_{14} & a_{31}
            \\
            a_{24} & a_{41} & 0 & a_{12}
            \\
            a_{32} & a_{13} & a_{21} & 0
          \end{pmatrix}\begin{pmatrix}
            0 & a_{12} & a_{13} & a_{14}
            \\
            a_{21} & 0 & a_{23} & a_{24}
            \\
            a_{31} & a_{32} & 0 & a_{34}
            \\
            a_{41} & a_{42} & a_{43} & 0
          \end{pmatrix}
          \\
          \\
          \\
          =\begin{pmatrix}
            -a_{12}a_{34}-a_{13}a_{42}-a_{23}a_{14} & 0 & 0 & 0
            \\
            \\
            0 & -a_{12}a_{34}-a_{13}a_{42}-a_{23}a_{14} & 0 & 0
            \\
            \\
            0 & 0 & -a_{12}a_{34}-a_{13}a_{42}-a_{23}a_{14} & 0
            \\
            \\
            0 & 0 & 0 & -a_{12}a_{34}-a_{13}a_{42}-a_{23}a_{14}
          \end{pmatrix}
          \\
          \\
          \\
          =\begin{pmatrix}
            a_{34}a_{21}+a_{42}a_{31}+a_{23}a_{41} & a_{42}a_{32}+a_{42}a_{23} & a_{34}a_{23}+a_{23}a_{43} & a_{34}a_{24}+a_{34}a_{42}
            \\
            \\
            a_{14}a_{31}+a_{31}a_{41} & a_{42}a_{31}+a_{43}a_{12}+a_{14}a_{32} & a_{43}a_{13}+a_{43}a_{31} & a_{43}a_{14}+a_{34}a_{14}
            \\
            \\
            a_{41}a_{21}+a_{41}a_{12} & a_{24}a_{12}+a_{42}a_{12} & a_{23}a_{41}+a_{43}a_{12}+a_{24}a_{13} & a_{14}a_{24}+a_{24}a_{41}
            \\
            \\
            a_{13}a_{21}+a_{31}a_{21} & a_{12}a_{32}+a_{32}a_{21} & a_{32}a_{13}+a_{23}a_{13} & a_{34} a_{21}+a_{14}a_{32}+a_{24}a_{13}
          \end{pmatrix}
          \\
          \\
          \\
          =\begin{pmatrix}
            -a_{12}a_{34}-a_{13}a_{42}-a_{23}a_{14} & 0 & 0 & 0
            \\
            \\
            0 & -a_{12}a_{34}-a_{13}a_{42}-a_{23}a_{14} & 0 & 0
            \\
            \\
            0 & 0 & -a_{12}a_{34}-a_{13}a_{42}-a_{23}a_{14} & 0
            \\
            \\
            0 & 0 & 0 & -a_{12}a_{34}-a_{13}a_{42}-a_{23}a_{14}
          \end{pmatrix}=\left(a_{12}a_{34}+a_{13}a_{42}+a_{23}a_{14}\right) I= \Delta I
          \\
          \\
          \\
          BA= \Delta I
          \\
          \\
          BA.A^{-1}=\Delta I A^{-1} \Longrightarrow A^{-1}=\dfrac{B}{\Delta}
        $
        \\
        \\
        \\
        Similiarly we can show that $B^{-1}=\dfrac{1}{\Delta}A$.
        \\
        \\
        \\
        $
          \therefore ~~~~~ \begin{cases}
            A^{-1}=\dfrac{1}{\Delta}B
            \\
            \\
            B^{-1}=\dfrac{1}{\Delta}A
          \end{cases}
        $
      }
  \end{enumerate}

\end{document}
