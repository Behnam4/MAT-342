\documentclass[fleqn]{article}
\oddsidemargin 0.0in
\textwidth 6.0in
\thispagestyle{empty}
\usepackage{import}
\usepackage{amsmath}
\usepackage{graphicx}
\usepackage{flexisym}
\usepackage{amssymb}
\usepackage{bigints} 
\usepackage[english]{babel}
\usepackage[utf8x]{inputenc}
\usepackage{float}
\usepackage[colorinlistoftodos]{todonotes}

\definecolor{hwColor}{HTML}{AD53BA}

\begin{document}

  \begin{titlepage}

    \newcommand{\HRule}{\rule{\linewidth}{0.5mm}}

    \center


    \textsc{\LARGE Arizona State University}\\[1.5cm]

    \textsc{\LARGE Linear Algebra }\\[1.5cm]


    \begin{figure}
      \includegraphics[width=\linewidth]{asu.png}
    \end{figure}


    \HRule \\[0.4cm]
    { \huge \bfseries Quiz 2}\\[0.4cm] 
    \HRule \\[1.5cm]

    \textbf{Behnam Amiri}

    \bigbreak

    \textbf{Prof: Sergei Suslov}

    \bigbreak


    \textbf{{\large \today}\\[2cm]}

    \vfill

  \end{titlepage}

  \begin{enumerate}
    \item (10 points) Let $A$ and $B$ be $n \times n$ similar matrices, namely, $B=S^{-1} AS$. Show that the matrices $A$ and $B$ have 
    the same characteristic polynomial and,  consequently, the same eigenvalues.

    \item Find the eigenvalues and associated eigenvectors of a given matrix.
    \begin{itemize}
      \item (10 points) $A=\begin{pmatrix}
        2 & -1 
        \\
        -6 & -3
      \end{pmatrix}$

        \textcolor{hwColor}{
          $
            A=\begin{pmatrix}
              2 & -1 
              \\
              -6 & -3
            \end{pmatrix}
            \\
            \\
            \lambda I= \lambda \begin{pmatrix}
              1 & 0
              \\
              0 & 1
            \end{pmatrix}=\begin{pmatrix}
              \lambda & 0
              \\
              0 & \lambda
            \end{pmatrix}
            \\
            \\
            \\
            A-\lambda I=\begin{pmatrix}
              2 & -1 
              \\
              -6 & -3
            \end{pmatrix}-\begin{pmatrix}
              \lambda & 0
              \\
              0 & \lambda
            \end{pmatrix}=\begin{pmatrix}
              2-\lambda & -1
              \\
              -6 & -3-\lambda
            \end{pmatrix}
            \\
            \\
            \\
            det\left(A-\lambda I\right)=\begin{vmatrix}
              2-\lambda & -1
              \\
              -6 & -3-\lambda
            \end{vmatrix}=\left(2-\lambda\right) \left(-3-\lambda\right)- (-6)(-1)
            \\
            \\
            =\lambda^2+\lambda-12=\left(\lambda-3\right) \left(\lambda+4\right)
            \\
            \\
            \\
            det\left(A-\lambda I\right)=0 \Longrightarrow \left(\lambda-3\right) \left(\lambda+4\right)=0 
            \Longrightarrow \begin{cases}
              \lambda=-4
              \\
              \lambda=3
            \end{cases}
          $
          \\
          \\
          \\
          \rule{15cm}{3pt}
          \\
          \\
          When $\lambda=-4$ we have:
          \\
          \\
          $
            A-\lambda I=\begin{pmatrix}
              2-\lambda & -1
              \\
              -6 & -3-\lambda
            \end{pmatrix}=\begin{pmatrix}
              6 & -1
              \\
              -6 & 1
            \end{pmatrix}
            \\
            \\
            \left(A-\lambda I\right)X=0 \Longrightarrow \begin{pmatrix}
              6 & -1
              \\
              -6 & 1
            \end{pmatrix} \begin{pmatrix}
              x_1
              \\
              x_2
            \end{pmatrix}=\begin{pmatrix}
              0
              \\
              0
            \end{pmatrix}
            \\
            \\
            \\
            \left(\begin{array}{cc|c}  
              6 & -1 & 0
              \\  
              -6 & 1 & 0
            \end{array}\right)
            \\
            \\
            \rule{15cm}{1pt}
            \\
            \\
            \dfrac{R_1}{6} \rightarrow R_1
            \\
            \\
            \left(\begin{array}{cc|c}  
              1 & -\dfrac{1}{6} & 0
              \\  
              -6 & 1 & 0
            \end{array}\right)
            \\
            \\
            \rule{15cm}{1pt}
            \\
            \\
            R_2+6R_1 \rightarrow R_2
            \\
            \\
            \left(\begin{array}{cc|c}  
              1 & -\dfrac{1}{6} & 0
              \\  
              0 & 0 & 0
            \end{array}\right)
            \\
            \\
            \\
            \therefore ~~~~ x_1-\dfrac{1}{6}x_2=0 \longrightarrow \begin{cases}
              x_1=\dfrac{1}{6}x_2
              \\
              \\
              x_2=x_2
            \end{cases}
          $
          \\
          \\
          Now let's set $x_2=1$ then we have:
          \\
          \\
          $
            X=\begin{pmatrix}
              \dfrac{1}{6} (1)
              \\
              \\
              1
            \end{pmatrix}=\begin{pmatrix}
              \dfrac{1}{6}
              \\
              \\
              1
            \end{pmatrix}
          $
          \\
          \\
          \rule{15cm}{3pt}
          \\
          \\
          When $\lambda=3$ we have:
          \\
          \\
          $
            A-\lambda I=\begin{pmatrix}
              2-\lambda & -1
              \\
              -6 & -3-\lambda
            \end{pmatrix}=\begin{pmatrix}
              -1 & -1
              \\
              -6 & -6
            \end{pmatrix}
            \\
            \\
            \\
            \left(A-\lambda I\right)X=0 \longrightarrow \begin{pmatrix}
              -1 & -1
              \\
              -6 & -6
            \end{pmatrix} \begin{pmatrix}
              x_1
              \\
              x_2
            \end{pmatrix}=\begin{pmatrix}
              0
              \\
              0
            \end{pmatrix}
            \\
            \\
            \left(\begin{array}{cc|c}  
              -1 & -1 & 0
              \\  
              -6 & -6 & 0
            \end{array}\right)
            \\
            \\
            \rule{15cm}{1pt}
            \\
            \\
            -R_1 \rightarrow R_1
            \\
            \\
            \left(\begin{array}{cc|c}  
              1 & 1 & 0
              \\  
              -6 & -6 & 0
            \end{array}\right)
            \\
            \\
            \rule{15cm}{1pt}
            \\
            \\
            R_2+6R_1 \rightarrow R_2
            \\
            \\
            \left(\begin{array}{cc|c}  
              1 & 1 & 0
              \\  
              0 & 0 & 0
            \end{array}\right)
            \\
            \\
            \\
            \therefore ~~~~ x_1+x_2=0 \Longrightarrow X=\begin{pmatrix}
              -x_2
              \\
              x_2
            \end{pmatrix}
          $
          \\
          \\
          By setting $x_2=1$ we get:
          \\
          \\
          $
            X=\begin{pmatrix}
              -1
              \\
              1
            \end{pmatrix}
          $
          \\
          \\
          Summary:
          \\
          \\
          $
            \therefore ~~~~~ \begin{cases}
              \lambda=-4 \Longrightarrow X=\begin{pmatrix}
                \dfrac{1}{6}
                \\
                \\
                1
              \end{pmatrix}
              \\
              \\
              \lambda=3 \Longrightarrow X=\begin{pmatrix}
                -1
                \\
                1
              \end{pmatrix}
            \end{cases}
            \\
            \\
          $
        }
  
      \item (15 points) $A=\begin{pmatrix}
        -3 & 0 & 0
        \\
        3 & 7 & 2
        \\
        1 & -2 & 2
      \end{pmatrix}$
    \end{itemize}

    \item (15 points) Solve the following initial value problem:
    $$
      \dfrac{d}{dx} X=\begin{pmatrix}
        -3 & 0 & 0
        \\
        3 & 7 & 2 
        \\
        1 & -2 & 2
      \end{pmatrix} X, ~~~~~~~ X(0)=\begin{pmatrix}
        0 
        \\
        1
        \\
        -1
      \end{pmatrix}
    $$

    \item  (Extra Credit, 15 points) Let
    $$
      A=\begin{pmatrix}
        0 & a_{12} & a_{13} & a_{14}
        \\
        a_{21} & 0 & a_{23} & a_{24}
        \\
        a_{31} & a_{32} & 0 & a_{34}
        \\
        a_{41} & a_{42} & a_{43} & 0
      \end{pmatrix}
      , B=\begin{pmatrix}
        0 & a_{34} & a_{42} & a_{23}
        \\
        a_{43} & 0 & a_{14} & a_{31}
        \\
        a_{24} & a_{41} & 0 & a_{12}
        \\
        a_{32} & a_{13} & a_{21} & 0
      \end{pmatrix}
    $$
    be two antisymmetric matrices, where $a_{ik}=-a_{ki}$ or $A^T=-A$ and $B^T=-B$. Show that 
    $AB=BA$ and present this diagonal matrix as follows
    $$
      BA=AB=\left(a_{32} a_{14}+a_{13} a_{24}+a_{21} a_{34}\right) I
    $$ 
    Where $I$ is the $4 \times 4$ identity matrix. Find $A^{-1}$ and $B^{-1}$. (H. Minkowski, 1908)
  \end{enumerate}

\end{document}
